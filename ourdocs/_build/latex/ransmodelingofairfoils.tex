%% Generated by Sphinx.
\def\sphinxdocclass{report}
\documentclass[letterpaper,10pt,english]{sphinxmanual}
\ifdefined\pdfpxdimen
   \let\sphinxpxdimen\pdfpxdimen\else\newdimen\sphinxpxdimen
\fi \sphinxpxdimen=.75bp\relax
\ifdefined\pdfimageresolution
    \pdfimageresolution= \numexpr \dimexpr1in\relax/\sphinxpxdimen\relax
\fi
%% let collapsable pdf bookmarks panel have high depth per default
\PassOptionsToPackage{bookmarksdepth=5}{hyperref}

\PassOptionsToPackage{warn}{textcomp}
\usepackage[utf8]{inputenc}
\ifdefined\DeclareUnicodeCharacter
% support both utf8 and utf8x syntaxes
  \ifdefined\DeclareUnicodeCharacterAsOptional
    \def\sphinxDUC#1{\DeclareUnicodeCharacter{"#1}}
  \else
    \let\sphinxDUC\DeclareUnicodeCharacter
  \fi
  \sphinxDUC{00A0}{\nobreakspace}
  \sphinxDUC{2500}{\sphinxunichar{2500}}
  \sphinxDUC{2502}{\sphinxunichar{2502}}
  \sphinxDUC{2514}{\sphinxunichar{2514}}
  \sphinxDUC{251C}{\sphinxunichar{251C}}
  \sphinxDUC{2572}{\textbackslash}
\fi
\usepackage{cmap}
\usepackage[T1]{fontenc}
\usepackage{amsmath,amssymb,amstext}
\usepackage{babel}



\usepackage{tgtermes}
\usepackage{tgheros}
\renewcommand{\ttdefault}{txtt}



\usepackage[Bjarne]{fncychap}
\usepackage{sphinx}

\fvset{fontsize=auto}
\usepackage{geometry}


% Include hyperref last.
\usepackage{hyperref}
% Fix anchor placement for figures with captions.
\usepackage{hypcap}% it must be loaded after hyperref.
% Set up styles of URL: it should be placed after hyperref.
\urlstyle{same}

\addto\captionsenglish{\renewcommand{\contentsname}{Contents:}}

\usepackage{sphinxmessages}
\setcounter{tocdepth}{1}



\title{RANS Modeling of Airfoils}
\date{Dec 20, 2021}
\release{1.0}
\author{Alex Taylor\sphinxhyphen{}Lash, Andy Rothstein, Brian Wynne, Nick Conlin, Satya Butler, Vedin Dewan}
\newcommand{\sphinxlogo}{\vbox{}}
\renewcommand{\releasename}{Release}
\makeindex
\begin{document}

\pagestyle{empty}
\sphinxmaketitle
\pagestyle{plain}
\sphinxtableofcontents
\pagestyle{normal}
\phantomsection\label{\detokenize{index::doc}}



\chapter{API Reference}
\label{\detokenize{autoapi/index:api-reference}}\label{\detokenize{autoapi/index::doc}}
\sphinxAtStartPar
This page contains auto\sphinxhyphen{}generated API reference documentation %
\begin{footnote}[1]\sphinxAtStartFootnote
Created with \sphinxhref{https://github.com/readthedocs/sphinx-autoapi}{sphinx\sphinxhyphen{}autoapi}
%
\end{footnote}.


\section{\sphinxstyleliteralintitle{\sphinxupquote{AirfoilMap}}}
\label{\detokenize{autoapi/AirfoilMap/index:module-AirfoilMap}}\label{\detokenize{autoapi/AirfoilMap/index:airfoilmap}}\label{\detokenize{autoapi/AirfoilMap/index::doc}}\index{module@\spxentry{module}!AirfoilMap@\spxentry{AirfoilMap}}\index{AirfoilMap@\spxentry{AirfoilMap}!module@\spxentry{module}}
\sphinxAtStartPar
This module creates am AirfoilMap object containing x,xc and vol Field objects
\begin{description}
\item[{Libraries/Modules:}] \leavevmode
\sphinxAtStartPar
numpy

\sphinxAtStartPar
typing

\sphinxAtStartPar
Field

\sphinxAtStartPar
Grid

\sphinxAtStartPar
Input

\sphinxAtStartPar
airfoil\_map

\end{description}


\subsection{Module Contents}
\label{\detokenize{autoapi/AirfoilMap/index:module-contents}}

\subsubsection{Classes}
\label{\detokenize{autoapi/AirfoilMap/index:classes}}

\begin{savenotes}\sphinxatlongtablestart\begin{longtable}[c]{\X{1}{2}\X{1}{2}}
\hline

\endfirsthead

\multicolumn{2}{c}%
{\makebox[0pt]{\sphinxtablecontinued{\tablename\ \thetable{} \textendash{} continued from previous page}}}\\
\hline

\endhead

\hline
\multicolumn{2}{r}{\makebox[0pt][r]{\sphinxtablecontinued{continues on next page}}}\\
\endfoot

\endlastfoot

\sphinxAtStartPar
{\hyperref[\detokenize{autoapi/AirfoilMap/index:AirfoilMap.AirfoilMap}]{\sphinxcrossref{\sphinxcode{\sphinxupquote{AirfoilMap}}}}}
&
\sphinxAtStartPar
Creates Airfoil map object containing x,xc and vol as Field objects.
\\
\hline
\end{longtable}\sphinxatlongtableend\end{savenotes}
\index{AirfoilMap (class in AirfoilMap)@\spxentry{AirfoilMap}\spxextra{class in AirfoilMap}}

\begin{fulllineitems}
\phantomsection\label{\detokenize{autoapi/AirfoilMap/index:AirfoilMap.AirfoilMap}}\pysiglinewithargsret{\sphinxbfcode{\sphinxupquote{class }}\sphinxcode{\sphinxupquote{AirfoilMap.}}\sphinxbfcode{\sphinxupquote{AirfoilMap}}}{\emph{\DUrole{n}{num\_divisions}}}{}
\sphinxAtStartPar
Bases: \sphinxcode{\sphinxupquote{bin.Grid.Grid}}

\sphinxAtStartPar
Creates Airfoil map object containing x,xc and vol as Field objects.
\begin{description}
\item[{Constructor (not intended to be implemented directly):}] \leavevmode\begin{description}
\item[{Args:}] \leavevmode
\sphinxAtStartPar
num\_divisions (list):Number of cells in the x and y directions.

\item[{Returns:}] \leavevmode
\sphinxAtStartPar
A new AirfoilMap object.

\item[{Notes:}] \leavevmode
\sphinxAtStartPar
Directly use from\_file() method to perform confromal mapping
and use from\_grid() method to convert grid to coarser version

\end{description}

\end{description}
\index{from\_file() (AirfoilMap.AirfoilMap class method)@\spxentry{from\_file()}\spxextra{AirfoilMap.AirfoilMap class method}}

\begin{fulllineitems}
\phantomsection\label{\detokenize{autoapi/AirfoilMap/index:AirfoilMap.AirfoilMap.from_file}}\pysiglinewithargsret{\sphinxbfcode{\sphinxupquote{classmethod }}\sphinxbfcode{\sphinxupquote{from\_file}}}{\emph{\DUrole{n}{thisClass}}, \emph{\DUrole{n}{num\_divisions}}, \emph{\DUrole{n}{input}}}{}
\sphinxAtStartPar
Initializes new AirfoilMap from datafile input.
\begin{quote}\begin{description}
\item[{Parameters}] \leavevmode\begin{itemize}
\item {} 
\sphinxAtStartPar
\sphinxstyleliteralstrong{\sphinxupquote{num\_divisions}} (\sphinxstyleliteralemphasis{\sphinxupquote{list}}) \textendash{} Number of cells in the x and y directions.

\item {} 
\sphinxAtStartPar
\sphinxstyleliteralstrong{\sphinxupquote{input}} (\sphinxstyleliteralemphasis{\sphinxupquote{dict}}) \textendash{} Dictionary containing data\sphinxhyphen{}file values

\end{itemize}

\item[{Returns}] \leavevmode
\sphinxAtStartPar
new AirfoilMap object

\item[{Return type}] \leavevmode
\sphinxAtStartPar
grid (obj)

\end{description}\end{quote}

\end{fulllineitems}

\index{from\_grid() (AirfoilMap.AirfoilMap class method)@\spxentry{from\_grid()}\spxextra{AirfoilMap.AirfoilMap class method}}

\begin{fulllineitems}
\phantomsection\label{\detokenize{autoapi/AirfoilMap/index:AirfoilMap.AirfoilMap.from_grid}}\pysiglinewithargsret{\sphinxbfcode{\sphinxupquote{classmethod }}\sphinxbfcode{\sphinxupquote{from\_grid}}}{\emph{\DUrole{n}{thisClass}}, \emph{\DUrole{n}{grid}}}{}
\sphinxAtStartPar
Initializes new AirfoilMap from existing object. The new grid will be half the size.
\begin{quote}\begin{description}
\item[{Parameters}] \leavevmode
\sphinxAtStartPar
\sphinxstyleliteralstrong{\sphinxupquote{grid}} (\sphinxstyleliteralemphasis{\sphinxupquote{obj}}) \textendash{} AirfoilMap object

\item[{Returns}] \leavevmode
\sphinxAtStartPar
new AirfoilMap object

\item[{Return type}] \leavevmode
\sphinxAtStartPar
newGrid (obj)

\end{description}\end{quote}

\end{fulllineitems}

\index{get\_dims() (AirfoilMap.AirfoilMap method)@\spxentry{get\_dims()}\spxextra{AirfoilMap.AirfoilMap method}}

\begin{fulllineitems}
\phantomsection\label{\detokenize{autoapi/AirfoilMap/index:AirfoilMap.AirfoilMap.get_dims}}\pysiglinewithargsret{\sphinxbfcode{\sphinxupquote{get\_dims}}}{\emph{\DUrole{n}{self}}}{}
\sphinxAtStartPar
Gets dimensions of grid

\end{fulllineitems}

\index{get\_geometry() (AirfoilMap.AirfoilMap method)@\spxentry{get\_geometry()}\spxextra{AirfoilMap.AirfoilMap method}}

\begin{fulllineitems}
\phantomsection\label{\detokenize{autoapi/AirfoilMap/index:AirfoilMap.AirfoilMap.get_geometry}}\pysiglinewithargsret{\sphinxbfcode{\sphinxupquote{get\_geometry}}}{\emph{\DUrole{n}{self}}}{}
\sphinxAtStartPar
Gets geometry

\end{fulllineitems}

\index{get\_size() (AirfoilMap.AirfoilMap method)@\spxentry{get\_size()}\spxextra{AirfoilMap.AirfoilMap method}}

\begin{fulllineitems}
\phantomsection\label{\detokenize{autoapi/AirfoilMap/index:AirfoilMap.AirfoilMap.get_size}}\pysiglinewithargsret{\sphinxbfcode{\sphinxupquote{get\_size}}}{\emph{\DUrole{n}{self}}}{}
\sphinxAtStartPar
Gets size

\end{fulllineitems}


\end{fulllineitems}



\section{\sphinxstyleliteralintitle{\sphinxupquote{BoundaryConditioner}}}
\label{\detokenize{autoapi/BoundaryConditioner/index:module-BoundaryConditioner}}\label{\detokenize{autoapi/BoundaryConditioner/index:boundaryconditioner}}\label{\detokenize{autoapi/BoundaryConditioner/index::doc}}\index{module@\spxentry{module}!BoundaryConditioner@\spxentry{BoundaryConditioner}}\index{BoundaryConditioner@\spxentry{BoundaryConditioner}!module@\spxentry{module}}
\sphinxAtStartPar
This module contains an abstract base class Grid


\subsection{Module Contents}
\label{\detokenize{autoapi/BoundaryConditioner/index:module-contents}}

\subsubsection{Classes}
\label{\detokenize{autoapi/BoundaryConditioner/index:classes}}

\begin{savenotes}\sphinxatlongtablestart\begin{longtable}[c]{\X{1}{2}\X{1}{2}}
\hline

\endfirsthead

\multicolumn{2}{c}%
{\makebox[0pt]{\sphinxtablecontinued{\tablename\ \thetable{} \textendash{} continued from previous page}}}\\
\hline

\endhead

\hline
\multicolumn{2}{r}{\makebox[0pt][r]{\sphinxtablecontinued{continues on next page}}}\\
\endfoot

\endlastfoot

\sphinxAtStartPar
{\hyperref[\detokenize{autoapi/BoundaryConditioner/index:BoundaryConditioner.BoundaryConditioner}]{\sphinxcrossref{\sphinxcode{\sphinxupquote{BoundaryConditioner}}}}}
&
\sphinxAtStartPar
Abstract base class, never directly instantiated
\\
\hline
\end{longtable}\sphinxatlongtableend\end{savenotes}
\index{BoundaryConditioner (class in BoundaryConditioner)@\spxentry{BoundaryConditioner}\spxextra{class in BoundaryConditioner}}

\begin{fulllineitems}
\phantomsection\label{\detokenize{autoapi/BoundaryConditioner/index:BoundaryConditioner.BoundaryConditioner}}\pysiglinewithargsret{\sphinxbfcode{\sphinxupquote{class }}\sphinxcode{\sphinxupquote{BoundaryConditioner.}}\sphinxbfcode{\sphinxupquote{BoundaryConditioner}}}{\emph{\DUrole{n}{input}}}{}
\sphinxAtStartPar
Bases: \sphinxcode{\sphinxupquote{abc.ABC}}

\sphinxAtStartPar
Abstract base class, never directly instantiated

\sphinxAtStartPar
NS\_Airfoil is a child class of this ABC
\index{update\_stability() (BoundaryConditioner.BoundaryConditioner method)@\spxentry{update\_stability()}\spxextra{BoundaryConditioner.BoundaryConditioner method}}

\begin{fulllineitems}
\phantomsection\label{\detokenize{autoapi/BoundaryConditioner/index:BoundaryConditioner.BoundaryConditioner.update_stability}}\pysiglinewithargsret{\sphinxbfcode{\sphinxupquote{abstract }}\sphinxbfcode{\sphinxupquote{update\_stability}}}{\emph{\DUrole{n}{self}}, \emph{\DUrole{n}{model}}, \emph{\DUrole{n}{workspace}}, \emph{\DUrole{n}{state}}}{}
\sphinxAtStartPar
updates stability parameters for time step calculations
\begin{quote}\begin{description}
\item[{Parameters}] \leavevmode\begin{itemize}
\item {} 
\sphinxAtStartPar
\sphinxstyleliteralstrong{\sphinxupquote{model}} \textendash{} instance of class inheriting from Model

\item {} 
\sphinxAtStartPar
\sphinxstyleliteralstrong{\sphinxupquote{workspace}} \textendash{} instance of Workspace class (or child)

\item {} 
\sphinxAtStartPar
\sphinxstyleliteralstrong{\sphinxupquote{state}} ({\hyperref[\detokenize{autoapi/Field/index:Field.Field}]{\sphinxcrossref{\sphinxstyleliteralemphasis{\sphinxupquote{Field}}}}}) \textendash{} current state of the system (density, momentum, energy)

\end{itemize}

\end{description}\end{quote}

\end{fulllineitems}

\index{update\_physics() (BoundaryConditioner.BoundaryConditioner method)@\spxentry{update\_physics()}\spxextra{BoundaryConditioner.BoundaryConditioner method}}

\begin{fulllineitems}
\phantomsection\label{\detokenize{autoapi/BoundaryConditioner/index:BoundaryConditioner.BoundaryConditioner.update_physics}}\pysiglinewithargsret{\sphinxbfcode{\sphinxupquote{abstract }}\sphinxbfcode{\sphinxupquote{update\_physics}}}{\emph{\DUrole{n}{self}}, \emph{\DUrole{n}{model}}, \emph{\DUrole{n}{workspace}}, \emph{\DUrole{n}{state}}}{}
\sphinxAtStartPar
updates physical parameters for calculation of boundary conditions
\begin{quote}\begin{description}
\item[{Parameters}] \leavevmode\begin{itemize}
\item {} 
\sphinxAtStartPar
\sphinxstyleliteralstrong{\sphinxupquote{model}} \textendash{} instance of class inheriting from Model

\item {} 
\sphinxAtStartPar
\sphinxstyleliteralstrong{\sphinxupquote{workspace}} \textendash{} instance of Workspace class (or child)

\item {} 
\sphinxAtStartPar
\sphinxstyleliteralstrong{\sphinxupquote{state}} ({\hyperref[\detokenize{autoapi/Field/index:Field.Field}]{\sphinxcrossref{\sphinxstyleliteralemphasis{\sphinxupquote{Field}}}}}) \textendash{} current state of the system (density, momentum, energy)

\end{itemize}

\end{description}\end{quote}

\end{fulllineitems}

\index{bc\_wall() (BoundaryConditioner.BoundaryConditioner method)@\spxentry{bc\_wall()}\spxextra{BoundaryConditioner.BoundaryConditioner method}}

\begin{fulllineitems}
\phantomsection\label{\detokenize{autoapi/BoundaryConditioner/index:BoundaryConditioner.BoundaryConditioner.bc_wall}}\pysiglinewithargsret{\sphinxbfcode{\sphinxupquote{abstract }}\sphinxbfcode{\sphinxupquote{bc\_wall}}}{\emph{\DUrole{n}{self}}, \emph{\DUrole{n}{model}}, \emph{\DUrole{n}{workspace}}, \emph{\DUrole{n}{state}}}{}
\sphinxAtStartPar
apply boundary condition along the wall
\begin{quote}
\begin{description}
\item[{Args:}] \leavevmode
\sphinxAtStartPar
model: instance of class inheriting from Model
workspace: instance of Workspace class (or child)
state (Field): current state of the system (density, momentum, energy)

\end{description}
\end{quote}

\end{fulllineitems}

\index{bc\_far() (BoundaryConditioner.BoundaryConditioner method)@\spxentry{bc\_far()}\spxextra{BoundaryConditioner.BoundaryConditioner method}}

\begin{fulllineitems}
\phantomsection\label{\detokenize{autoapi/BoundaryConditioner/index:BoundaryConditioner.BoundaryConditioner.bc_far}}\pysiglinewithargsret{\sphinxbfcode{\sphinxupquote{abstract }}\sphinxbfcode{\sphinxupquote{bc\_far}}}{\emph{\DUrole{n}{self}}, \emph{\DUrole{n}{model}}, \emph{\DUrole{n}{workspace}}, \emph{\DUrole{n}{state}}}{}
\sphinxAtStartPar
apply boundary condition in the far field
\begin{quote}\begin{description}
\item[{Parameters}] \leavevmode\begin{itemize}
\item {} 
\sphinxAtStartPar
\sphinxstyleliteralstrong{\sphinxupquote{model}} \textendash{} instance of class inheriting from Model

\item {} 
\sphinxAtStartPar
\sphinxstyleliteralstrong{\sphinxupquote{workspace}} \textendash{} instance of Workspace class (or child)

\item {} 
\sphinxAtStartPar
\sphinxstyleliteralstrong{\sphinxupquote{state}} ({\hyperref[\detokenize{autoapi/Field/index:Field.Field}]{\sphinxcrossref{\sphinxstyleliteralemphasis{\sphinxupquote{Field}}}}}) \textendash{} current state of the system (density, momentum, energy)

\end{itemize}

\end{description}\end{quote}

\end{fulllineitems}

\index{halo() (BoundaryConditioner.BoundaryConditioner method)@\spxentry{halo()}\spxextra{BoundaryConditioner.BoundaryConditioner method}}

\begin{fulllineitems}
\phantomsection\label{\detokenize{autoapi/BoundaryConditioner/index:BoundaryConditioner.BoundaryConditioner.halo}}\pysiglinewithargsret{\sphinxbfcode{\sphinxupquote{abstract }}\sphinxbfcode{\sphinxupquote{halo}}}{\emph{\DUrole{n}{self}}, \emph{\DUrole{n}{model}}, \emph{\DUrole{n}{workspace}}, \emph{\DUrole{n}{state}}}{}
\sphinxAtStartPar
set the values in the halo
\begin{quote}
\begin{description}
\item[{Args:}] \leavevmode
\sphinxAtStartPar
model: instance of class inheriting from Model
workspace: instance of Workspace class (or child)
state (Field): current state of the system (density, momentum, energy)

\end{description}
\end{quote}

\end{fulllineitems}

\index{bc\_all() (BoundaryConditioner.BoundaryConditioner method)@\spxentry{bc\_all()}\spxextra{BoundaryConditioner.BoundaryConditioner method}}

\begin{fulllineitems}
\phantomsection\label{\detokenize{autoapi/BoundaryConditioner/index:BoundaryConditioner.BoundaryConditioner.bc_all}}\pysiglinewithargsret{\sphinxbfcode{\sphinxupquote{abstract }}\sphinxbfcode{\sphinxupquote{bc\_all}}}{\emph{\DUrole{n}{self}}, \emph{\DUrole{n}{model}}, \emph{\DUrole{n}{workspace}}, \emph{\DUrole{n}{state}}}{}
\sphinxAtStartPar
do wall boundaries, far field and set halo values at once
\begin{quote}
\begin{description}
\item[{Args:}] \leavevmode
\sphinxAtStartPar
model: instance of class inheriting from Model
workspace: instance of Workspace class (or child)
state (Field): current state of the system (density, momentum, energy)

\end{description}
\end{quote}

\end{fulllineitems}

\index{transfer\_down() (BoundaryConditioner.BoundaryConditioner method)@\spxentry{transfer\_down()}\spxextra{BoundaryConditioner.BoundaryConditioner method}}

\begin{fulllineitems}
\phantomsection\label{\detokenize{autoapi/BoundaryConditioner/index:BoundaryConditioner.BoundaryConditioner.transfer_down}}\pysiglinewithargsret{\sphinxbfcode{\sphinxupquote{abstract }}\sphinxbfcode{\sphinxupquote{transfer\_down}}}{\emph{\DUrole{n}{self}}, \emph{\DUrole{n}{model}}, \emph{\DUrole{n}{workspace1}}, \emph{\DUrole{n}{workspace2}}}{}
\sphinxAtStartPar
transfer boundary information between workspaces
\begin{quote}
\begin{description}
\item[{Args:}] \leavevmode
\sphinxAtStartPar
model: instance of class inheriting from Model
workspace: instance of Workspace class (or child)
state (Field): current state of the system (density, momentum, energy)

\end{description}
\end{quote}

\end{fulllineitems}


\end{fulllineitems}



\section{\sphinxstyleliteralintitle{\sphinxupquote{CellCenterWS}}}
\label{\detokenize{autoapi/CellCenterWS/index:module-CellCenterWS}}\label{\detokenize{autoapi/CellCenterWS/index:cellcenterws}}\label{\detokenize{autoapi/CellCenterWS/index::doc}}\index{module@\spxentry{module}!CellCenterWS@\spxentry{CellCenterWS}}\index{CellCenterWS@\spxentry{CellCenterWS}!module@\spxentry{module}}
\sphinxAtStartPar
This module is an inherited class of workspace.
Libraries/Modules:
\begin{quote}

\sphinxAtStartPar
Workspace

\sphinxAtStartPar
numpy

\sphinxAtStartPar
Field
\end{quote}


\subsection{Module Contents}
\label{\detokenize{autoapi/CellCenterWS/index:module-contents}}

\subsubsection{Classes}
\label{\detokenize{autoapi/CellCenterWS/index:classes}}

\begin{savenotes}\sphinxatlongtablestart\begin{longtable}[c]{\X{1}{2}\X{1}{2}}
\hline

\endfirsthead

\multicolumn{2}{c}%
{\makebox[0pt]{\sphinxtablecontinued{\tablename\ \thetable{} \textendash{} continued from previous page}}}\\
\hline

\endhead

\hline
\multicolumn{2}{r}{\makebox[0pt][r]{\sphinxtablecontinued{continues on next page}}}\\
\endfoot

\endlastfoot

\sphinxAtStartPar
{\hyperref[\detokenize{autoapi/CellCenterWS/index:CellCenterWS.CellCenterWS}]{\sphinxcrossref{\sphinxcode{\sphinxupquote{CellCenterWS}}}}}
&
\sphinxAtStartPar
Implements a Workspace using cell\sphinxhyphen{}centered discretization of the grid
\\
\hline
\end{longtable}\sphinxatlongtableend\end{savenotes}
\index{CellCenterWS (class in CellCenterWS)@\spxentry{CellCenterWS}\spxextra{class in CellCenterWS}}

\begin{fulllineitems}
\phantomsection\label{\detokenize{autoapi/CellCenterWS/index:CellCenterWS.CellCenterWS}}\pysigline{\sphinxbfcode{\sphinxupquote{class }}\sphinxcode{\sphinxupquote{CellCenterWS.}}\sphinxbfcode{\sphinxupquote{CellCenterWS}}}
\sphinxAtStartPar
Bases: \sphinxcode{\sphinxupquote{bin.Workspace.Workspace}}

\sphinxAtStartPar
Implements a Workspace using cell\sphinxhyphen{}centered discretization of the grid
\begin{description}
\item[{Constructor:}] \leavevmode\begin{description}
\item[{Args:}] \leavevmode
\sphinxAtStartPar
grid (Grid): a Grid object specifying the geometry

\end{description}

\end{description}
\index{grid (CellCenterWS.CellCenterWS.self attribute)@\spxentry{grid}\spxextra{CellCenterWS.CellCenterWS.self attribute}}

\begin{fulllineitems}
\phantomsection\label{\detokenize{autoapi/CellCenterWS/index:CellCenterWS.CellCenterWS.self.grid}}\pysigline{\sphinxcode{\sphinxupquote{self.}}\sphinxbfcode{\sphinxupquote{grid}}}
\sphinxAtStartPar
Inputted grid

\end{fulllineitems}

\index{flds (CellCenterWS.CellCenterWS.self attribute)@\spxentry{flds}\spxextra{CellCenterWS.CellCenterWS.self attribute}}

\begin{fulllineitems}
\phantomsection\label{\detokenize{autoapi/CellCenterWS/index:CellCenterWS.CellCenterWS.self.flds}}\pysigline{\sphinxcode{\sphinxupquote{self.}}\sphinxbfcode{\sphinxupquote{flds}}}
\sphinxAtStartPar
Dictionary of fields residing on grid

\end{fulllineitems}

\index{make\_new() (CellCenterWS.CellCenterWS method)@\spxentry{make\_new()}\spxextra{CellCenterWS.CellCenterWS method}}

\begin{fulllineitems}
\phantomsection\label{\detokenize{autoapi/CellCenterWS/index:CellCenterWS.CellCenterWS.make_new}}\pysiglinewithargsret{\sphinxbfcode{\sphinxupquote{make\_new}}}{\emph{\DUrole{n}{self}}, \emph{\DUrole{n}{grid}}}{}
\sphinxAtStartPar
Creates a new workspace corresponding to a grid of half the size
\begin{quote}\begin{description}
\item[{Parameters}] \leavevmode\begin{itemize}
\item {} 
\sphinxAtStartPar
\sphinxstyleliteralstrong{\sphinxupquote{grid}} ({\hyperref[\detokenize{autoapi/Grid/index:Grid.Grid}]{\sphinxcrossref{\sphinxstyleliteralemphasis{\sphinxupquote{Grid}}}}}) \textendash{} The grid for the current workspace

\item {} 
\sphinxAtStartPar
\sphinxstyleliteralstrong{\sphinxupquote{isFinest}} (\sphinxstyleliteralemphasis{\sphinxupquote{bool}}) \textendash{} Whether or not this is

\end{itemize}

\end{description}\end{quote}

\end{fulllineitems}

\index{field\_size() (CellCenterWS.CellCenterWS method)@\spxentry{field\_size()}\spxextra{CellCenterWS.CellCenterWS method}}

\begin{fulllineitems}
\phantomsection\label{\detokenize{autoapi/CellCenterWS/index:CellCenterWS.CellCenterWS.field_size}}\pysiglinewithargsret{\sphinxbfcode{\sphinxupquote{field\_size}}}{\emph{\DUrole{n}{self}}}{}
\sphinxAtStartPar
Returns the 2\sphinxhyphen{}dimenstional size of the field \_\_\textgreater{} (n, 1) for a 1\sphinxhyphen{}d Field

\end{fulllineitems}

\index{edges() (CellCenterWS.CellCenterWS method)@\spxentry{edges()}\spxextra{CellCenterWS.CellCenterWS method}}

\begin{fulllineitems}
\phantomsection\label{\detokenize{autoapi/CellCenterWS/index:CellCenterWS.CellCenterWS.edges}}\pysiglinewithargsret{\sphinxbfcode{\sphinxupquote{edges}}}{\emph{\DUrole{n}{self}}, \emph{\DUrole{n}{dim}}}{}
\sphinxAtStartPar
Returns a Field containing the edge vectors
\begin{quote}\begin{description}
\item[{Parameters}] \leavevmode
\sphinxAtStartPar
\sphinxstyleliteralstrong{\sphinxupquote{dim}} (\sphinxstyleliteralemphasis{\sphinxupquote{0}}\sphinxstyleliteralemphasis{\sphinxupquote{ or }}\sphinxstyleliteralemphasis{\sphinxupquote{1}}) \textendash{} Which edges will be returned (0 for i, 1 for j edges)

\end{description}\end{quote}

\end{fulllineitems}

\index{edge\_normals() (CellCenterWS.CellCenterWS method)@\spxentry{edge\_normals()}\spxextra{CellCenterWS.CellCenterWS method}}

\begin{fulllineitems}
\phantomsection\label{\detokenize{autoapi/CellCenterWS/index:CellCenterWS.CellCenterWS.edge_normals}}\pysiglinewithargsret{\sphinxbfcode{\sphinxupquote{edge\_normals}}}{\emph{\DUrole{n}{self}}, \emph{\DUrole{n}{dim}}}{}
\sphinxAtStartPar
Returns a Field containing the unit normal vectors to the edges along a given dimension
\begin{quote}\begin{description}
\item[{Parameters}] \leavevmode
\sphinxAtStartPar
\sphinxstyleliteralstrong{\sphinxupquote{dim}} (\sphinxstyleliteralemphasis{\sphinxupquote{0}}\sphinxstyleliteralemphasis{\sphinxupquote{ or }}\sphinxstyleliteralemphasis{\sphinxupquote{1}}) \textendash{} Which edges normals will be returned for (0 for i, 1 for j edges)

\end{description}\end{quote}

\end{fulllineitems}

\index{\_\_calc\_edges() (CellCenterWS.CellCenterWS method)@\spxentry{\_\_calc\_edges()}\spxextra{CellCenterWS.CellCenterWS method}}

\begin{fulllineitems}
\phantomsection\label{\detokenize{autoapi/CellCenterWS/index:CellCenterWS.CellCenterWS.__calc_edges}}\pysiglinewithargsret{\sphinxbfcode{\sphinxupquote{\_\_calc\_edges}}}{\emph{\DUrole{n}{self}}, \emph{\DUrole{n}{dim}}}{}
\end{fulllineitems}

\index{\_\_calc\_normals() (CellCenterWS.CellCenterWS method)@\spxentry{\_\_calc\_normals()}\spxextra{CellCenterWS.CellCenterWS method}}

\begin{fulllineitems}
\phantomsection\label{\detokenize{autoapi/CellCenterWS/index:CellCenterWS.CellCenterWS.__calc_normals}}\pysiglinewithargsret{\sphinxbfcode{\sphinxupquote{\_\_calc\_normals}}}{\emph{\DUrole{n}{self}}, \emph{\DUrole{n}{dim}}}{}
\end{fulllineitems}


\end{fulllineitems}



\section{\sphinxstyleliteralintitle{\sphinxupquote{Contractinator}}}
\label{\detokenize{autoapi/Contractinator/index:module-Contractinator}}\label{\detokenize{autoapi/Contractinator/index:contractinator}}\label{\detokenize{autoapi/Contractinator/index::doc}}\index{module@\spxentry{module}!Contractinator@\spxentry{Contractinator}}\index{Contractinator@\spxentry{Contractinator}!module@\spxentry{module}}
\sphinxAtStartPar
Description

\sphinxAtStartPar
Contracts Field objects from finer meshes to coarser meshes.

\sphinxAtStartPar
Libraries/Modules

\sphinxAtStartPar
bin.Field

\sphinxAtStartPar
numpy


\subsection{Module Contents}
\label{\detokenize{autoapi/Contractinator/index:module-contents}}

\subsubsection{Functions}
\label{\detokenize{autoapi/Contractinator/index:functions}}

\begin{savenotes}\sphinxatlongtablestart\begin{longtable}[c]{\X{1}{2}\X{1}{2}}
\hline

\endfirsthead

\multicolumn{2}{c}%
{\makebox[0pt]{\sphinxtablecontinued{\tablename\ \thetable{} \textendash{} continued from previous page}}}\\
\hline

\endhead

\hline
\multicolumn{2}{r}{\makebox[0pt][r]{\sphinxtablecontinued{continues on next page}}}\\
\endfoot

\endlastfoot

\sphinxAtStartPar
{\hyperref[\detokenize{autoapi/Contractinator/index:Contractinator.simple}]{\sphinxcrossref{\sphinxcode{\sphinxupquote{simple}}}}}(fine, coarse)
&
\sphinxAtStartPar
Performs a simple contraction where every other value is deleted.
\\
\hline
\sphinxAtStartPar
{\hyperref[\detokenize{autoapi/Contractinator/index:Contractinator.sum4way}]{\sphinxcrossref{\sphinxcode{\sphinxupquote{sum4way}}}}}(fine, coarse)
&
\sphinxAtStartPar
Contracts the Field by summing 4 values into 1. Only setup for 2D Fields (?) so far.
\\
\hline
\sphinxAtStartPar
{\hyperref[\detokenize{autoapi/Contractinator/index:Contractinator.conservative4way}]{\sphinxcrossref{\sphinxcode{\sphinxupquote{conservative4way}}}}}(fine, coarse, weights=None)
&
\sphinxAtStartPar
Contracts the Field by averaging 4 values into 1 with weighting terms. Only set up for 2D Fields (?) so far.
\\
\hline
\end{longtable}\sphinxatlongtableend\end{savenotes}
\index{simple() (in module Contractinator)@\spxentry{simple()}\spxextra{in module Contractinator}}

\begin{fulllineitems}
\phantomsection\label{\detokenize{autoapi/Contractinator/index:Contractinator.simple}}\pysiglinewithargsret{\sphinxcode{\sphinxupquote{Contractinator.}}\sphinxbfcode{\sphinxupquote{simple}}}{\emph{\DUrole{n}{fine}}, \emph{\DUrole{n}{coarse}}}{}
\sphinxAtStartPar
Performs a simple contraction where every other value is deleted.

\sphinxAtStartPar
Args:
\begin{description}
\item[{fine:}] \leavevmode
\sphinxAtStartPar
The Field object on the finer grid

\item[{coarse:}] \leavevmode
\sphinxAtStartPar
The Field object on the coarser grid

\end{description}

\end{fulllineitems}

\index{sum4way() (in module Contractinator)@\spxentry{sum4way()}\spxextra{in module Contractinator}}

\begin{fulllineitems}
\phantomsection\label{\detokenize{autoapi/Contractinator/index:Contractinator.sum4way}}\pysiglinewithargsret{\sphinxcode{\sphinxupquote{Contractinator.}}\sphinxbfcode{\sphinxupquote{sum4way}}}{\emph{\DUrole{n}{fine}}, \emph{\DUrole{n}{coarse}}}{}
\sphinxAtStartPar
Contracts the Field by summing 4 values into 1. Only setup for 2D Fields (?) so far.

\sphinxAtStartPar
Args:
\begin{description}
\item[{fine:}] \leavevmode
\sphinxAtStartPar
The Field object on the finer grid

\item[{coarse:}] \leavevmode
\sphinxAtStartPar
The Field object on the coarser grid

\end{description}

\end{fulllineitems}

\index{conservative4way() (in module Contractinator)@\spxentry{conservative4way()}\spxextra{in module Contractinator}}

\begin{fulllineitems}
\phantomsection\label{\detokenize{autoapi/Contractinator/index:Contractinator.conservative4way}}\pysiglinewithargsret{\sphinxcode{\sphinxupquote{Contractinator.}}\sphinxbfcode{\sphinxupquote{conservative4way}}}{\emph{\DUrole{n}{fine}}, \emph{\DUrole{n}{coarse}}, \emph{\DUrole{n}{weights}\DUrole{o}{=}\DUrole{default_value}{None}}}{}
\sphinxAtStartPar
Contracts the Field by averaging 4 values into 1 with weighting terms. Only set up for 2D Fields (?) so far.

\sphinxAtStartPar
Args:
\begin{description}
\item[{fine:}] \leavevmode
\sphinxAtStartPar
The Field object on the finer grid

\item[{coarse:}] \leavevmode
\sphinxAtStartPar
The Field object on the coarser grid

\end{description}

\end{fulllineitems}



\section{\sphinxstyleliteralintitle{\sphinxupquote{Cycle}}}
\label{\detokenize{autoapi/Cycle/index:module-Cycle}}\label{\detokenize{autoapi/Cycle/index:cycle}}\label{\detokenize{autoapi/Cycle/index::doc}}\index{module@\spxentry{module}!Cycle@\spxentry{Cycle}}\index{Cycle@\spxentry{Cycle}!module@\spxentry{module}}

\subsection{Module Contents}
\label{\detokenize{autoapi/Cycle/index:module-contents}}

\subsubsection{Classes}
\label{\detokenize{autoapi/Cycle/index:classes}}

\begin{savenotes}\sphinxatlongtablestart\begin{longtable}[c]{\X{1}{2}\X{1}{2}}
\hline

\endfirsthead

\multicolumn{2}{c}%
{\makebox[0pt]{\sphinxtablecontinued{\tablename\ \thetable{} \textendash{} continued from previous page}}}\\
\hline

\endhead

\hline
\multicolumn{2}{r}{\makebox[0pt][r]{\sphinxtablecontinued{continues on next page}}}\\
\endfoot

\endlastfoot

\sphinxAtStartPar
{\hyperref[\detokenize{autoapi/Cycle/index:Cycle.Cycle}]{\sphinxcrossref{\sphinxcode{\sphinxupquote{Cycle}}}}}
&
\sphinxAtStartPar
Contains information about the shape and depth of multigrid cycle
\\
\hline
\end{longtable}\sphinxatlongtableend\end{savenotes}
\index{Cycle (class in Cycle)@\spxentry{Cycle}\spxextra{class in Cycle}}

\begin{fulllineitems}
\phantomsection\label{\detokenize{autoapi/Cycle/index:Cycle.Cycle}}\pysiglinewithargsret{\sphinxbfcode{\sphinxupquote{class }}\sphinxcode{\sphinxupquote{Cycle.}}\sphinxbfcode{\sphinxupquote{Cycle}}}{\emph{\DUrole{n}{input}}}{}
\sphinxAtStartPar
Bases: \sphinxcode{\sphinxupquote{abc.ABC}}

\sphinxAtStartPar
Contains information about the shape and depth of multigrid cycle
\begin{description}
\item[{Constructor:}] \leavevmode\begin{description}
\item[{Args:}] \leavevmode
\sphinxAtStartPar
input (dictionary) : placeholder for potential input

\item[{Returns:}] \leavevmode
\sphinxAtStartPar
A new Cycle object

\item[{Notes:}] \leavevmode
\sphinxAtStartPar
Could be expanded to include default options such as “V” and “W” cycle

\end{description}

\end{description}
\index{pattern (Cycle.Cycle attribute)@\spxentry{pattern}\spxextra{Cycle.Cycle attribute}}

\begin{fulllineitems}
\phantomsection\label{\detokenize{autoapi/Cycle/index:Cycle.Cycle.pattern}}\pysigline{\sphinxbfcode{\sphinxupquote{pattern}}}
\sphinxAtStartPar
array with sequence of directions for a cycle
\begin{quote}\begin{description}
\item[{Type}] \leavevmode
\sphinxAtStartPar
np.ndarray

\end{description}\end{quote}

\end{fulllineitems}

\index{levels (Cycle.Cycle attribute)@\spxentry{levels}\spxextra{Cycle.Cycle attribute}}

\begin{fulllineitems}
\phantomsection\label{\detokenize{autoapi/Cycle/index:Cycle.Cycle.levels}}\pysigline{\sphinxbfcode{\sphinxupquote{levels}}}
\sphinxAtStartPar
depth of cycle
\begin{quote}\begin{description}
\item[{Type}] \leavevmode
\sphinxAtStartPar
int

\end{description}\end{quote}

\end{fulllineitems}

\index{path() (Cycle.Cycle method)@\spxentry{path()}\spxextra{Cycle.Cycle method}}

\begin{fulllineitems}
\phantomsection\label{\detokenize{autoapi/Cycle/index:Cycle.Cycle.path}}\pysiglinewithargsret{\sphinxbfcode{\sphinxupquote{path}}}{\emph{\DUrole{n}{self}}}{}
\end{fulllineitems}

\index{depth() (Cycle.Cycle method)@\spxentry{depth()}\spxextra{Cycle.Cycle method}}

\begin{fulllineitems}
\phantomsection\label{\detokenize{autoapi/Cycle/index:Cycle.Cycle.depth}}\pysiglinewithargsret{\sphinxbfcode{\sphinxupquote{depth}}}{\emph{\DUrole{n}{self}}}{}
\end{fulllineitems}


\end{fulllineitems}



\section{\sphinxstyleliteralintitle{\sphinxupquote{Expandinator}}}
\label{\detokenize{autoapi/Expandinator/index:module-Expandinator}}\label{\detokenize{autoapi/Expandinator/index:expandinator}}\label{\detokenize{autoapi/Expandinator/index::doc}}\index{module@\spxentry{module}!Expandinator@\spxentry{Expandinator}}\index{Expandinator@\spxentry{Expandinator}!module@\spxentry{module}}
\sphinxAtStartPar
Description

\sphinxAtStartPar
Expands Field objects from coarser meshes to finer meshes.

\sphinxAtStartPar
Libraries/Modules

\sphinxAtStartPar
bin.Field

\sphinxAtStartPar
numpy


\subsection{Module Contents}
\label{\detokenize{autoapi/Expandinator/index:module-contents}}

\subsubsection{Functions}
\label{\detokenize{autoapi/Expandinator/index:functions}}

\begin{savenotes}\sphinxatlongtablestart\begin{longtable}[c]{\X{1}{2}\X{1}{2}}
\hline

\endfirsthead

\multicolumn{2}{c}%
{\makebox[0pt]{\sphinxtablecontinued{\tablename\ \thetable{} \textendash{} continued from previous page}}}\\
\hline

\endhead

\hline
\multicolumn{2}{r}{\makebox[0pt][r]{\sphinxtablecontinued{continues on next page}}}\\
\endfoot

\endlastfoot

\sphinxAtStartPar
{\hyperref[\detokenize{autoapi/Expandinator/index:Expandinator.bilinear4way}]{\sphinxcrossref{\sphinxcode{\sphinxupquote{bilinear4way}}}}}(coarse, fine)
&
\sphinxAtStartPar

\\
\hline
\end{longtable}\sphinxatlongtableend\end{savenotes}
\index{bilinear4way() (in module Expandinator)@\spxentry{bilinear4way()}\spxextra{in module Expandinator}}

\begin{fulllineitems}
\phantomsection\label{\detokenize{autoapi/Expandinator/index:Expandinator.bilinear4way}}\pysiglinewithargsret{\sphinxcode{\sphinxupquote{Expandinator.}}\sphinxbfcode{\sphinxupquote{bilinear4way}}}{\emph{\DUrole{n}{coarse}}, \emph{\DUrole{n}{fine}}}{}
\end{fulllineitems}



\section{\sphinxstyleliteralintitle{\sphinxupquote{Field}}}
\label{\detokenize{autoapi/Field/index:module-Field}}\label{\detokenize{autoapi/Field/index:field}}\label{\detokenize{autoapi/Field/index::doc}}\index{module@\spxentry{module}!Field@\spxentry{Field}}\index{Field@\spxentry{Field}!module@\spxentry{module}}

\subsection{Module Contents}
\label{\detokenize{autoapi/Field/index:module-contents}}

\subsubsection{Classes}
\label{\detokenize{autoapi/Field/index:classes}}

\begin{savenotes}\sphinxatlongtablestart\begin{longtable}[c]{\X{1}{2}\X{1}{2}}
\hline

\endfirsthead

\multicolumn{2}{c}%
{\makebox[0pt]{\sphinxtablecontinued{\tablename\ \thetable{} \textendash{} continued from previous page}}}\\
\hline

\endhead

\hline
\multicolumn{2}{r}{\makebox[0pt][r]{\sphinxtablecontinued{continues on next page}}}\\
\endfoot

\endlastfoot

\sphinxAtStartPar
{\hyperref[\detokenize{autoapi/Field/index:Field.Field}]{\sphinxcrossref{\sphinxcode{\sphinxupquote{Field}}}}}
&
\sphinxAtStartPar
Holds numeric data on a Grid. Meant to be used in a similar fashion to a numpy array.
\\
\hline
\end{longtable}\sphinxatlongtableend\end{savenotes}


\subsubsection{Functions}
\label{\detokenize{autoapi/Field/index:functions}}

\begin{savenotes}\sphinxatlongtablestart\begin{longtable}[c]{\X{1}{2}\X{1}{2}}
\hline

\endfirsthead

\multicolumn{2}{c}%
{\makebox[0pt]{\sphinxtablecontinued{\tablename\ \thetable{} \textendash{} continued from previous page}}}\\
\hline

\endhead

\hline
\multicolumn{2}{r}{\makebox[0pt][r]{\sphinxtablecontinued{continues on next page}}}\\
\endfoot

\endlastfoot

\sphinxAtStartPar
{\hyperref[\detokenize{autoapi/Field/index:Field.is_numpy}]{\sphinxcrossref{\sphinxcode{\sphinxupquote{is\_numpy}}}}}(var)
&
\sphinxAtStartPar

\\
\hline
\sphinxAtStartPar
{\hyperref[\detokenize{autoapi/Field/index:Field.is_field}]{\sphinxcrossref{\sphinxcode{\sphinxupquote{is\_field}}}}}(var)
&
\sphinxAtStartPar

\\
\hline
\sphinxAtStartPar
{\hyperref[\detokenize{autoapi/Field/index:Field.array_equal}]{\sphinxcrossref{\sphinxcode{\sphinxupquote{array\_equal}}}}}(array1, array2)
&
\sphinxAtStartPar

\\
\hline
\sphinxAtStartPar
{\hyperref[\detokenize{autoapi/Field/index:Field.copy}]{\sphinxcrossref{\sphinxcode{\sphinxupquote{copy}}}}}(array)
&
\sphinxAtStartPar

\\
\hline
\sphinxAtStartPar
{\hyperref[\detokenize{autoapi/Field/index:Field.mean}]{\sphinxcrossref{\sphinxcode{\sphinxupquote{mean}}}}}(array, axis=None)
&
\sphinxAtStartPar

\\
\hline
\sphinxAtStartPar
{\hyperref[\detokenize{autoapi/Field/index:Field.abs}]{\sphinxcrossref{\sphinxcode{\sphinxupquote{abs}}}}}(array)
&
\sphinxAtStartPar

\\
\hline
\sphinxAtStartPar
{\hyperref[\detokenize{autoapi/Field/index:Field.max}]{\sphinxcrossref{\sphinxcode{\sphinxupquote{max}}}}}(array, axis=None)
&
\sphinxAtStartPar

\\
\hline
\sphinxAtStartPar
{\hyperref[\detokenize{autoapi/Field/index:Field.min}]{\sphinxcrossref{\sphinxcode{\sphinxupquote{min}}}}}(array, axis=None)
&
\sphinxAtStartPar

\\
\hline
\sphinxAtStartPar
{\hyperref[\detokenize{autoapi/Field/index:Field.sum}]{\sphinxcrossref{\sphinxcode{\sphinxupquote{sum}}}}}(array)
&
\sphinxAtStartPar

\\
\hline
\sphinxAtStartPar
{\hyperref[\detokenize{autoapi/Field/index:Field.sqrt}]{\sphinxcrossref{\sphinxcode{\sphinxupquote{sqrt}}}}}(array)
&
\sphinxAtStartPar

\\
\hline
\sphinxAtStartPar
{\hyperref[\detokenize{autoapi/Field/index:Field.square}]{\sphinxcrossref{\sphinxcode{\sphinxupquote{square}}}}}(array)
&
\sphinxAtStartPar

\\
\hline
\sphinxAtStartPar
{\hyperref[\detokenize{autoapi/Field/index:Field.pow}]{\sphinxcrossref{\sphinxcode{\sphinxupquote{pow}}}}}(array, power)
&
\sphinxAtStartPar

\\
\hline
\sphinxAtStartPar
{\hyperref[\detokenize{autoapi/Field/index:Field.norm}]{\sphinxcrossref{\sphinxcode{\sphinxupquote{norm}}}}}(array1, array2)
&
\sphinxAtStartPar

\\
\hline
\sphinxAtStartPar
{\hyperref[\detokenize{autoapi/Field/index:Field.pos_diff}]{\sphinxcrossref{\sphinxcode{\sphinxupquote{pos\_diff}}}}}(array1, array2)
&
\sphinxAtStartPar

\\
\hline
\sphinxAtStartPar
{\hyperref[\detokenize{autoapi/Field/index:Field.isfinite}]{\sphinxcrossref{\sphinxcode{\sphinxupquote{isfinite}}}}}(array)
&
\sphinxAtStartPar

\\
\hline
\sphinxAtStartPar
{\hyperref[\detokenize{autoapi/Field/index:Field.isscalar}]{\sphinxcrossref{\sphinxcode{\sphinxupquote{isscalar}}}}}(array)
&
\sphinxAtStartPar

\\
\hline
\sphinxAtStartPar
{\hyperref[\detokenize{autoapi/Field/index:Field.minimum}]{\sphinxcrossref{\sphinxcode{\sphinxupquote{minimum}}}}}(array1, array2)
&
\sphinxAtStartPar

\\
\hline
\sphinxAtStartPar
{\hyperref[\detokenize{autoapi/Field/index:Field.maximum}]{\sphinxcrossref{\sphinxcode{\sphinxupquote{maximum}}}}}(array1, array2)
&
\sphinxAtStartPar

\\
\hline
\sphinxAtStartPar
{\hyperref[\detokenize{autoapi/Field/index:Field.mismatch_mul}]{\sphinxcrossref{\sphinxcode{\sphinxupquote{mismatch\_mul}}}}}(self, other)
&
\sphinxAtStartPar

\\
\hline
\sphinxAtStartPar
{\hyperref[\detokenize{autoapi/Field/index:Field.mismatch_truediv}]{\sphinxcrossref{\sphinxcode{\sphinxupquote{mismatch\_truediv}}}}}(self, other)
&
\sphinxAtStartPar

\\
\hline
\end{longtable}\sphinxatlongtableend\end{savenotes}


\subsubsection{Attributes}
\label{\detokenize{autoapi/Field/index:attributes}}

\begin{savenotes}\sphinxatlongtablestart\begin{longtable}[c]{\X{1}{2}\X{1}{2}}
\hline

\endfirsthead

\multicolumn{2}{c}%
{\makebox[0pt]{\sphinxtablecontinued{\tablename\ \thetable{} \textendash{} continued from previous page}}}\\
\hline

\endhead

\hline
\multicolumn{2}{r}{\makebox[0pt][r]{\sphinxtablecontinued{continues on next page}}}\\
\endfoot

\endlastfoot

\sphinxAtStartPar
{\hyperref[\detokenize{autoapi/Field/index:Field.Infinity}]{\sphinxcrossref{\sphinxcode{\sphinxupquote{Infinity}}}}}
&
\sphinxAtStartPar

\\
\hline
\end{longtable}\sphinxatlongtableend\end{savenotes}
\index{Field (class in Field)@\spxentry{Field}\spxextra{class in Field}}

\begin{fulllineitems}
\phantomsection\label{\detokenize{autoapi/Field/index:Field.Field}}\pysiglinewithargsret{\sphinxbfcode{\sphinxupquote{class }}\sphinxcode{\sphinxupquote{Field.}}\sphinxbfcode{\sphinxupquote{Field}}}{\emph{\DUrole{n}{shape}}, \emph{\DUrole{n}{vals}\DUrole{o}{=}\DUrole{default_value}{None}}}{}~\begin{description}
\item[{Holds numeric data on a Grid. Meant to be used in a similar fashion to a numpy array.}] \leavevmode
\sphinxAtStartPar
Can be indexed and operators are overloaded for basic math operations.

\item[{Constructor:}] \leavevmode\begin{description}
\item[{Args:}] \leavevmode
\sphinxAtStartPar
shape (tuple): n dimensional array of Field dimensions

\item[{Returns:}] \leavevmode
\sphinxAtStartPar
A new Field object

\item[{Notes:}] \leavevmode
\sphinxAtStartPar
Check top of Input.py file to see the contents of each of the five dictionanries

\end{description}

\end{description}
\index{vals (Field.Field attribute)@\spxentry{vals}\spxextra{Field.Field attribute}}

\begin{fulllineitems}
\phantomsection\label{\detokenize{autoapi/Field/index:Field.Field.vals}}\pysigline{\sphinxbfcode{\sphinxupquote{vals}}}
\sphinxAtStartPar
numeric values of the Field
\begin{quote}\begin{description}
\item[{Type}] \leavevmode
\sphinxAtStartPar
np.ndarray

\end{description}\end{quote}

\end{fulllineitems}

\index{size() (Field.Field method)@\spxentry{size()}\spxextra{Field.Field method}}

\begin{fulllineitems}
\phantomsection\label{\detokenize{autoapi/Field/index:Field.Field.size}}\pysiglinewithargsret{\sphinxbfcode{\sphinxupquote{size}}}{\emph{\DUrole{n}{self}}}{}
\sphinxAtStartPar
2\sphinxhyphen{}d size of field

\sphinxAtStartPar
Returns
\begin{description}
\item[{:}] \leavevmode
\sphinxAtStartPar
The 2\sphinxhyphen{}d size of the field.
This is important for fields living on a 2\sphinxhyphen{}d grid

\end{description}

\end{fulllineitems}

\index{shape() (Field.Field method)@\spxentry{shape()}\spxextra{Field.Field method}}

\begin{fulllineitems}
\phantomsection\label{\detokenize{autoapi/Field/index:Field.Field.shape}}\pysiglinewithargsret{\sphinxbfcode{\sphinxupquote{shape}}}{\emph{\DUrole{n}{self}}}{}
\sphinxAtStartPar
shape of field

\sphinxAtStartPar
Returns
\begin{description}
\item[{:}] \leavevmode
\sphinxAtStartPar
The shape of the underlying numpy array.

\end{description}

\end{fulllineitems}

\index{dim() (Field.Field method)@\spxentry{dim()}\spxextra{Field.Field method}}

\begin{fulllineitems}
\phantomsection\label{\detokenize{autoapi/Field/index:Field.Field.dim}}\pysiglinewithargsret{\sphinxbfcode{\sphinxupquote{dim}}}{\emph{\DUrole{n}{self}}}{}
\sphinxAtStartPar
dimensions of variable

\sphinxAtStartPar
Returns
\begin{description}
\item[{:}] \leavevmode
\sphinxAtStartPar
The dimensions of the field vector living at each point in a 2\sphinxhyphen{}d grid
This value is 1 for 1\sphinxhyphen{}d and 2\sphinxhyphen{}d arrays

\end{description}

\end{fulllineitems}

\index{get\_vals() (Field.Field method)@\spxentry{get\_vals()}\spxextra{Field.Field method}}

\begin{fulllineitems}
\phantomsection\label{\detokenize{autoapi/Field/index:Field.Field.get_vals}}\pysiglinewithargsret{\sphinxbfcode{\sphinxupquote{get\_vals}}}{\emph{\DUrole{n}{self}}}{}
\sphinxAtStartPar
get the underlying numpy representation

\sphinxAtStartPar
Returns
\begin{description}
\item[{:}] \leavevmode
\sphinxAtStartPar
The underlying numpy ndarray that stores the values

\end{description}

\end{fulllineitems}

\index{astype() (Field.Field method)@\spxentry{astype()}\spxextra{Field.Field method}}

\begin{fulllineitems}
\phantomsection\label{\detokenize{autoapi/Field/index:Field.Field.astype}}\pysiglinewithargsret{\sphinxbfcode{\sphinxupquote{astype}}}{\emph{\DUrole{n}{self}}, \emph{\DUrole{n}{dtype}\DUrole{o}{=}\DUrole{default_value}{None}}}{}
\sphinxAtStartPar
return a field with data stored as the given type

\sphinxAtStartPar
Returns
\begin{description}
\item[{:}] \leavevmode
\sphinxAtStartPar
A field with values stored as the given type

\end{description}

\end{fulllineitems}

\index{T() (Field.Field method)@\spxentry{T()}\spxextra{Field.Field method}}

\begin{fulllineitems}
\phantomsection\label{\detokenize{autoapi/Field/index:Field.Field.T}}\pysiglinewithargsret{\sphinxbfcode{\sphinxupquote{T}}}{\emph{\DUrole{n}{self}}}{}
\sphinxAtStartPar
return a transposed Field

\sphinxAtStartPar
Returns
\begin{description}
\item[{:}] \leavevmode
\sphinxAtStartPar
A field of the size of the transposed input

\end{description}

\end{fulllineitems}

\index{\_\_getitem\_\_() (Field.Field method)@\spxentry{\_\_getitem\_\_()}\spxextra{Field.Field method}}

\begin{fulllineitems}
\phantomsection\label{\detokenize{autoapi/Field/index:Field.Field.__getitem__}}\pysiglinewithargsret{\sphinxbfcode{\sphinxupquote{\_\_getitem\_\_}}}{\emph{\DUrole{n}{self}}, \emph{\DUrole{n}{indx}}}{}
\end{fulllineitems}

\index{\_\_setitem\_\_() (Field.Field method)@\spxentry{\_\_setitem\_\_()}\spxextra{Field.Field method}}

\begin{fulllineitems}
\phantomsection\label{\detokenize{autoapi/Field/index:Field.Field.__setitem__}}\pysiglinewithargsret{\sphinxbfcode{\sphinxupquote{\_\_setitem\_\_}}}{\emph{\DUrole{n}{self}}, \emph{\DUrole{n}{indx}}, \emph{\DUrole{n}{value}}}{}
\end{fulllineitems}

\index{set\_val() (Field.Field method)@\spxentry{set\_val()}\spxextra{Field.Field method}}

\begin{fulllineitems}
\phantomsection\label{\detokenize{autoapi/Field/index:Field.Field.set_val}}\pysiglinewithargsret{\sphinxbfcode{\sphinxupquote{set\_val}}}{\emph{\DUrole{n}{self}}, \emph{\DUrole{n}{new\_vals}}}{}
\sphinxAtStartPar
assign values to a Field of the same size

\sphinxAtStartPar
Returns
\begin{description}
\item[{:}] \leavevmode
\sphinxAtStartPar
A field with given values

\end{description}

\end{fulllineitems}

\index{\_\_str\_\_() (Field.Field method)@\spxentry{\_\_str\_\_()}\spxextra{Field.Field method}}

\begin{fulllineitems}
\phantomsection\label{\detokenize{autoapi/Field/index:Field.Field.__str__}}\pysiglinewithargsret{\sphinxbfcode{\sphinxupquote{\_\_str\_\_}}}{\emph{\DUrole{n}{self}}}{}
\sphinxAtStartPar
Return str(self).

\end{fulllineitems}

\index{\_\_len\_\_() (Field.Field method)@\spxentry{\_\_len\_\_()}\spxextra{Field.Field method}}

\begin{fulllineitems}
\phantomsection\label{\detokenize{autoapi/Field/index:Field.Field.__len__}}\pysiglinewithargsret{\sphinxbfcode{\sphinxupquote{\_\_len\_\_}}}{\emph{\DUrole{n}{self}}}{}
\end{fulllineitems}

\index{\_\_lt\_\_() (Field.Field method)@\spxentry{\_\_lt\_\_()}\spxextra{Field.Field method}}

\begin{fulllineitems}
\phantomsection\label{\detokenize{autoapi/Field/index:Field.Field.__lt__}}\pysiglinewithargsret{\sphinxbfcode{\sphinxupquote{\_\_lt\_\_}}}{\emph{\DUrole{n}{self}}, \emph{\DUrole{n}{other}}}{}
\sphinxAtStartPar
Return self\textless{}value.

\end{fulllineitems}

\index{\_\_le\_\_() (Field.Field method)@\spxentry{\_\_le\_\_()}\spxextra{Field.Field method}}

\begin{fulllineitems}
\phantomsection\label{\detokenize{autoapi/Field/index:Field.Field.__le__}}\pysiglinewithargsret{\sphinxbfcode{\sphinxupquote{\_\_le\_\_}}}{\emph{\DUrole{n}{self}}, \emph{\DUrole{n}{other}}}{}
\sphinxAtStartPar
Return self\textless{}=value.

\end{fulllineitems}

\index{\_\_gt\_\_() (Field.Field method)@\spxentry{\_\_gt\_\_()}\spxextra{Field.Field method}}

\begin{fulllineitems}
\phantomsection\label{\detokenize{autoapi/Field/index:Field.Field.__gt__}}\pysiglinewithargsret{\sphinxbfcode{\sphinxupquote{\_\_gt\_\_}}}{\emph{\DUrole{n}{self}}, \emph{\DUrole{n}{other}}}{}
\sphinxAtStartPar
Return self\textgreater{}value.

\end{fulllineitems}

\index{\_\_ge\_\_() (Field.Field method)@\spxentry{\_\_ge\_\_()}\spxextra{Field.Field method}}

\begin{fulllineitems}
\phantomsection\label{\detokenize{autoapi/Field/index:Field.Field.__ge__}}\pysiglinewithargsret{\sphinxbfcode{\sphinxupquote{\_\_ge\_\_}}}{\emph{\DUrole{n}{self}}, \emph{\DUrole{n}{other}}}{}
\sphinxAtStartPar
Return self\textgreater{}=value.

\end{fulllineitems}

\index{\_\_eq\_\_() (Field.Field method)@\spxentry{\_\_eq\_\_()}\spxextra{Field.Field method}}

\begin{fulllineitems}
\phantomsection\label{\detokenize{autoapi/Field/index:Field.Field.__eq__}}\pysiglinewithargsret{\sphinxbfcode{\sphinxupquote{\_\_eq\_\_}}}{\emph{\DUrole{n}{self}}, \emph{\DUrole{n}{other}}}{}
\sphinxAtStartPar
Return self==value.

\end{fulllineitems}

\index{\_\_ne\_\_() (Field.Field method)@\spxentry{\_\_ne\_\_()}\spxextra{Field.Field method}}

\begin{fulllineitems}
\phantomsection\label{\detokenize{autoapi/Field/index:Field.Field.__ne__}}\pysiglinewithargsret{\sphinxbfcode{\sphinxupquote{\_\_ne\_\_}}}{\emph{\DUrole{n}{self}}, \emph{\DUrole{n}{other}}}{}
\sphinxAtStartPar
Return self!=value.

\end{fulllineitems}

\index{\_\_bool\_\_() (Field.Field method)@\spxentry{\_\_bool\_\_()}\spxextra{Field.Field method}}

\begin{fulllineitems}
\phantomsection\label{\detokenize{autoapi/Field/index:Field.Field.__bool__}}\pysiglinewithargsret{\sphinxbfcode{\sphinxupquote{\_\_bool\_\_}}}{\emph{\DUrole{n}{self}}}{}
\end{fulllineitems}

\index{\_\_add\_\_() (Field.Field method)@\spxentry{\_\_add\_\_()}\spxextra{Field.Field method}}

\begin{fulllineitems}
\phantomsection\label{\detokenize{autoapi/Field/index:Field.Field.__add__}}\pysiglinewithargsret{\sphinxbfcode{\sphinxupquote{\_\_add\_\_}}}{\emph{\DUrole{n}{self}}, \emph{\DUrole{n}{other}}}{}
\end{fulllineitems}

\index{\_\_sub\_\_() (Field.Field method)@\spxentry{\_\_sub\_\_()}\spxextra{Field.Field method}}

\begin{fulllineitems}
\phantomsection\label{\detokenize{autoapi/Field/index:Field.Field.__sub__}}\pysiglinewithargsret{\sphinxbfcode{\sphinxupquote{\_\_sub\_\_}}}{\emph{\DUrole{n}{self}}, \emph{\DUrole{n}{other}}}{}
\end{fulllineitems}

\index{\_\_mul\_\_() (Field.Field method)@\spxentry{\_\_mul\_\_()}\spxextra{Field.Field method}}

\begin{fulllineitems}
\phantomsection\label{\detokenize{autoapi/Field/index:Field.Field.__mul__}}\pysiglinewithargsret{\sphinxbfcode{\sphinxupquote{\_\_mul\_\_}}}{\emph{\DUrole{n}{self}}, \emph{\DUrole{n}{other}}}{}
\end{fulllineitems}

\index{\_\_pow\_\_() (Field.Field method)@\spxentry{\_\_pow\_\_()}\spxextra{Field.Field method}}

\begin{fulllineitems}
\phantomsection\label{\detokenize{autoapi/Field/index:Field.Field.__pow__}}\pysiglinewithargsret{\sphinxbfcode{\sphinxupquote{\_\_pow\_\_}}}{\emph{\DUrole{n}{self}}, \emph{\DUrole{n}{other}}}{}
\end{fulllineitems}

\index{\_\_truediv\_\_() (Field.Field method)@\spxentry{\_\_truediv\_\_()}\spxextra{Field.Field method}}

\begin{fulllineitems}
\phantomsection\label{\detokenize{autoapi/Field/index:Field.Field.__truediv__}}\pysiglinewithargsret{\sphinxbfcode{\sphinxupquote{\_\_truediv\_\_}}}{\emph{\DUrole{n}{self}}, \emph{\DUrole{n}{other}}}{}
\end{fulllineitems}

\index{\_\_floordiv\_\_() (Field.Field method)@\spxentry{\_\_floordiv\_\_()}\spxextra{Field.Field method}}

\begin{fulllineitems}
\phantomsection\label{\detokenize{autoapi/Field/index:Field.Field.__floordiv__}}\pysiglinewithargsret{\sphinxbfcode{\sphinxupquote{\_\_floordiv\_\_}}}{\emph{\DUrole{n}{self}}, \emph{\DUrole{n}{other}}}{}
\end{fulllineitems}

\index{\_\_mod\_\_() (Field.Field method)@\spxentry{\_\_mod\_\_()}\spxextra{Field.Field method}}

\begin{fulllineitems}
\phantomsection\label{\detokenize{autoapi/Field/index:Field.Field.__mod__}}\pysiglinewithargsret{\sphinxbfcode{\sphinxupquote{\_\_mod\_\_}}}{\emph{\DUrole{n}{self}}, \emph{\DUrole{n}{other}}}{}
\end{fulllineitems}

\index{\_\_iadd\_\_() (Field.Field method)@\spxentry{\_\_iadd\_\_()}\spxextra{Field.Field method}}

\begin{fulllineitems}
\phantomsection\label{\detokenize{autoapi/Field/index:Field.Field.__iadd__}}\pysiglinewithargsret{\sphinxbfcode{\sphinxupquote{\_\_iadd\_\_}}}{\emph{\DUrole{n}{self}}, \emph{\DUrole{n}{other}}}{}
\end{fulllineitems}

\index{\_\_isub\_\_() (Field.Field method)@\spxentry{\_\_isub\_\_()}\spxextra{Field.Field method}}

\begin{fulllineitems}
\phantomsection\label{\detokenize{autoapi/Field/index:Field.Field.__isub__}}\pysiglinewithargsret{\sphinxbfcode{\sphinxupquote{\_\_isub\_\_}}}{\emph{\DUrole{n}{self}}, \emph{\DUrole{n}{other}}}{}
\end{fulllineitems}

\index{\_\_imul\_\_() (Field.Field method)@\spxentry{\_\_imul\_\_()}\spxextra{Field.Field method}}

\begin{fulllineitems}
\phantomsection\label{\detokenize{autoapi/Field/index:Field.Field.__imul__}}\pysiglinewithargsret{\sphinxbfcode{\sphinxupquote{\_\_imul\_\_}}}{\emph{\DUrole{n}{self}}, \emph{\DUrole{n}{other}}}{}
\end{fulllineitems}

\index{\_\_ipow\_\_() (Field.Field method)@\spxentry{\_\_ipow\_\_()}\spxextra{Field.Field method}}

\begin{fulllineitems}
\phantomsection\label{\detokenize{autoapi/Field/index:Field.Field.__ipow__}}\pysiglinewithargsret{\sphinxbfcode{\sphinxupquote{\_\_ipow\_\_}}}{\emph{\DUrole{n}{self}}, \emph{\DUrole{n}{other}}}{}
\end{fulllineitems}

\index{\_\_itruediv\_\_() (Field.Field method)@\spxentry{\_\_itruediv\_\_()}\spxextra{Field.Field method}}

\begin{fulllineitems}
\phantomsection\label{\detokenize{autoapi/Field/index:Field.Field.__itruediv__}}\pysiglinewithargsret{\sphinxbfcode{\sphinxupquote{\_\_itruediv\_\_}}}{\emph{\DUrole{n}{self}}, \emph{\DUrole{n}{other}}}{}
\end{fulllineitems}

\index{\_\_radd\_\_() (Field.Field method)@\spxentry{\_\_radd\_\_()}\spxextra{Field.Field method}}

\begin{fulllineitems}
\phantomsection\label{\detokenize{autoapi/Field/index:Field.Field.__radd__}}\pysiglinewithargsret{\sphinxbfcode{\sphinxupquote{\_\_radd\_\_}}}{\emph{\DUrole{n}{self}}, \emph{\DUrole{n}{other}}}{}
\end{fulllineitems}

\index{\_\_rsub\_\_() (Field.Field method)@\spxentry{\_\_rsub\_\_()}\spxextra{Field.Field method}}

\begin{fulllineitems}
\phantomsection\label{\detokenize{autoapi/Field/index:Field.Field.__rsub__}}\pysiglinewithargsret{\sphinxbfcode{\sphinxupquote{\_\_rsub\_\_}}}{\emph{\DUrole{n}{self}}, \emph{\DUrole{n}{other}}}{}
\end{fulllineitems}

\index{\_\_rmul\_\_() (Field.Field method)@\spxentry{\_\_rmul\_\_()}\spxextra{Field.Field method}}

\begin{fulllineitems}
\phantomsection\label{\detokenize{autoapi/Field/index:Field.Field.__rmul__}}\pysiglinewithargsret{\sphinxbfcode{\sphinxupquote{\_\_rmul\_\_}}}{\emph{\DUrole{n}{self}}, \emph{\DUrole{n}{other}}}{}
\end{fulllineitems}

\index{\_\_rpow\_\_() (Field.Field method)@\spxentry{\_\_rpow\_\_()}\spxextra{Field.Field method}}

\begin{fulllineitems}
\phantomsection\label{\detokenize{autoapi/Field/index:Field.Field.__rpow__}}\pysiglinewithargsret{\sphinxbfcode{\sphinxupquote{\_\_rpow\_\_}}}{\emph{\DUrole{n}{self}}, \emph{\DUrole{n}{other}}}{}
\end{fulllineitems}

\index{\_\_rtruediv\_\_() (Field.Field method)@\spxentry{\_\_rtruediv\_\_()}\spxextra{Field.Field method}}

\begin{fulllineitems}
\phantomsection\label{\detokenize{autoapi/Field/index:Field.Field.__rtruediv__}}\pysiglinewithargsret{\sphinxbfcode{\sphinxupquote{\_\_rtruediv\_\_}}}{\emph{\DUrole{n}{self}}, \emph{\DUrole{n}{other}}}{}
\end{fulllineitems}

\index{\_\_rfloordiv\_\_() (Field.Field method)@\spxentry{\_\_rfloordiv\_\_()}\spxextra{Field.Field method}}

\begin{fulllineitems}
\phantomsection\label{\detokenize{autoapi/Field/index:Field.Field.__rfloordiv__}}\pysiglinewithargsret{\sphinxbfcode{\sphinxupquote{\_\_rfloordiv\_\_}}}{\emph{\DUrole{n}{self}}, \emph{\DUrole{n}{other}}}{}
\end{fulllineitems}

\index{\_\_neg\_\_() (Field.Field method)@\spxentry{\_\_neg\_\_()}\spxextra{Field.Field method}}

\begin{fulllineitems}
\phantomsection\label{\detokenize{autoapi/Field/index:Field.Field.__neg__}}\pysiglinewithargsret{\sphinxbfcode{\sphinxupquote{\_\_neg\_\_}}}{\emph{\DUrole{n}{self}}}{}
\end{fulllineitems}

\index{\_\_pos\_\_() (Field.Field method)@\spxentry{\_\_pos\_\_()}\spxextra{Field.Field method}}

\begin{fulllineitems}
\phantomsection\label{\detokenize{autoapi/Field/index:Field.Field.__pos__}}\pysiglinewithargsret{\sphinxbfcode{\sphinxupquote{\_\_pos\_\_}}}{\emph{\DUrole{n}{self}}}{}
\end{fulllineitems}

\index{\_\_mismatch\_imul() (Field.Field method)@\spxentry{\_\_mismatch\_imul()}\spxextra{Field.Field method}}

\begin{fulllineitems}
\phantomsection\label{\detokenize{autoapi/Field/index:Field.Field.__mismatch_imul}}\pysiglinewithargsret{\sphinxbfcode{\sphinxupquote{\_\_mismatch\_imul}}}{\emph{\DUrole{n}{self}}, \emph{\DUrole{n}{other}}}{}
\end{fulllineitems}

\index{\_\_mismatch\_itruediv() (Field.Field method)@\spxentry{\_\_mismatch\_itruediv()}\spxextra{Field.Field method}}

\begin{fulllineitems}
\phantomsection\label{\detokenize{autoapi/Field/index:Field.Field.__mismatch_itruediv}}\pysiglinewithargsret{\sphinxbfcode{\sphinxupquote{\_\_mismatch\_itruediv}}}{\emph{\DUrole{n}{self}}, \emph{\DUrole{n}{other}}}{}
\end{fulllineitems}


\end{fulllineitems}

\index{is\_numpy() (in module Field)@\spxentry{is\_numpy()}\spxextra{in module Field}}

\begin{fulllineitems}
\phantomsection\label{\detokenize{autoapi/Field/index:Field.is_numpy}}\pysiglinewithargsret{\sphinxcode{\sphinxupquote{Field.}}\sphinxbfcode{\sphinxupquote{is\_numpy}}}{\emph{\DUrole{n}{var}}}{}
\end{fulllineitems}

\index{is\_field() (in module Field)@\spxentry{is\_field()}\spxextra{in module Field}}

\begin{fulllineitems}
\phantomsection\label{\detokenize{autoapi/Field/index:Field.is_field}}\pysiglinewithargsret{\sphinxcode{\sphinxupquote{Field.}}\sphinxbfcode{\sphinxupquote{is\_field}}}{\emph{\DUrole{n}{var}}}{}
\end{fulllineitems}

\index{array\_equal() (in module Field)@\spxentry{array\_equal()}\spxextra{in module Field}}

\begin{fulllineitems}
\phantomsection\label{\detokenize{autoapi/Field/index:Field.array_equal}}\pysiglinewithargsret{\sphinxcode{\sphinxupquote{Field.}}\sphinxbfcode{\sphinxupquote{array\_equal}}}{\emph{\DUrole{n}{array1}}, \emph{\DUrole{n}{array2}}}{}
\end{fulllineitems}

\index{copy() (in module Field)@\spxentry{copy()}\spxextra{in module Field}}

\begin{fulllineitems}
\phantomsection\label{\detokenize{autoapi/Field/index:Field.copy}}\pysiglinewithargsret{\sphinxcode{\sphinxupquote{Field.}}\sphinxbfcode{\sphinxupquote{copy}}}{\emph{\DUrole{n}{array}}}{}
\end{fulllineitems}

\index{mean() (in module Field)@\spxentry{mean()}\spxextra{in module Field}}

\begin{fulllineitems}
\phantomsection\label{\detokenize{autoapi/Field/index:Field.mean}}\pysiglinewithargsret{\sphinxcode{\sphinxupquote{Field.}}\sphinxbfcode{\sphinxupquote{mean}}}{\emph{\DUrole{n}{array}}, \emph{\DUrole{n}{axis}\DUrole{o}{=}\DUrole{default_value}{None}}}{}
\end{fulllineitems}

\index{abs() (in module Field)@\spxentry{abs()}\spxextra{in module Field}}

\begin{fulllineitems}
\phantomsection\label{\detokenize{autoapi/Field/index:Field.abs}}\pysiglinewithargsret{\sphinxcode{\sphinxupquote{Field.}}\sphinxbfcode{\sphinxupquote{abs}}}{\emph{\DUrole{n}{array}}}{}
\end{fulllineitems}

\index{max() (in module Field)@\spxentry{max()}\spxextra{in module Field}}

\begin{fulllineitems}
\phantomsection\label{\detokenize{autoapi/Field/index:Field.max}}\pysiglinewithargsret{\sphinxcode{\sphinxupquote{Field.}}\sphinxbfcode{\sphinxupquote{max}}}{\emph{\DUrole{n}{array}}, \emph{\DUrole{n}{axis}\DUrole{o}{=}\DUrole{default_value}{None}}}{}
\end{fulllineitems}

\index{min() (in module Field)@\spxentry{min()}\spxextra{in module Field}}

\begin{fulllineitems}
\phantomsection\label{\detokenize{autoapi/Field/index:Field.min}}\pysiglinewithargsret{\sphinxcode{\sphinxupquote{Field.}}\sphinxbfcode{\sphinxupquote{min}}}{\emph{\DUrole{n}{array}}, \emph{\DUrole{n}{axis}\DUrole{o}{=}\DUrole{default_value}{None}}}{}
\end{fulllineitems}

\index{sum() (in module Field)@\spxentry{sum()}\spxextra{in module Field}}

\begin{fulllineitems}
\phantomsection\label{\detokenize{autoapi/Field/index:Field.sum}}\pysiglinewithargsret{\sphinxcode{\sphinxupquote{Field.}}\sphinxbfcode{\sphinxupquote{sum}}}{\emph{\DUrole{n}{array}}}{}
\end{fulllineitems}

\index{sqrt() (in module Field)@\spxentry{sqrt()}\spxextra{in module Field}}

\begin{fulllineitems}
\phantomsection\label{\detokenize{autoapi/Field/index:Field.sqrt}}\pysiglinewithargsret{\sphinxcode{\sphinxupquote{Field.}}\sphinxbfcode{\sphinxupquote{sqrt}}}{\emph{\DUrole{n}{array}}}{}
\end{fulllineitems}

\index{square() (in module Field)@\spxentry{square()}\spxextra{in module Field}}

\begin{fulllineitems}
\phantomsection\label{\detokenize{autoapi/Field/index:Field.square}}\pysiglinewithargsret{\sphinxcode{\sphinxupquote{Field.}}\sphinxbfcode{\sphinxupquote{square}}}{\emph{\DUrole{n}{array}}}{}
\end{fulllineitems}

\index{pow() (in module Field)@\spxentry{pow()}\spxextra{in module Field}}

\begin{fulllineitems}
\phantomsection\label{\detokenize{autoapi/Field/index:Field.pow}}\pysiglinewithargsret{\sphinxcode{\sphinxupquote{Field.}}\sphinxbfcode{\sphinxupquote{pow}}}{\emph{\DUrole{n}{array}}, \emph{\DUrole{n}{power}}}{}
\end{fulllineitems}

\index{norm() (in module Field)@\spxentry{norm()}\spxextra{in module Field}}

\begin{fulllineitems}
\phantomsection\label{\detokenize{autoapi/Field/index:Field.norm}}\pysiglinewithargsret{\sphinxcode{\sphinxupquote{Field.}}\sphinxbfcode{\sphinxupquote{norm}}}{\emph{\DUrole{n}{array1}}, \emph{\DUrole{n}{array2}}}{}
\end{fulllineitems}

\index{pos\_diff() (in module Field)@\spxentry{pos\_diff()}\spxextra{in module Field}}

\begin{fulllineitems}
\phantomsection\label{\detokenize{autoapi/Field/index:Field.pos_diff}}\pysiglinewithargsret{\sphinxcode{\sphinxupquote{Field.}}\sphinxbfcode{\sphinxupquote{pos\_diff}}}{\emph{\DUrole{n}{array1}}, \emph{\DUrole{n}{array2}}}{}
\end{fulllineitems}

\index{isfinite() (in module Field)@\spxentry{isfinite()}\spxextra{in module Field}}

\begin{fulllineitems}
\phantomsection\label{\detokenize{autoapi/Field/index:Field.isfinite}}\pysiglinewithargsret{\sphinxcode{\sphinxupquote{Field.}}\sphinxbfcode{\sphinxupquote{isfinite}}}{\emph{\DUrole{n}{array}}}{}
\end{fulllineitems}

\index{isscalar() (in module Field)@\spxentry{isscalar()}\spxextra{in module Field}}

\begin{fulllineitems}
\phantomsection\label{\detokenize{autoapi/Field/index:Field.isscalar}}\pysiglinewithargsret{\sphinxcode{\sphinxupquote{Field.}}\sphinxbfcode{\sphinxupquote{isscalar}}}{\emph{\DUrole{n}{array}}}{}
\end{fulllineitems}

\index{minimum() (in module Field)@\spxentry{minimum()}\spxextra{in module Field}}

\begin{fulllineitems}
\phantomsection\label{\detokenize{autoapi/Field/index:Field.minimum}}\pysiglinewithargsret{\sphinxcode{\sphinxupquote{Field.}}\sphinxbfcode{\sphinxupquote{minimum}}}{\emph{\DUrole{n}{array1}}, \emph{\DUrole{n}{array2}}}{}
\end{fulllineitems}

\index{maximum() (in module Field)@\spxentry{maximum()}\spxextra{in module Field}}

\begin{fulllineitems}
\phantomsection\label{\detokenize{autoapi/Field/index:Field.maximum}}\pysiglinewithargsret{\sphinxcode{\sphinxupquote{Field.}}\sphinxbfcode{\sphinxupquote{maximum}}}{\emph{\DUrole{n}{array1}}, \emph{\DUrole{n}{array2}}}{}
\end{fulllineitems}

\index{mismatch\_mul() (in module Field)@\spxentry{mismatch\_mul()}\spxextra{in module Field}}

\begin{fulllineitems}
\phantomsection\label{\detokenize{autoapi/Field/index:Field.mismatch_mul}}\pysiglinewithargsret{\sphinxcode{\sphinxupquote{Field.}}\sphinxbfcode{\sphinxupquote{mismatch\_mul}}}{\emph{\DUrole{n}{self}}, \emph{\DUrole{n}{other}}}{}
\end{fulllineitems}

\index{mismatch\_truediv() (in module Field)@\spxentry{mismatch\_truediv()}\spxextra{in module Field}}

\begin{fulllineitems}
\phantomsection\label{\detokenize{autoapi/Field/index:Field.mismatch_truediv}}\pysiglinewithargsret{\sphinxcode{\sphinxupquote{Field.}}\sphinxbfcode{\sphinxupquote{mismatch\_truediv}}}{\emph{\DUrole{n}{self}}, \emph{\DUrole{n}{other}}}{}
\end{fulllineitems}

\index{Infinity (in module Field)@\spxentry{Infinity}\spxextra{in module Field}}

\begin{fulllineitems}
\phantomsection\label{\detokenize{autoapi/Field/index:Field.Infinity}}\pysigline{\sphinxcode{\sphinxupquote{Field.}}\sphinxbfcode{\sphinxupquote{Infinity}}}
\end{fulllineitems}



\section{\sphinxstyleliteralintitle{\sphinxupquote{flo103\_ConvergenceChecker}}}
\label{\detokenize{autoapi/flo103_ConvergenceChecker/index:module-flo103_ConvergenceChecker}}\label{\detokenize{autoapi/flo103_ConvergenceChecker/index:flo103-convergencechecker}}\label{\detokenize{autoapi/flo103_ConvergenceChecker/index::doc}}\index{module@\spxentry{module}!flo103\_ConvergenceChecker@\spxentry{flo103\_ConvergenceChecker}}\index{flo103\_ConvergenceChecker@\spxentry{flo103\_ConvergenceChecker}!module@\spxentry{module}}
\sphinxAtStartPar
This module runs a set number of cycles.
Libraries/Modules:
\begin{quote}

\sphinxAtStartPar
Would use: numpy
\end{quote}


\subsection{Module Contents}
\label{\detokenize{autoapi/flo103_ConvergenceChecker/index:module-contents}}

\subsubsection{Classes}
\label{\detokenize{autoapi/flo103_ConvergenceChecker/index:classes}}

\begin{savenotes}\sphinxatlongtablestart\begin{longtable}[c]{\X{1}{2}\X{1}{2}}
\hline

\endfirsthead

\multicolumn{2}{c}%
{\makebox[0pt]{\sphinxtablecontinued{\tablename\ \thetable{} \textendash{} continued from previous page}}}\\
\hline

\endhead

\hline
\multicolumn{2}{r}{\makebox[0pt][r]{\sphinxtablecontinued{continues on next page}}}\\
\endfoot

\endlastfoot

\sphinxAtStartPar
{\hyperref[\detokenize{autoapi/flo103_ConvergenceChecker/index:flo103_ConvergenceChecker.flo103_ConvergenceChecker}]{\sphinxcrossref{\sphinxcode{\sphinxupquote{flo103\_ConvergenceChecker}}}}}
&
\sphinxAtStartPar
Current implementation is to run a pre\sphinxhyphen{}determined
\\
\hline
\end{longtable}\sphinxatlongtableend\end{savenotes}
\index{flo103\_ConvergenceChecker (class in flo103\_ConvergenceChecker)@\spxentry{flo103\_ConvergenceChecker}\spxextra{class in flo103\_ConvergenceChecker}}

\begin{fulllineitems}
\phantomsection\label{\detokenize{autoapi/flo103_ConvergenceChecker/index:flo103_ConvergenceChecker.flo103_ConvergenceChecker}}\pysiglinewithargsret{\sphinxbfcode{\sphinxupquote{class }}\sphinxcode{\sphinxupquote{flo103\_ConvergenceChecker.}}\sphinxbfcode{\sphinxupquote{flo103\_ConvergenceChecker}}}{\emph{\DUrole{n}{input}}}{}~\begin{description}
\item[{Current implementation is to run a pre\sphinxhyphen{}determined}] \leavevmode
\sphinxAtStartPar
number of cycles, and decide that convergence is when that number
of cycles has been reached.
A future improvement would be to actually implement a way to check
the convergence of the solution, based on residuals or errors, and
continue runnning until that convergence creiteria is below a
certain threshold.

\end{description}
\index{num\_cycles (flo103\_ConvergenceChecker.flo103\_ConvergenceChecker.self attribute)@\spxentry{num\_cycles}\spxextra{flo103\_ConvergenceChecker.flo103\_ConvergenceChecker.self attribute}}

\begin{fulllineitems}
\phantomsection\label{\detokenize{autoapi/flo103_ConvergenceChecker/index:flo103_ConvergenceChecker.flo103_ConvergenceChecker.self.num_cycles}}\pysigline{\sphinxcode{\sphinxupquote{self.}}\sphinxbfcode{\sphinxupquote{num\_cycles}}}
\sphinxAtStartPar
Cycle number

\end{fulllineitems}

\index{n\_runs (flo103\_ConvergenceChecker.flo103\_ConvergenceChecker.self attribute)@\spxentry{n\_runs}\spxextra{flo103\_ConvergenceChecker.flo103\_ConvergenceChecker.self attribute}}

\begin{fulllineitems}
\phantomsection\label{\detokenize{autoapi/flo103_ConvergenceChecker/index:flo103_ConvergenceChecker.flo103_ConvergenceChecker.self.n_runs}}\pysigline{\sphinxcode{\sphinxupquote{self.}}\sphinxbfcode{\sphinxupquote{n\_runs}}}
\sphinxAtStartPar
Number of runs

\end{fulllineitems}

\subsubsection*{Notes}

\sphinxAtStartPar
Later implementation would have more sophisticated convergence.
\index{is\_converged() (flo103\_ConvergenceChecker.flo103\_ConvergenceChecker method)@\spxentry{is\_converged()}\spxextra{flo103\_ConvergenceChecker.flo103\_ConvergenceChecker method}}

\begin{fulllineitems}
\phantomsection\label{\detokenize{autoapi/flo103_ConvergenceChecker/index:flo103_ConvergenceChecker.flo103_ConvergenceChecker.is_converged}}\pysiglinewithargsret{\sphinxbfcode{\sphinxupquote{is\_converged}}}{\emph{\DUrole{n}{self}}, \emph{\DUrole{n}{residuals}}}{}
\sphinxAtStartPar
Cycle number is increased by 1 on each call.
Compared the cycle number to number of runs.
Returns true if all cycles have completed.
Ideally, would have a way to actually monitor convergence.
Then would return true when a certain criterion is met.

\end{fulllineitems}


\end{fulllineitems}



\section{\sphinxstyleliteralintitle{\sphinxupquote{flo103\_PostProcessor}}}
\label{\detokenize{autoapi/flo103_PostProcessor/index:module-flo103_PostProcessor}}\label{\detokenize{autoapi/flo103_PostProcessor/index:flo103-postprocessor}}\label{\detokenize{autoapi/flo103_PostProcessor/index::doc}}\index{module@\spxentry{module}!flo103\_PostProcessor@\spxentry{flo103\_PostProcessor}}\index{flo103\_PostProcessor@\spxentry{flo103\_PostProcessor}!module@\spxentry{module}}
\sphinxAtStartPar
This would provide postprocessing of results.
\begin{description}
\item[{Libraries/Modules:}] \leavevmode
\sphinxAtStartPar
Would use: numpy

\sphinxAtStartPar
Would use: pandas

\end{description}


\subsection{Module Contents}
\label{\detokenize{autoapi/flo103_PostProcessor/index:module-contents}}

\subsubsection{Classes}
\label{\detokenize{autoapi/flo103_PostProcessor/index:classes}}

\begin{savenotes}\sphinxatlongtablestart\begin{longtable}[c]{\X{1}{2}\X{1}{2}}
\hline

\endfirsthead

\multicolumn{2}{c}%
{\makebox[0pt]{\sphinxtablecontinued{\tablename\ \thetable{} \textendash{} continued from previous page}}}\\
\hline

\endhead

\hline
\multicolumn{2}{r}{\makebox[0pt][r]{\sphinxtablecontinued{continues on next page}}}\\
\endfoot

\endlastfoot

\sphinxAtStartPar
{\hyperref[\detokenize{autoapi/flo103_PostProcessor/index:flo103_PostProcessor.flo103_PostProcessor}]{\sphinxcrossref{\sphinxcode{\sphinxupquote{flo103\_PostProcessor}}}}}
&
\sphinxAtStartPar
Not implemented as this time.
\\
\hline
\end{longtable}\sphinxatlongtableend\end{savenotes}
\index{flo103\_PostProcessor (class in flo103\_PostProcessor)@\spxentry{flo103\_PostProcessor}\spxextra{class in flo103\_PostProcessor}}

\begin{fulllineitems}
\phantomsection\label{\detokenize{autoapi/flo103_PostProcessor/index:flo103_PostProcessor.flo103_PostProcessor}}\pysiglinewithargsret{\sphinxbfcode{\sphinxupquote{class }}\sphinxcode{\sphinxupquote{flo103\_PostProcessor.}}\sphinxbfcode{\sphinxupquote{flo103\_PostProcessor}}}{\emph{\DUrole{n}{input}}}{}~\begin{description}
\item[{Not implemented as this time.}] \leavevmode
\sphinxAtStartPar
What it would do:
Plot pressure profile and entropy potential lines.
Plot outputs of wall stress and aerodynamics of airfoil.

\end{description}


\begin{fulllineitems}
\pysigline{\sphinxbfcode{\sphinxupquote{None~currently~used.}}}
\end{fulllineitems}


\sphinxAtStartPar
Notes:
Currently just passed when called.

\end{fulllineitems}



\section{\sphinxstyleliteralintitle{\sphinxupquote{Grid}}}
\label{\detokenize{autoapi/Grid/index:module-Grid}}\label{\detokenize{autoapi/Grid/index:grid}}\label{\detokenize{autoapi/Grid/index::doc}}\index{module@\spxentry{module}!Grid@\spxentry{Grid}}\index{Grid@\spxentry{Grid}!module@\spxentry{module}}
\sphinxAtStartPar
This module contains an abstract base class Grid


\subsection{Module Contents}
\label{\detokenize{autoapi/Grid/index:module-contents}}

\subsubsection{Classes}
\label{\detokenize{autoapi/Grid/index:classes}}

\begin{savenotes}\sphinxatlongtablestart\begin{longtable}[c]{\X{1}{2}\X{1}{2}}
\hline

\endfirsthead

\multicolumn{2}{c}%
{\makebox[0pt]{\sphinxtablecontinued{\tablename\ \thetable{} \textendash{} continued from previous page}}}\\
\hline

\endhead

\hline
\multicolumn{2}{r}{\makebox[0pt][r]{\sphinxtablecontinued{continues on next page}}}\\
\endfoot

\endlastfoot

\sphinxAtStartPar
{\hyperref[\detokenize{autoapi/Grid/index:Grid.Grid}]{\sphinxcrossref{\sphinxcode{\sphinxupquote{Grid}}}}}
&
\sphinxAtStartPar
Abstract base class, never directly instantiated
\\
\hline
\end{longtable}\sphinxatlongtableend\end{savenotes}
\index{Grid (class in Grid)@\spxentry{Grid}\spxextra{class in Grid}}

\begin{fulllineitems}
\phantomsection\label{\detokenize{autoapi/Grid/index:Grid.Grid}}\pysigline{\sphinxbfcode{\sphinxupquote{class }}\sphinxcode{\sphinxupquote{Grid.}}\sphinxbfcode{\sphinxupquote{Grid}}}
\sphinxAtStartPar
Bases: \sphinxcode{\sphinxupquote{abc.ABC}}

\sphinxAtStartPar
Abstract base class, never directly instantiated

\sphinxAtStartPar
AirfoilMap is a child class of this ABC
\index{get\_size() (Grid.Grid method)@\spxentry{get\_size()}\spxextra{Grid.Grid method}}

\begin{fulllineitems}
\phantomsection\label{\detokenize{autoapi/Grid/index:Grid.Grid.get_size}}\pysiglinewithargsret{\sphinxbfcode{\sphinxupquote{abstract }}\sphinxbfcode{\sphinxupquote{get\_size}}}{\emph{\DUrole{n}{self}}}{}
\end{fulllineitems}

\index{get\_geometry() (Grid.Grid method)@\spxentry{get\_geometry()}\spxextra{Grid.Grid method}}

\begin{fulllineitems}
\phantomsection\label{\detokenize{autoapi/Grid/index:Grid.Grid.get_geometry}}\pysiglinewithargsret{\sphinxbfcode{\sphinxupquote{abstract }}\sphinxbfcode{\sphinxupquote{get\_geometry}}}{\emph{\DUrole{n}{self}}}{}
\end{fulllineitems}


\end{fulllineitems}



\section{\sphinxstyleliteralintitle{\sphinxupquote{ImplicitEuler}}}
\label{\detokenize{autoapi/ImplicitEuler/index:module-ImplicitEuler}}\label{\detokenize{autoapi/ImplicitEuler/index:impliciteuler}}\label{\detokenize{autoapi/ImplicitEuler/index::doc}}\index{module@\spxentry{module}!ImplicitEuler@\spxentry{ImplicitEuler}}\index{ImplicitEuler@\spxentry{ImplicitEuler}!module@\spxentry{module}}

\subsection{Module Contents}
\label{\detokenize{autoapi/ImplicitEuler/index:module-contents}}

\subsubsection{Classes}
\label{\detokenize{autoapi/ImplicitEuler/index:classes}}

\begin{savenotes}\sphinxatlongtablestart\begin{longtable}[c]{\X{1}{2}\X{1}{2}}
\hline

\endfirsthead

\multicolumn{2}{c}%
{\makebox[0pt]{\sphinxtablecontinued{\tablename\ \thetable{} \textendash{} continued from previous page}}}\\
\hline

\endhead

\hline
\multicolumn{2}{r}{\makebox[0pt][r]{\sphinxtablecontinued{continues on next page}}}\\
\endfoot

\endlastfoot

\sphinxAtStartPar
{\hyperref[\detokenize{autoapi/ImplicitEuler/index:ImplicitEuler.ImplicitEuler}]{\sphinxcrossref{\sphinxcode{\sphinxupquote{ImplicitEuler}}}}}
&
\sphinxAtStartPar
Implicit Euler mulistage integration scheme
\\
\hline
\end{longtable}\sphinxatlongtableend\end{savenotes}
\index{ImplicitEuler (class in ImplicitEuler)@\spxentry{ImplicitEuler}\spxextra{class in ImplicitEuler}}

\begin{fulllineitems}
\phantomsection\label{\detokenize{autoapi/ImplicitEuler/index:ImplicitEuler.ImplicitEuler}}\pysiglinewithargsret{\sphinxbfcode{\sphinxupquote{class }}\sphinxcode{\sphinxupquote{ImplicitEuler.}}\sphinxbfcode{\sphinxupquote{ImplicitEuler}}}{\emph{\DUrole{n}{model}}, \emph{\DUrole{n}{input}}}{}
\sphinxAtStartPar
Bases: \sphinxcode{\sphinxupquote{bin.Integrator.Integrator}}

\sphinxAtStartPar
Implicit Euler mulistage integration scheme
\begin{description}
\item[{Constructor:}] \leavevmode\begin{description}
\item[{Args:}] \leavevmode
\sphinxAtStartPar
model (Model): physics model
input (Dict): Dictionary with the following items:
\begin{quote}

\sphinxAtStartPar
mstage (int):   number of stages in the multistage integration scheme
cdis:           flux update relaxation factor \textendash{}\textgreater{} 0: no update, 1: full update
cstp:           timestep relaxation factor \textendash{}\textgreater{} 0: no timestep, 1: full step
\end{quote}

\item[{Returns:}] \leavevmode
\sphinxAtStartPar
A new ImplicitEuler object

\end{description}

\end{description}
\index{Model (ImplicitEuler.ImplicitEuler attribute)@\spxentry{Model}\spxextra{ImplicitEuler.ImplicitEuler attribute}}

\begin{fulllineitems}
\phantomsection\label{\detokenize{autoapi/ImplicitEuler/index:ImplicitEuler.ImplicitEuler.Model}}\pysigline{\sphinxbfcode{\sphinxupquote{Model}}}
\sphinxAtStartPar
physics model

\end{fulllineitems}

\index{className (ImplicitEuler.ImplicitEuler attribute)@\spxentry{className}\spxextra{ImplicitEuler.ImplicitEuler attribute}}

\begin{fulllineitems}
\phantomsection\label{\detokenize{autoapi/ImplicitEuler/index:ImplicitEuler.ImplicitEuler.className}}\pysigline{\sphinxbfcode{\sphinxupquote{className}}}
\sphinxAtStartPar
name of class
\begin{quote}\begin{description}
\item[{Type}] \leavevmode
\sphinxAtStartPar
str

\end{description}\end{quote}

\end{fulllineitems}

\index{step() (ImplicitEuler.ImplicitEuler method)@\spxentry{step()}\spxextra{ImplicitEuler.ImplicitEuler method}}

\begin{fulllineitems}
\phantomsection\label{\detokenize{autoapi/ImplicitEuler/index:ImplicitEuler.ImplicitEuler.step}}\pysiglinewithargsret{\sphinxbfcode{\sphinxupquote{step}}}{\emph{\DUrole{n}{self}}, \emph{\DUrole{n}{workspace}}, \emph{\DUrole{n}{state}}, \emph{\DUrole{n}{forcing}\DUrole{o}{=}\DUrole{default_value}{0}}}{}
\sphinxAtStartPar
Returns the local timestep such that stability is maintained.
\begin{quote}\begin{description}
\item[{Parameters}] \leavevmode\begin{itemize}
\item {} 
\sphinxAtStartPar
\sphinxstyleliteralstrong{\sphinxupquote{workspace}} \textendash{} The Workspace object

\item {} 
\sphinxAtStartPar
\sphinxstyleliteralstrong{\sphinxupquote{state}} \textendash{} A Field containing the current state

\item {} 
\sphinxAtStartPar
\sphinxstyleliteralstrong{\sphinxupquote{forcing}} \textendash{} Field of values on the right hand side of the equation that “force” the ODE

\end{itemize}

\end{description}\end{quote}

\end{fulllineitems}

\index{\_\_check\_vars() (ImplicitEuler.ImplicitEuler method)@\spxentry{\_\_check\_vars()}\spxextra{ImplicitEuler.ImplicitEuler method}}

\begin{fulllineitems}
\phantomsection\label{\detokenize{autoapi/ImplicitEuler/index:ImplicitEuler.ImplicitEuler.__check_vars}}\pysiglinewithargsret{\sphinxbfcode{\sphinxupquote{\_\_check\_vars}}}{\emph{\DUrole{n}{self}}, \emph{\DUrole{n}{workspace}}}{}
\end{fulllineitems}

\index{\_\_init\_vars() (ImplicitEuler.ImplicitEuler method)@\spxentry{\_\_init\_vars()}\spxextra{ImplicitEuler.ImplicitEuler method}}

\begin{fulllineitems}
\phantomsection\label{\detokenize{autoapi/ImplicitEuler/index:ImplicitEuler.ImplicitEuler.__init_vars}}\pysiglinewithargsret{\sphinxbfcode{\sphinxupquote{\_\_init\_vars}}}{\emph{\DUrole{n}{self}}, \emph{\DUrole{n}{workspace}}}{}
\end{fulllineitems}


\end{fulllineitems}



\section{\sphinxstyleliteralintitle{\sphinxupquote{Input}}}
\label{\detokenize{autoapi/Input/index:module-Input}}\label{\detokenize{autoapi/Input/index:input}}\label{\detokenize{autoapi/Input/index::doc}}\index{module@\spxentry{module}!Input@\spxentry{Input}}\index{Input@\spxentry{Input}!module@\spxentry{module}}
\sphinxAtStartPar
This module unpacks the input from a .data file to dictionaries in an input object
\begin{description}
\item[{Libraries/Modules:}] \leavevmode
\sphinxAtStartPar
numpy

\sphinxAtStartPar
pandas

\end{description}


\subsection{Module Contents}
\label{\detokenize{autoapi/Input/index:module-contents}}

\subsubsection{Classes}
\label{\detokenize{autoapi/Input/index:classes}}

\begin{savenotes}\sphinxatlongtablestart\begin{longtable}[c]{\X{1}{2}\X{1}{2}}
\hline

\endfirsthead

\multicolumn{2}{c}%
{\makebox[0pt]{\sphinxtablecontinued{\tablename\ \thetable{} \textendash{} continued from previous page}}}\\
\hline

\endhead

\hline
\multicolumn{2}{r}{\makebox[0pt][r]{\sphinxtablecontinued{continues on next page}}}\\
\endfoot

\endlastfoot

\sphinxAtStartPar
{\hyperref[\detokenize{autoapi/Input/index:Input.Input}]{\sphinxcrossref{\sphinxcode{\sphinxupquote{Input}}}}}
&
\sphinxAtStartPar
Reads in .data file and unpacks the parameter into dictionaries.
\\
\hline
\end{longtable}\sphinxatlongtableend\end{savenotes}
\index{Input (class in Input)@\spxentry{Input}\spxextra{class in Input}}

\begin{fulllineitems}
\phantomsection\label{\detokenize{autoapi/Input/index:Input.Input}}\pysiglinewithargsret{\sphinxbfcode{\sphinxupquote{class }}\sphinxcode{\sphinxupquote{Input.}}\sphinxbfcode{\sphinxupquote{Input}}}{\emph{\DUrole{n}{filename}}}{}
\sphinxAtStartPar
Reads in .data file and unpacks the parameter into dictionaries.
\begin{description}
\item[{Constructor:}] \leavevmode\begin{description}
\item[{Args:}] \leavevmode
\sphinxAtStartPar
filename (str): Input .data file with input params and airfoil geometry

\item[{Returns:}] \leavevmode
\sphinxAtStartPar
A new Input object containing five dicts \sphinxhyphen{} dims, solv\_param, flo\_param, geo\_param and in\_var

\item[{Notes:}] \leavevmode
\sphinxAtStartPar
Check top of Input.py file to see the contents of each of the five dictionanries

\end{description}

\end{description}
\index{dim\_p (Input.Input attribute)@\spxentry{dim\_p}\spxextra{Input.Input attribute}}

\begin{fulllineitems}
\phantomsection\label{\detokenize{autoapi/Input/index:Input.Input.dim_p}}\pysigline{\sphinxbfcode{\sphinxupquote{dim\_p}}}
\sphinxAtStartPar
List of paramters to get from input file to dims dict.
\begin{quote}\begin{description}
\item[{Type}] \leavevmode
\sphinxAtStartPar
list

\end{description}\end{quote}

\end{fulllineitems}

\index{solv\_p (Input.Input attribute)@\spxentry{solv\_p}\spxextra{Input.Input attribute}}

\begin{fulllineitems}
\phantomsection\label{\detokenize{autoapi/Input/index:Input.Input.solv_p}}\pysigline{\sphinxbfcode{\sphinxupquote{solv\_p}}}
\sphinxAtStartPar
List of paramters to get from input file to solv\_param dict.
\begin{quote}\begin{description}
\item[{Type}] \leavevmode
\sphinxAtStartPar
list

\end{description}\end{quote}

\end{fulllineitems}

\index{flo\_p (Input.Input attribute)@\spxentry{flo\_p}\spxextra{Input.Input attribute}}

\begin{fulllineitems}
\phantomsection\label{\detokenize{autoapi/Input/index:Input.Input.flo_p}}\pysigline{\sphinxbfcode{\sphinxupquote{flo\_p}}}
\sphinxAtStartPar
List of paramters to get from input file to flo\_param dict.
\begin{quote}\begin{description}
\item[{Type}] \leavevmode
\sphinxAtStartPar
list

\end{description}\end{quote}

\end{fulllineitems}

\index{geo\_p (Input.Input attribute)@\spxentry{geo\_p}\spxextra{Input.Input attribute}}

\begin{fulllineitems}
\phantomsection\label{\detokenize{autoapi/Input/index:Input.Input.geo_p}}\pysigline{\sphinxbfcode{\sphinxupquote{geo\_p}}}
\sphinxAtStartPar
List of paramters to get from input file to geo\_param dict.
\begin{quote}\begin{description}
\item[{Type}] \leavevmode
\sphinxAtStartPar
list

\end{description}\end{quote}

\end{fulllineitems}

\index{in\_v (Input.Input attribute)@\spxentry{in\_v}\spxextra{Input.Input attribute}}

\begin{fulllineitems}
\phantomsection\label{\detokenize{autoapi/Input/index:Input.Input.in_v}}\pysigline{\sphinxbfcode{\sphinxupquote{in\_v}}}
\sphinxAtStartPar
List of paramters to get from input file to in\_var dict.
\begin{quote}\begin{description}
\item[{Type}] \leavevmode
\sphinxAtStartPar
list

\end{description}\end{quote}

\end{fulllineitems}


\begin{sphinxadmonition}{note}{Note:}
\sphinxAtStartPar
Check top of Input.py file to see the contents of each of the five dictionanries.
\end{sphinxadmonition}
\index{dim\_p (Input.Input attribute)@\spxentry{dim\_p}\spxextra{Input.Input attribute}}

\begin{fulllineitems}
\phantomsection\label{\detokenize{autoapi/Input/index:id0}}\pysigline{\sphinxbfcode{\sphinxupquote{dim\_p}}\sphinxbfcode{\sphinxupquote{ = {[}{[}\textquotesingle{}nx\textquotesingle{}, \textquotesingle{}ny\textquotesingle{}{]}{]}}}}
\end{fulllineitems}

\index{solv\_p (Input.Input attribute)@\spxentry{solv\_p}\spxextra{Input.Input attribute}}

\begin{fulllineitems}
\phantomsection\label{\detokenize{autoapi/Input/index:id1}}\pysigline{\sphinxbfcode{\sphinxupquote{solv\_p}}\sphinxbfcode{\sphinxupquote{ = {[}{[}\textquotesingle{}fcyc\textquotesingle{}, \textquotesingle{}fprnt\textquotesingle{}, \textquotesingle{}fout\textquotesingle{}, \textquotesingle{}ftim\textquotesingle{}, \textquotesingle{}gprnt\textquotesingle{}, \textquotesingle{}hprnt\textquotesingle{}, \textquotesingle{}hmesh\textquotesingle{}{]}, {[}\textquotesingle{}cflf\textquotesingle{}, \textquotesingle{}cflim\textquotesingle{}, \textquotesingle{}vis2\textquotesingle{}, \textquotesingle{}vis4\textquotesingle{},...}}}
\end{fulllineitems}

\index{flo\_p (Input.Input attribute)@\spxentry{flo\_p}\spxextra{Input.Input attribute}}

\begin{fulllineitems}
\phantomsection\label{\detokenize{autoapi/Input/index:id2}}\pysigline{\sphinxbfcode{\sphinxupquote{flo\_p}}\sphinxbfcode{\sphinxupquote{ = {[}{[}\textquotesingle{}rm\textquotesingle{}, \textquotesingle{}al\textquotesingle{}, \textquotesingle{}fcl\textquotesingle{}, \textquotesingle{}clt\textquotesingle{}, \textquotesingle{}cd0\textquotesingle{}{]}, {[}\textquotesingle{}re\textquotesingle{}, \textquotesingle{}prn\textquotesingle{}, \textquotesingle{}prt\textquotesingle{}, \textquotesingle{}t0\textquotesingle{}, \textquotesingle{}xtran\textquotesingle{}, \textquotesingle{}kvis\textquotesingle{}{]}{]}}}}
\end{fulllineitems}

\index{geo\_p (Input.Input attribute)@\spxentry{geo\_p}\spxextra{Input.Input attribute}}

\begin{fulllineitems}
\phantomsection\label{\detokenize{autoapi/Input/index:id3}}\pysigline{\sphinxbfcode{\sphinxupquote{geo\_p}}\sphinxbfcode{\sphinxupquote{ = {[}{[}\textquotesingle{}boundx\textquotesingle{}, \textquotesingle{}boundy\textquotesingle{}, \textquotesingle{}bunch\textquotesingle{}{]}, {[}\textquotesingle{}xte\textquotesingle{}, \textquotesingle{}ylim1\textquotesingle{}, \textquotesingle{}ylim2\textquotesingle{}, \textquotesingle{}ax\textquotesingle{}, \textquotesingle{}ay\textquotesingle{}, \textquotesingle{}sy\textquotesingle{}{]}, {[}\textquotesingle{}aplus\textquotesingle{}, \textquotesingle{}ncut\textquotesingle{}{]},...}}}
\end{fulllineitems}

\index{in\_v (Input.Input attribute)@\spxentry{in\_v}\spxextra{Input.Input attribute}}

\begin{fulllineitems}
\phantomsection\label{\detokenize{autoapi/Input/index:id4}}\pysigline{\sphinxbfcode{\sphinxupquote{in\_v}}\sphinxbfcode{\sphinxupquote{ = {[}\textquotesingle{}xn\textquotesingle{}, \textquotesingle{}yn\textquotesingle{}{]}}}}
\end{fulllineitems}

\index{max\_no\_cols() (Input.Input method)@\spxentry{max\_no\_cols()}\spxextra{Input.Input method}}

\begin{fulllineitems}
\phantomsection\label{\detokenize{autoapi/Input/index:Input.Input.max_no_cols}}\pysiglinewithargsret{\sphinxbfcode{\sphinxupquote{max\_no\_cols}}}{\emph{\DUrole{n}{self}}, \emph{\DUrole{n}{file}}}{}
\sphinxAtStartPar
Finds max number of columns in a row in the input file.
\begin{quote}\begin{description}
\item[{Parameters}] \leavevmode
\sphinxAtStartPar
\sphinxstyleliteralstrong{\sphinxupquote{file}} (\sphinxstyleliteralemphasis{\sphinxupquote{str}}) \textendash{} .data input file

\end{description}\end{quote}

\end{fulllineitems}

\index{read() (Input.Input method)@\spxentry{read()}\spxextra{Input.Input method}}

\begin{fulllineitems}
\phantomsection\label{\detokenize{autoapi/Input/index:Input.Input.read}}\pysiglinewithargsret{\sphinxbfcode{\sphinxupquote{read}}}{\emph{\DUrole{n}{self}}, \emph{\DUrole{n}{file}}, \emph{\DUrole{n}{max\_cols}}}{}
\sphinxAtStartPar
Reads in .data file using pandas.
\begin{quote}\begin{description}
\item[{Parameters}] \leavevmode\begin{itemize}
\item {} 
\sphinxAtStartPar
\sphinxstyleliteralstrong{\sphinxupquote{file}} (\sphinxstyleliteralemphasis{\sphinxupquote{str}}) \textendash{} .data input file

\item {} 
\sphinxAtStartPar
\sphinxstyleliteralstrong{\sphinxupquote{max\_cols}} \textendash{} maximum number of colums of all the rows

\end{itemize}

\end{description}\end{quote}

\end{fulllineitems}

\index{update\_dict() (Input.Input method)@\spxentry{update\_dict()}\spxextra{Input.Input method}}

\begin{fulllineitems}
\phantomsection\label{\detokenize{autoapi/Input/index:Input.Input.update_dict}}\pysiglinewithargsret{\sphinxbfcode{\sphinxupquote{update\_dict}}}{\emph{\DUrole{n}{self}}, \emph{\DUrole{n}{df}}, \emph{\DUrole{n}{dict}}, \emph{\DUrole{n}{params}}, \emph{\DUrole{n}{strt\_row}}}{}
\sphinxAtStartPar
Slices through pandas dataframe to unpack input params into the a dict.
\begin{quote}\begin{description}
\item[{Parameters}] \leavevmode\begin{itemize}
\item {} 
\sphinxAtStartPar
\sphinxstyleliteralstrong{\sphinxupquote{file}} (\sphinxstyleliteralemphasis{\sphinxupquote{str}}) \textendash{} .data input file

\item {} 
\sphinxAtStartPar
\sphinxstyleliteralstrong{\sphinxupquote{df}} \textendash{} pandas data frame of .data input file

\item {} 
\sphinxAtStartPar
\sphinxstyleliteralstrong{\sphinxupquote{dict}} \textendash{} dictionary to assing values to

\item {} 
\sphinxAtStartPar
\sphinxstyleliteralstrong{\sphinxupquote{params}} \textendash{} list of params to assign to dictionary

\item {} 
\sphinxAtStartPar
\sphinxstyleliteralstrong{\sphinxupquote{str\_row}} \textendash{} row of df to start unpacking from

\end{itemize}

\end{description}\end{quote}

\end{fulllineitems}

\index{update\_geom() (Input.Input method)@\spxentry{update\_geom()}\spxextra{Input.Input method}}

\begin{fulllineitems}
\phantomsection\label{\detokenize{autoapi/Input/index:Input.Input.update_geom}}\pysiglinewithargsret{\sphinxbfcode{\sphinxupquote{update\_geom}}}{\emph{\DUrole{n}{self}}, \emph{\DUrole{n}{df}}, \emph{\DUrole{n}{dict}}, \emph{\DUrole{n}{params}}, \emph{\DUrole{n}{strt\_row}}}{}
\sphinxAtStartPar
Slices through pandas dataframe to unpack airfoil geometry.
\begin{quote}\begin{description}
\item[{Parameters}] \leavevmode\begin{itemize}
\item {} 
\sphinxAtStartPar
\sphinxstyleliteralstrong{\sphinxupquote{file}} (\sphinxstyleliteralemphasis{\sphinxupquote{str}}) \textendash{} .data input file

\item {} 
\sphinxAtStartPar
\sphinxstyleliteralstrong{\sphinxupquote{df}} \textendash{} pandas data frame of .data input file

\item {} 
\sphinxAtStartPar
\sphinxstyleliteralstrong{\sphinxupquote{dict}} \textendash{} dictionary to assing values to

\item {} 
\sphinxAtStartPar
\sphinxstyleliteralstrong{\sphinxupquote{params}} \textendash{} list of params to assign to dictionary

\item {} 
\sphinxAtStartPar
\sphinxstyleliteralstrong{\sphinxupquote{str\_row}} \textendash{} row of df to start unpacking from

\end{itemize}

\end{description}\end{quote}

\end{fulllineitems}

\index{add\_dicts() (Input.Input method)@\spxentry{add\_dicts()}\spxextra{Input.Input method}}

\begin{fulllineitems}
\phantomsection\label{\detokenize{autoapi/Input/index:Input.Input.add_dicts}}\pysiglinewithargsret{\sphinxbfcode{\sphinxupquote{add\_dicts}}}{\emph{\DUrole{n}{self}}, \emph{\DUrole{n}{dict1}}, \emph{\DUrole{n}{dict2}}}{}
\sphinxAtStartPar
Merge two dicts into on dict .
\begin{quote}\begin{description}
\item[{Parameters}] \leavevmode\begin{itemize}
\item {} 
\sphinxAtStartPar
\sphinxstyleliteralstrong{\sphinxupquote{dict1}} \textendash{} first dictionary to merge

\item {} 
\sphinxAtStartPar
\sphinxstyleliteralstrong{\sphinxupquote{dict2}} \textendash{} second dictionary to merge

\end{itemize}

\end{description}\end{quote}

\end{fulllineitems}


\end{fulllineitems}



\section{\sphinxstyleliteralintitle{\sphinxupquote{Integrator}}}
\label{\detokenize{autoapi/Integrator/index:module-Integrator}}\label{\detokenize{autoapi/Integrator/index:integrator}}\label{\detokenize{autoapi/Integrator/index::doc}}\index{module@\spxentry{module}!Integrator@\spxentry{Integrator}}\index{Integrator@\spxentry{Integrator}!module@\spxentry{module}}

\subsection{Module Contents}
\label{\detokenize{autoapi/Integrator/index:module-contents}}

\subsubsection{Classes}
\label{\detokenize{autoapi/Integrator/index:classes}}

\begin{savenotes}\sphinxatlongtablestart\begin{longtable}[c]{\X{1}{2}\X{1}{2}}
\hline

\endfirsthead

\multicolumn{2}{c}%
{\makebox[0pt]{\sphinxtablecontinued{\tablename\ \thetable{} \textendash{} continued from previous page}}}\\
\hline

\endhead

\hline
\multicolumn{2}{r}{\makebox[0pt][r]{\sphinxtablecontinued{continues on next page}}}\\
\endfoot

\endlastfoot

\sphinxAtStartPar
{\hyperref[\detokenize{autoapi/Integrator/index:Integrator.Integrator}]{\sphinxcrossref{\sphinxcode{\sphinxupquote{Integrator}}}}}
&
\sphinxAtStartPar
Abstract base class, never directly instantiated
\\
\hline
\end{longtable}\sphinxatlongtableend\end{savenotes}
\index{Integrator (class in Integrator)@\spxentry{Integrator}\spxextra{class in Integrator}}

\begin{fulllineitems}
\phantomsection\label{\detokenize{autoapi/Integrator/index:Integrator.Integrator}}\pysiglinewithargsret{\sphinxbfcode{\sphinxupquote{class }}\sphinxcode{\sphinxupquote{Integrator.}}\sphinxbfcode{\sphinxupquote{Integrator}}}{\emph{\DUrole{n}{model}}, \emph{\DUrole{n}{input}}}{}
\sphinxAtStartPar
Bases: \sphinxcode{\sphinxupquote{abc.ABC}}

\sphinxAtStartPar
Abstract base class, never directly instantiated
NS\_Airfoil is a child class of this ABC
\begin{description}
\item[{Constructor:}] \leavevmode\begin{description}
\item[{Args:}] \leavevmode
\sphinxAtStartPar
model (Model): physics model
input: necessary input parameters

\end{description}

\end{description}
\index{step() (Integrator.Integrator method)@\spxentry{step()}\spxextra{Integrator.Integrator method}}

\begin{fulllineitems}
\phantomsection\label{\detokenize{autoapi/Integrator/index:Integrator.Integrator.step}}\pysiglinewithargsret{\sphinxbfcode{\sphinxupquote{abstract }}\sphinxbfcode{\sphinxupquote{step}}}{\emph{\DUrole{n}{self}}, \emph{\DUrole{n}{workspace}}, \emph{\DUrole{n}{state}}, \emph{\DUrole{n}{forcing}}}{}
\sphinxAtStartPar
Returns the local timestep such that stability is maintained.
\begin{quote}\begin{description}
\item[{Parameters}] \leavevmode\begin{itemize}
\item {} 
\sphinxAtStartPar
\sphinxstyleliteralstrong{\sphinxupquote{workspace}} \textendash{} The Workspace object

\item {} 
\sphinxAtStartPar
\sphinxstyleliteralstrong{\sphinxupquote{state}} \textendash{} A Field containing the current state

\item {} 
\sphinxAtStartPar
\sphinxstyleliteralstrong{\sphinxupquote{forcing}} \textendash{} Field of values on the right hand side of the equation that “force” the ODE

\end{itemize}

\end{description}\end{quote}

\end{fulllineitems}


\end{fulllineitems}



\section{\sphinxstyleliteralintitle{\sphinxupquote{Model}}}
\label{\detokenize{autoapi/Model/index:module-Model}}\label{\detokenize{autoapi/Model/index:model}}\label{\detokenize{autoapi/Model/index::doc}}\index{module@\spxentry{module}!Model@\spxentry{Model}}\index{Model@\spxentry{Model}!module@\spxentry{module}}

\subsection{Module Contents}
\label{\detokenize{autoapi/Model/index:module-contents}}

\subsubsection{Classes}
\label{\detokenize{autoapi/Model/index:classes}}

\begin{savenotes}\sphinxatlongtablestart\begin{longtable}[c]{\X{1}{2}\X{1}{2}}
\hline

\endfirsthead

\multicolumn{2}{c}%
{\makebox[0pt]{\sphinxtablecontinued{\tablename\ \thetable{} \textendash{} continued from previous page}}}\\
\hline

\endhead

\hline
\multicolumn{2}{r}{\makebox[0pt][r]{\sphinxtablecontinued{continues on next page}}}\\
\endfoot

\endlastfoot

\sphinxAtStartPar
{\hyperref[\detokenize{autoapi/Model/index:Model.Model}]{\sphinxcrossref{\sphinxcode{\sphinxupquote{Model}}}}}
&
\sphinxAtStartPar
Abstract base class for a physics model. never to be instantiated.
\\
\hline
\end{longtable}\sphinxatlongtableend\end{savenotes}
\index{Model (class in Model)@\spxentry{Model}\spxextra{class in Model}}

\begin{fulllineitems}
\phantomsection\label{\detokenize{autoapi/Model/index:Model.Model}}\pysiglinewithargsret{\sphinxbfcode{\sphinxupquote{class }}\sphinxcode{\sphinxupquote{Model.}}\sphinxbfcode{\sphinxupquote{Model}}}{\emph{\DUrole{n}{bcmodel}}, \emph{\DUrole{n}{input}}}{}
\sphinxAtStartPar
Bases: \sphinxcode{\sphinxupquote{abc.ABC}}

\sphinxAtStartPar
Abstract base class for a physics model. never to be instantiated.
\begin{description}
\item[{Constructor:}] \leavevmode\begin{description}
\item[{Args:}] \leavevmode
\sphinxAtStartPar
bcmodel (BoundaryConditioner): Boundary condition model

\item[{Returns:}] \leavevmode
\sphinxAtStartPar
A new Model object

\item[{Notes:}] \leavevmode
\sphinxAtStartPar
Check top of Input.py file to see the contents of each of the five dictionanries

\end{description}

\end{description}

\sphinxAtStartPar
Attributes:
\index{init\_state() (Model.Model method)@\spxentry{init\_state()}\spxextra{Model.Model method}}

\begin{fulllineitems}
\phantomsection\label{\detokenize{autoapi/Model/index:Model.Model.init_state}}\pysiglinewithargsret{\sphinxbfcode{\sphinxupquote{abstract }}\sphinxbfcode{\sphinxupquote{init\_state}}}{\emph{\DUrole{n}{self}}, \emph{\DUrole{n}{workspace}}, \emph{\DUrole{n}{state}}}{}
\end{fulllineitems}

\index{get\_flux() (Model.Model method)@\spxentry{get\_flux()}\spxextra{Model.Model method}}

\begin{fulllineitems}
\phantomsection\label{\detokenize{autoapi/Model/index:Model.Model.get_flux}}\pysiglinewithargsret{\sphinxbfcode{\sphinxupquote{abstract }}\sphinxbfcode{\sphinxupquote{get\_flux}}}{\emph{\DUrole{n}{self}}, \emph{\DUrole{n}{workspace}}, \emph{\DUrole{n}{state}}, \emph{\DUrole{n}{output}}, \emph{\DUrole{n}{update\_factor}\DUrole{o}{=}\DUrole{default_value}{1}}}{}
\end{fulllineitems}

\index{update\_cfl\_limit() (Model.Model method)@\spxentry{update\_cfl\_limit()}\spxextra{Model.Model method}}

\begin{fulllineitems}
\phantomsection\label{\detokenize{autoapi/Model/index:Model.Model.update_cfl_limit}}\pysiglinewithargsret{\sphinxbfcode{\sphinxupquote{update\_cfl\_limit}}}{\emph{\DUrole{n}{self}}, \emph{\DUrole{n}{cfl\_lim}\DUrole{o}{=}\DUrole{default_value}{Infinity}}}{}
\end{fulllineitems}

\index{get\_safe\_timestep() (Model.Model method)@\spxentry{get\_safe\_timestep()}\spxextra{Model.Model method}}

\begin{fulllineitems}
\phantomsection\label{\detokenize{autoapi/Model/index:Model.Model.get_safe_timestep}}\pysiglinewithargsret{\sphinxbfcode{\sphinxupquote{abstract }}\sphinxbfcode{\sphinxupquote{get\_safe\_timestep}}}{\emph{\DUrole{n}{self}}, \emph{\DUrole{n}{workspace}}, \emph{\DUrole{n}{state}}, \emph{\DUrole{n}{dt}}}{}
\end{fulllineitems}

\index{update\_physics() (Model.Model method)@\spxentry{update\_physics()}\spxextra{Model.Model method}}

\begin{fulllineitems}
\phantomsection\label{\detokenize{autoapi/Model/index:Model.Model.update_physics}}\pysiglinewithargsret{\sphinxbfcode{\sphinxupquote{abstract }}\sphinxbfcode{\sphinxupquote{update\_physics}}}{\emph{\DUrole{n}{self}}, \emph{\DUrole{n}{workspace}}, \emph{\DUrole{n}{state}}}{}
\end{fulllineitems}

\index{update\_stability() (Model.Model method)@\spxentry{update\_stability()}\spxextra{Model.Model method}}

\begin{fulllineitems}
\phantomsection\label{\detokenize{autoapi/Model/index:Model.Model.update_stability}}\pysiglinewithargsret{\sphinxbfcode{\sphinxupquote{abstract }}\sphinxbfcode{\sphinxupquote{update\_stability}}}{\emph{\DUrole{n}{self}}, \emph{\DUrole{n}{workspace}}, \emph{\DUrole{n}{state}}}{}
\end{fulllineitems}

\index{transfer\_down() (Model.Model method)@\spxentry{transfer\_down()}\spxextra{Model.Model method}}

\begin{fulllineitems}
\phantomsection\label{\detokenize{autoapi/Model/index:Model.Model.transfer_down}}\pysiglinewithargsret{\sphinxbfcode{\sphinxupquote{abstract }}\sphinxbfcode{\sphinxupquote{transfer\_down}}}{\emph{\DUrole{n}{self}}, \emph{\DUrole{n}{workspace1}}, \emph{\DUrole{n}{workspace2}}}{}
\end{fulllineitems}

\index{dim() (Model.Model method)@\spxentry{dim()}\spxextra{Model.Model method}}

\begin{fulllineitems}
\phantomsection\label{\detokenize{autoapi/Model/index:Model.Model.dim}}\pysiglinewithargsret{\sphinxbfcode{\sphinxupquote{dim}}}{\emph{\DUrole{n}{self}}}{}
\end{fulllineitems}


\end{fulllineitems}



\section{\sphinxstyleliteralintitle{\sphinxupquote{MultiGrid}}}
\label{\detokenize{autoapi/MultiGrid/index:module-MultiGrid}}\label{\detokenize{autoapi/MultiGrid/index:multigrid}}\label{\detokenize{autoapi/MultiGrid/index::doc}}\index{module@\spxentry{module}!MultiGrid@\spxentry{MultiGrid}}\index{MultiGrid@\spxentry{MultiGrid}!module@\spxentry{module}}

\subsection{Module Contents}
\label{\detokenize{autoapi/MultiGrid/index:module-contents}}

\subsubsection{Classes}
\label{\detokenize{autoapi/MultiGrid/index:classes}}

\begin{savenotes}\sphinxatlongtablestart\begin{longtable}[c]{\X{1}{2}\X{1}{2}}
\hline

\endfirsthead

\multicolumn{2}{c}%
{\makebox[0pt]{\sphinxtablecontinued{\tablename\ \thetable{} \textendash{} continued from previous page}}}\\
\hline

\endhead

\hline
\multicolumn{2}{r}{\makebox[0pt][r]{\sphinxtablecontinued{continues on next page}}}\\
\endfoot

\endlastfoot

\sphinxAtStartPar
{\hyperref[\detokenize{autoapi/MultiGrid/index:MultiGrid.MultiGrid}]{\sphinxcrossref{\sphinxcode{\sphinxupquote{MultiGrid}}}}}
&
\sphinxAtStartPar
Uses mulitple coarser grids to apply corrections to the values on the current grid.
\\
\hline
\end{longtable}\sphinxatlongtableend\end{savenotes}
\index{MultiGrid (class in MultiGrid)@\spxentry{MultiGrid}\spxextra{class in MultiGrid}}

\begin{fulllineitems}
\phantomsection\label{\detokenize{autoapi/MultiGrid/index:MultiGrid.MultiGrid}}\pysiglinewithargsret{\sphinxbfcode{\sphinxupquote{class }}\sphinxcode{\sphinxupquote{MultiGrid.}}\sphinxbfcode{\sphinxupquote{MultiGrid}}}{\emph{\DUrole{n}{workspace}}, \emph{\DUrole{n}{model}}, \emph{\DUrole{n}{integrator}}, \emph{\DUrole{n}{input}}}{}~\begin{description}
\item[{Uses mulitple coarser grids to apply corrections to the values on the current grid.}] \leavevmode
\sphinxAtStartPar
Correcting the soluion this way allows for much faster convergence to be achieved
than if the solution were only updated with the finer grid

\item[{Constructor:}] \leavevmode\begin{description}
\item[{Args:}] \leavevmode\begin{description}
\item[{workspace (Workspace):}] \leavevmode
\sphinxAtStartPar
The workspace corresponding to the grid on which the solution will be calculated

\item[{model (Model):}] \leavevmode
\sphinxAtStartPar
The physics model to be used

\item[{integrator (Integrator):}] \leavevmode
\sphinxAtStartPar
The integration scheme to be used

\item[{input (Dict):}] \leavevmode\begin{description}
\item[{Dictionary of parameters containing:}] \leavevmode
\sphinxAtStartPar
ftim: the interval at which the stability will be updated
fcoll: the relaxation factor on the residuals transferred from the finer mesh

\end{description}

\end{description}

\item[{Returns:}] \leavevmode
\sphinxAtStartPar
A new Input object containing five dicts \sphinxhyphen{} dims, solv\_param, flo\_param, geo\_param and in\_var

\item[{Notes:}] \leavevmode
\sphinxAtStartPar
Check top of Input.py file to see the contents of each of the five dictionanries

\end{description}

\end{description}
\index{performCycle() (MultiGrid.MultiGrid method)@\spxentry{performCycle()}\spxextra{MultiGrid.MultiGrid method}}

\begin{fulllineitems}
\phantomsection\label{\detokenize{autoapi/MultiGrid/index:MultiGrid.MultiGrid.performCycle}}\pysiglinewithargsret{\sphinxbfcode{\sphinxupquote{performCycle}}}{\emph{\DUrole{n}{self}}}{}
\sphinxAtStartPar
Performs one multi\sphinxhyphen{}grid cycle and calculates new state.

\end{fulllineitems}

\index{residuals() (MultiGrid.MultiGrid method)@\spxentry{residuals()}\spxextra{MultiGrid.MultiGrid method}}

\begin{fulllineitems}
\phantomsection\label{\detokenize{autoapi/MultiGrid/index:MultiGrid.MultiGrid.residuals}}\pysiglinewithargsret{\sphinxbfcode{\sphinxupquote{residuals}}}{\emph{\DUrole{n}{self}}, \emph{\DUrole{n}{output}}}{}
\sphinxAtStartPar
Copies the residual values to the output Field.
\begin{quote}\begin{description}
\item[{Parameters}] \leavevmode
\sphinxAtStartPar
\sphinxstyleliteralstrong{\sphinxupquote{output}} ({\hyperref[\detokenize{autoapi/Field/index:Field.Field}]{\sphinxcrossref{\sphinxstyleliteralemphasis{\sphinxupquote{Field}}}}}) \textendash{} Field that will store the residual values

\end{description}\end{quote}

\end{fulllineitems}

\index{solution() (MultiGrid.MultiGrid method)@\spxentry{solution()}\spxextra{MultiGrid.MultiGrid method}}

\begin{fulllineitems}
\phantomsection\label{\detokenize{autoapi/MultiGrid/index:MultiGrid.MultiGrid.solution}}\pysiglinewithargsret{\sphinxbfcode{\sphinxupquote{solution}}}{\emph{\DUrole{n}{self}}, \emph{\DUrole{n}{output}}}{}
\sphinxAtStartPar
Copies the state values to the output Field.
\begin{quote}\begin{description}
\item[{Parameters}] \leavevmode
\sphinxAtStartPar
\sphinxstyleliteralstrong{\sphinxupquote{output}} ({\hyperref[\detokenize{autoapi/Field/index:Field.Field}]{\sphinxcrossref{\sphinxstyleliteralemphasis{\sphinxupquote{Field}}}}}) \textendash{} Field that will store the values

\end{description}\end{quote}

\end{fulllineitems}


\end{fulllineitems}



\section{\sphinxstyleliteralintitle{\sphinxupquote{NavierStokes}}}
\label{\detokenize{autoapi/NavierStokes/index:module-NavierStokes}}\label{\detokenize{autoapi/NavierStokes/index:navierstokes}}\label{\detokenize{autoapi/NavierStokes/index::doc}}\index{module@\spxentry{module}!NavierStokes@\spxentry{NavierStokes}}\index{NavierStokes@\spxentry{NavierStokes}!module@\spxentry{module}}

\subsection{Module Contents}
\label{\detokenize{autoapi/NavierStokes/index:module-contents}}

\subsubsection{Classes}
\label{\detokenize{autoapi/NavierStokes/index:classes}}

\begin{savenotes}\sphinxatlongtablestart\begin{longtable}[c]{\X{1}{2}\X{1}{2}}
\hline

\endfirsthead

\multicolumn{2}{c}%
{\makebox[0pt]{\sphinxtablecontinued{\tablename\ \thetable{} \textendash{} continued from previous page}}}\\
\hline

\endhead

\hline
\multicolumn{2}{r}{\makebox[0pt][r]{\sphinxtablecontinued{continues on next page}}}\\
\endfoot

\endlastfoot

\sphinxAtStartPar
{\hyperref[\detokenize{autoapi/NavierStokes/index:NavierStokes.NavierStokes}]{\sphinxcrossref{\sphinxcode{\sphinxupquote{NavierStokes}}}}}
&
\sphinxAtStartPar
Physics model for fluid flow based on the Reynolds Averaged Navier Stokes (RANS) equations
\\
\hline
\end{longtable}\sphinxatlongtableend\end{savenotes}
\index{NavierStokes (class in NavierStokes)@\spxentry{NavierStokes}\spxextra{class in NavierStokes}}

\begin{fulllineitems}
\phantomsection\label{\detokenize{autoapi/NavierStokes/index:NavierStokes.NavierStokes}}\pysiglinewithargsret{\sphinxbfcode{\sphinxupquote{class }}\sphinxcode{\sphinxupquote{NavierStokes.}}\sphinxbfcode{\sphinxupquote{NavierStokes}}}{\emph{\DUrole{n}{bcmodel}}, \emph{\DUrole{n}{input}}}{}
\sphinxAtStartPar
Bases: \sphinxcode{\sphinxupquote{bin.Model.Model}}

\sphinxAtStartPar
Physics model for fluid flow based on the Reynolds Averaged Navier Stokes (RANS) equations
for use in a multigrid scheme. The state w is composed of Fields with density, x\sphinxhyphen{}momentum, y\sphinxhyphen{}momentum and energy.
Based on a finite volume formulation with ability to compute fluxes, update scheme stable timestep,
and update eddy viscocities. Contains terms for convective, artificial dissipative, and viscous fluxes.
\begin{description}
\item[{Constructor:}] \leavevmode\begin{description}
\item[{Args:}] \leavevmode
\sphinxAtStartPar
bcmodel (BoundaryConditioner): instance of BoundaryConditioner class
input (dictionary): dictionary with parameter values

\item[{Returns:}] \leavevmode
\sphinxAtStartPar
A new NavierStokes object

\end{description}

\end{description}
\index{className (NavierStokes.NavierStokes attribute)@\spxentry{className}\spxextra{NavierStokes.NavierStokes attribute}}

\begin{fulllineitems}
\phantomsection\label{\detokenize{autoapi/NavierStokes/index:NavierStokes.NavierStokes.className}}\pysigline{\sphinxbfcode{\sphinxupquote{className}}}
\sphinxAtStartPar
name of class for acessing it’s dictionaries in the workspace
\begin{quote}\begin{description}
\item[{Type}] \leavevmode
\sphinxAtStartPar
str

\end{description}\end{quote}

\end{fulllineitems}

\index{BCmodel (NavierStokes.NavierStokes attribute)@\spxentry{BCmodel}\spxextra{NavierStokes.NavierStokes attribute}}

\begin{fulllineitems}
\phantomsection\label{\detokenize{autoapi/NavierStokes/index:NavierStokes.NavierStokes.BCmodel}}\pysigline{\sphinxbfcode{\sphinxupquote{BCmodel}}}
\sphinxAtStartPar
boundary condition model, instance of BoundaryConditioner
\begin{quote}\begin{description}
\item[{Type}] \leavevmode
\sphinxAtStartPar
{\hyperref[\detokenize{autoapi/BoundaryConditioner/index:BoundaryConditioner.BoundaryConditioner}]{\sphinxcrossref{BoundaryConditioner}}}

\end{description}\end{quote}

\end{fulllineitems}

\index{padding (NavierStokes.NavierStokes attribute)@\spxentry{padding}\spxextra{NavierStokes.NavierStokes attribute}}

\begin{fulllineitems}
\phantomsection\label{\detokenize{autoapi/NavierStokes/index:NavierStokes.NavierStokes.padding}}\pysigline{\sphinxbfcode{\sphinxupquote{padding}}}
\sphinxAtStartPar
outter padding for boundary condition implementation
\begin{quote}\begin{description}
\item[{Type}] \leavevmode
\sphinxAtStartPar
int

\end{description}\end{quote}

\end{fulllineitems}

\index{params (NavierStokes.NavierStokes attribute)@\spxentry{params}\spxextra{NavierStokes.NavierStokes attribute}}

\begin{fulllineitems}
\phantomsection\label{\detokenize{autoapi/NavierStokes/index:NavierStokes.NavierStokes.params}}\pysigline{\sphinxbfcode{\sphinxupquote{params}}}
\sphinxAtStartPar
physics parameters from the input
\begin{quote}\begin{description}
\item[{Type}] \leavevmode
\sphinxAtStartPar
dictionary

\end{description}\end{quote}

\end{fulllineitems}

\index{dimensions (NavierStokes.NavierStokes attribute)@\spxentry{dimensions}\spxextra{NavierStokes.NavierStokes attribute}}

\begin{fulllineitems}
\phantomsection\label{\detokenize{autoapi/NavierStokes/index:NavierStokes.NavierStokes.dimensions}}\pysigline{\sphinxbfcode{\sphinxupquote{dimensions}}}
\sphinxAtStartPar
number of states (4)
\begin{quote}\begin{description}
\item[{Type}] \leavevmode
\sphinxAtStartPar
int

\end{description}\end{quote}

\end{fulllineitems}

\index{cfl\_fine (NavierStokes.NavierStokes attribute)@\spxentry{cfl\_fine}\spxextra{NavierStokes.NavierStokes attribute}}

\begin{fulllineitems}
\phantomsection\label{\detokenize{autoapi/NavierStokes/index:NavierStokes.NavierStokes.cfl_fine}}\pysigline{\sphinxbfcode{\sphinxupquote{cfl\_fine}}}
\sphinxAtStartPar
courant number on fine mesh
\begin{quote}\begin{description}
\item[{Type}] \leavevmode
\sphinxAtStartPar
np.ndarray

\end{description}\end{quote}

\end{fulllineitems}

\index{cfl\_coarse (NavierStokes.NavierStokes attribute)@\spxentry{cfl\_coarse}\spxextra{NavierStokes.NavierStokes attribute}}

\begin{fulllineitems}
\phantomsection\label{\detokenize{autoapi/NavierStokes/index:NavierStokes.NavierStokes.cfl_coarse}}\pysigline{\sphinxbfcode{\sphinxupquote{cfl\_coarse}}}
\sphinxAtStartPar
courant number on coarse mesh
\begin{quote}\begin{description}
\item[{Type}] \leavevmode
\sphinxAtStartPar
np.ndarray

\end{description}\end{quote}

\end{fulllineitems}

\index{cfl\_lim (NavierStokes.NavierStokes attribute)@\spxentry{cfl\_lim}\spxextra{NavierStokes.NavierStokes attribute}}

\begin{fulllineitems}
\phantomsection\label{\detokenize{autoapi/NavierStokes/index:NavierStokes.NavierStokes.cfl_lim}}\pysigline{\sphinxbfcode{\sphinxupquote{cfl\_lim}}}
\sphinxAtStartPar
upper limit on courant number
\begin{quote}\begin{description}
\item[{Type}] \leavevmode
\sphinxAtStartPar
float

\end{description}\end{quote}

\end{fulllineitems}

\index{cfl (NavierStokes.NavierStokes attribute)@\spxentry{cfl}\spxextra{NavierStokes.NavierStokes attribute}}

\begin{fulllineitems}
\phantomsection\label{\detokenize{autoapi/NavierStokes/index:NavierStokes.NavierStokes.cfl}}\pysigline{\sphinxbfcode{\sphinxupquote{cfl}}}
\sphinxAtStartPar
minimum cfl between fine and coarse grids
\begin{quote}\begin{description}
\item[{Type}] \leavevmode
\sphinxAtStartPar
float

\end{description}\end{quote}

\end{fulllineitems}


\begin{sphinxadmonition}{note}{Note:}
\sphinxAtStartPar
See report for more details on the physics
\end{sphinxadmonition}
\index{init\_state() (NavierStokes.NavierStokes method)@\spxentry{init\_state()}\spxextra{NavierStokes.NavierStokes method}}

\begin{fulllineitems}
\phantomsection\label{\detokenize{autoapi/NavierStokes/index:NavierStokes.NavierStokes.init_state}}\pysiglinewithargsret{\sphinxbfcode{\sphinxupquote{init\_state}}}{\emph{\DUrole{n}{self}}, \emph{\DUrole{n}{workspace}}, \emph{\DUrole{n}{state}}}{}
\sphinxAtStartPar
Finds max number of columns in a row in the input file.
\begin{quote}\begin{description}
\item[{Parameters}] \leavevmode\begin{itemize}
\item {} 
\sphinxAtStartPar
\sphinxstyleliteralstrong{\sphinxupquote{workspace}} ({\hyperref[\detokenize{autoapi/Workspace/index:Workspace.Workspace}]{\sphinxcrossref{\sphinxstyleliteralemphasis{\sphinxupquote{Workspace}}}}}) \textendash{} containins relevant fields to initialize

\item {} 
\sphinxAtStartPar
\sphinxstyleliteralstrong{\sphinxupquote{state}} ({\hyperref[\detokenize{autoapi/Field/index:Field.Field}]{\sphinxcrossref{\sphinxstyleliteralemphasis{\sphinxupquote{Field}}}}}) \textendash{} has the state variables

\end{itemize}

\end{description}\end{quote}

\end{fulllineitems}

\index{get\_flux() (NavierStokes.NavierStokes method)@\spxentry{get\_flux()}\spxextra{NavierStokes.NavierStokes method}}

\begin{fulllineitems}
\phantomsection\label{\detokenize{autoapi/NavierStokes/index:NavierStokes.NavierStokes.get_flux}}\pysiglinewithargsret{\sphinxbfcode{\sphinxupquote{get\_flux}}}{\emph{\DUrole{n}{self}}, \emph{\DUrole{n}{workspace}}, \emph{\DUrole{n}{state}}, \emph{\DUrole{n}{output}}, \emph{\DUrole{n}{update\_factor}\DUrole{o}{=}\DUrole{default_value}{1}}}{}
\sphinxAtStartPar
Calculates the spatial flux given the current state.
\begin{quote}\begin{description}
\item[{Parameters}] \leavevmode\begin{itemize}
\item {} 
\sphinxAtStartPar
\sphinxstyleliteralstrong{\sphinxupquote{workspace}} ({\hyperref[\detokenize{autoapi/Workspace/index:Workspace.Workspace}]{\sphinxcrossref{\sphinxstyleliteralemphasis{\sphinxupquote{Workspace}}}}}) \textendash{} contains the relevant fields

\item {} 
\sphinxAtStartPar
\sphinxstyleliteralstrong{\sphinxupquote{state}} ({\hyperref[\detokenize{autoapi/Field/index:Field.Field}]{\sphinxcrossref{\sphinxstyleliteralemphasis{\sphinxupquote{Field}}}}}) \textendash{} the current state

\item {} 
\sphinxAtStartPar
\sphinxstyleliteralstrong{\sphinxupquote{output}} ({\hyperref[\detokenize{autoapi/Field/index:Field.Field}]{\sphinxcrossref{\sphinxstyleliteralemphasis{\sphinxupquote{Field}}}}}) \textendash{} where the flux values will be stored

\end{itemize}

\end{description}\end{quote}

\end{fulllineitems}

\index{get\_safe\_timestep() (NavierStokes.NavierStokes method)@\spxentry{get\_safe\_timestep()}\spxextra{NavierStokes.NavierStokes method}}

\begin{fulllineitems}
\phantomsection\label{\detokenize{autoapi/NavierStokes/index:NavierStokes.NavierStokes.get_safe_timestep}}\pysiglinewithargsret{\sphinxbfcode{\sphinxupquote{get\_safe\_timestep}}}{\emph{\DUrole{n}{self}}, \emph{\DUrole{n}{workspace}}, \emph{\DUrole{n}{state}}, \emph{\DUrole{n}{timestep}}}{}
\sphinxAtStartPar
Returns the local timestep such that stability is maintained.
\begin{quote}\begin{description}
\item[{Parameters}] \leavevmode\begin{itemize}
\item {} 
\sphinxAtStartPar
\sphinxstyleliteralstrong{\sphinxupquote{workspace}} ({\hyperref[\detokenize{autoapi/Workspace/index:Workspace.Workspace}]{\sphinxcrossref{\sphinxstyleliteralemphasis{\sphinxupquote{Workspace}}}}}) \textendash{} contains the relevant fields

\item {} 
\sphinxAtStartPar
\sphinxstyleliteralstrong{\sphinxupquote{state}} ({\hyperref[\detokenize{autoapi/Field/index:Field.Field}]{\sphinxcrossref{\sphinxstyleliteralemphasis{\sphinxupquote{Field}}}}}) \textendash{} the current state

\item {} 
\sphinxAtStartPar
\sphinxstyleliteralstrong{\sphinxupquote{timestep}} ({\hyperref[\detokenize{autoapi/Field/index:Field.Field}]{\sphinxcrossref{\sphinxstyleliteralemphasis{\sphinxupquote{Field}}}}}) \textendash{} where the time steps will be stored

\end{itemize}

\end{description}\end{quote}

\end{fulllineitems}

\index{update\_physics() (NavierStokes.NavierStokes method)@\spxentry{update\_physics()}\spxextra{NavierStokes.NavierStokes method}}

\begin{fulllineitems}
\phantomsection\label{\detokenize{autoapi/NavierStokes/index:NavierStokes.NavierStokes.update_physics}}\pysiglinewithargsret{\sphinxbfcode{\sphinxupquote{update\_physics}}}{\emph{\DUrole{n}{self}}, \emph{\DUrole{n}{workspace}}, \emph{\DUrole{n}{state}}}{}
\sphinxAtStartPar
Updates physical properties of system based on state
\begin{quote}\begin{description}
\item[{Parameters}] \leavevmode\begin{itemize}
\item {} 
\sphinxAtStartPar
\sphinxstyleliteralstrong{\sphinxupquote{workspace}} ({\hyperref[\detokenize{autoapi/Workspace/index:Workspace.Workspace}]{\sphinxcrossref{\sphinxstyleliteralemphasis{\sphinxupquote{Workspace}}}}}) \textendash{} contains the relevant fields

\item {} 
\sphinxAtStartPar
\sphinxstyleliteralstrong{\sphinxupquote{state}} ({\hyperref[\detokenize{autoapi/Field/index:Field.Field}]{\sphinxcrossref{\sphinxstyleliteralemphasis{\sphinxupquote{Field}}}}}) \textendash{} the current state

\end{itemize}

\end{description}\end{quote}

\end{fulllineitems}

\index{update\_stability() (NavierStokes.NavierStokes method)@\spxentry{update\_stability()}\spxextra{NavierStokes.NavierStokes method}}

\begin{fulllineitems}
\phantomsection\label{\detokenize{autoapi/NavierStokes/index:NavierStokes.NavierStokes.update_stability}}\pysiglinewithargsret{\sphinxbfcode{\sphinxupquote{update\_stability}}}{\emph{\DUrole{n}{self}}, \emph{\DUrole{n}{workspace}}, \emph{\DUrole{n}{state}}}{}
\sphinxAtStartPar
Updates the stability parameters given the current state.
\begin{quote}\begin{description}
\item[{Parameters}] \leavevmode\begin{itemize}
\item {} 
\sphinxAtStartPar
\sphinxstyleliteralstrong{\sphinxupquote{workspace}} ({\hyperref[\detokenize{autoapi/Workspace/index:Workspace.Workspace}]{\sphinxcrossref{\sphinxstyleliteralemphasis{\sphinxupquote{Workspace}}}}}) \textendash{} contains the relevant fields

\item {} 
\sphinxAtStartPar
\sphinxstyleliteralstrong{\sphinxupquote{state}} ({\hyperref[\detokenize{autoapi/Field/index:Field.Field}]{\sphinxcrossref{\sphinxstyleliteralemphasis{\sphinxupquote{Field}}}}}) \textendash{} the current state

\end{itemize}

\end{description}\end{quote}

\end{fulllineitems}

\index{get\_cfl() (NavierStokes.NavierStokes method)@\spxentry{get\_cfl()}\spxextra{NavierStokes.NavierStokes method}}

\begin{fulllineitems}
\phantomsection\label{\detokenize{autoapi/NavierStokes/index:NavierStokes.NavierStokes.get_cfl}}\pysiglinewithargsret{\sphinxbfcode{\sphinxupquote{get\_cfl}}}{\emph{\DUrole{n}{self}}, \emph{\DUrole{n}{workspace}}}{}
\end{fulllineitems}

\index{transfer\_down() (NavierStokes.NavierStokes method)@\spxentry{transfer\_down()}\spxextra{NavierStokes.NavierStokes method}}

\begin{fulllineitems}
\phantomsection\label{\detokenize{autoapi/NavierStokes/index:NavierStokes.NavierStokes.transfer_down}}\pysiglinewithargsret{\sphinxbfcode{\sphinxupquote{transfer\_down}}}{\emph{\DUrole{n}{self}}, \emph{\DUrole{n}{workspace1}}, \emph{\DUrole{n}{workspace2}}}{}
\sphinxAtStartPar
Move workspace1 on fine mesh to workspace2 on coarse mesh
\begin{quote}\begin{description}
\item[{Parameters}] \leavevmode\begin{itemize}
\item {} 
\sphinxAtStartPar
\sphinxstyleliteralstrong{\sphinxupquote{workspace1}} ({\hyperref[\detokenize{autoapi/Workspace/index:Workspace.Workspace}]{\sphinxcrossref{\sphinxstyleliteralemphasis{\sphinxupquote{Workspace}}}}}) \textendash{} The Workspace object for the finer level

\item {} 
\sphinxAtStartPar
\sphinxstyleliteralstrong{\sphinxupquote{workspace2}} ({\hyperref[\detokenize{autoapi/Workspace/index:Workspace.Workspace}]{\sphinxcrossref{\sphinxstyleliteralemphasis{\sphinxupquote{Workspace}}}}}) \textendash{} The Workspace object for the coarser level

\end{itemize}

\end{description}\end{quote}

\end{fulllineitems}

\index{dim() (NavierStokes.NavierStokes method)@\spxentry{dim()}\spxextra{NavierStokes.NavierStokes method}}

\begin{fulllineitems}
\phantomsection\label{\detokenize{autoapi/NavierStokes/index:NavierStokes.NavierStokes.dim}}\pysiglinewithargsret{\sphinxbfcode{\sphinxupquote{dim}}}{\emph{\DUrole{n}{self}}}{}
\end{fulllineitems}

\index{\_\_copy\_in() (NavierStokes.NavierStokes method)@\spxentry{\_\_copy\_in()}\spxextra{NavierStokes.NavierStokes method}}

\begin{fulllineitems}
\phantomsection\label{\detokenize{autoapi/NavierStokes/index:NavierStokes.NavierStokes.__copy_in}}\pysiglinewithargsret{\sphinxbfcode{\sphinxupquote{\_\_copy\_in}}}{\emph{\DUrole{n}{self}}, \emph{\DUrole{n}{field}}, \emph{\DUrole{n}{paddedField}}}{}
\end{fulllineitems}

\index{\_\_copy\_out() (NavierStokes.NavierStokes method)@\spxentry{\_\_copy\_out()}\spxextra{NavierStokes.NavierStokes method}}

\begin{fulllineitems}
\phantomsection\label{\detokenize{autoapi/NavierStokes/index:NavierStokes.NavierStokes.__copy_out}}\pysiglinewithargsret{\sphinxbfcode{\sphinxupquote{\_\_copy\_out}}}{\emph{\DUrole{n}{self}}, \emph{\DUrole{n}{paddedField}}, \emph{\DUrole{n}{field}}}{}
\end{fulllineitems}

\index{\_\_check\_vars() (NavierStokes.NavierStokes method)@\spxentry{\_\_check\_vars()}\spxextra{NavierStokes.NavierStokes method}}

\begin{fulllineitems}
\phantomsection\label{\detokenize{autoapi/NavierStokes/index:NavierStokes.NavierStokes.__check_vars}}\pysiglinewithargsret{\sphinxbfcode{\sphinxupquote{\_\_check\_vars}}}{\emph{\DUrole{n}{self}}, \emph{\DUrole{n}{workspace}}}{}
\end{fulllineitems}

\index{\_\_init\_vars() (NavierStokes.NavierStokes method)@\spxentry{\_\_init\_vars()}\spxextra{NavierStokes.NavierStokes method}}

\begin{fulllineitems}
\phantomsection\label{\detokenize{autoapi/NavierStokes/index:NavierStokes.NavierStokes.__init_vars}}\pysiglinewithargsret{\sphinxbfcode{\sphinxupquote{\_\_init\_vars}}}{\emph{\DUrole{n}{self}}, \emph{\DUrole{n}{workspace}}}{}
\end{fulllineitems}

\index{\_\_init\_state() (NavierStokes.NavierStokes method)@\spxentry{\_\_init\_state()}\spxextra{NavierStokes.NavierStokes method}}

\begin{fulllineitems}
\phantomsection\label{\detokenize{autoapi/NavierStokes/index:NavierStokes.NavierStokes.__init_state}}\pysiglinewithargsret{\sphinxbfcode{\sphinxupquote{\_\_init\_state}}}{\emph{\DUrole{n}{self}}, \emph{\DUrole{n}{workspace}}, \emph{\DUrole{n}{state}}}{}
\end{fulllineitems}

\index{\_\_update\_pressure() (NavierStokes.NavierStokes method)@\spxentry{\_\_update\_pressure()}\spxextra{NavierStokes.NavierStokes method}}

\begin{fulllineitems}
\phantomsection\label{\detokenize{autoapi/NavierStokes/index:NavierStokes.NavierStokes.__update_pressure}}\pysiglinewithargsret{\sphinxbfcode{\sphinxupquote{\_\_update\_pressure}}}{\emph{\DUrole{n}{self}}, \emph{\DUrole{n}{workspace}}, \emph{\DUrole{n}{state}}}{}
\end{fulllineitems}


\end{fulllineitems}



\section{\sphinxstyleliteralintitle{\sphinxupquote{NS\_Airfoil}}}
\label{\detokenize{autoapi/NS_Airfoil/index:module-NS_Airfoil}}\label{\detokenize{autoapi/NS_Airfoil/index:ns-airfoil}}\label{\detokenize{autoapi/NS_Airfoil/index::doc}}\index{module@\spxentry{module}!NS\_Airfoil@\spxentry{NS\_Airfoil}}\index{NS\_Airfoil@\spxentry{NS\_Airfoil}!module@\spxentry{module}}

\subsection{Module Contents}
\label{\detokenize{autoapi/NS_Airfoil/index:module-contents}}

\subsubsection{Classes}
\label{\detokenize{autoapi/NS_Airfoil/index:classes}}

\begin{savenotes}\sphinxatlongtablestart\begin{longtable}[c]{\X{1}{2}\X{1}{2}}
\hline

\endfirsthead

\multicolumn{2}{c}%
{\makebox[0pt]{\sphinxtablecontinued{\tablename\ \thetable{} \textendash{} continued from previous page}}}\\
\hline

\endhead

\hline
\multicolumn{2}{r}{\makebox[0pt][r]{\sphinxtablecontinued{continues on next page}}}\\
\endfoot

\endlastfoot

\sphinxAtStartPar
{\hyperref[\detokenize{autoapi/NS_Airfoil/index:NS_Airfoil.NS_Airfoil}]{\sphinxcrossref{\sphinxcode{\sphinxupquote{NS\_Airfoil}}}}}
&
\sphinxAtStartPar
Implements boundary conditions for Navier Stokes based model of flow over an airfoil.
\\
\hline
\end{longtable}\sphinxatlongtableend\end{savenotes}
\index{NS\_Airfoil (class in NS\_Airfoil)@\spxentry{NS\_Airfoil}\spxextra{class in NS\_Airfoil}}

\begin{fulllineitems}
\phantomsection\label{\detokenize{autoapi/NS_Airfoil/index:NS_Airfoil.NS_Airfoil}}\pysiglinewithargsret{\sphinxbfcode{\sphinxupquote{class }}\sphinxcode{\sphinxupquote{NS\_Airfoil.}}\sphinxbfcode{\sphinxupquote{NS\_Airfoil}}}{\emph{\DUrole{n}{input}}}{}
\sphinxAtStartPar
Bases: \sphinxcode{\sphinxupquote{bin.BoundaryConditioner.BoundaryConditioner}}

\sphinxAtStartPar
Implements boundary conditions for Navier Stokes based model of flow over an airfoil.
A halo is formed around the mesh containing ghost nodes. Two types of boundary conditions
are implemented: wall boundaries and far field boundaries
\begin{quote}
\begin{description}
\item[{Constructor:}] \leavevmode\begin{description}
\item[{Args:}] \leavevmode
\sphinxAtStartPar
input (list): input dictionary

\item[{Returns:}] \leavevmode
\sphinxAtStartPar
A NS\_Airfoil object

\end{description}

\end{description}
\end{quote}
\index{class\_name (NS\_Airfoil.NS\_Airfoil attribute)@\spxentry{class\_name}\spxextra{NS\_Airfoil.NS\_Airfoil attribute}}

\begin{fulllineitems}
\phantomsection\label{\detokenize{autoapi/NS_Airfoil/index:NS_Airfoil.NS_Airfoil.class_name}}\pysigline{\sphinxbfcode{\sphinxupquote{class\_name}}}
\sphinxAtStartPar
name of class for accessing Fields in the workspace
\begin{quote}\begin{description}
\item[{Type}] \leavevmode
\sphinxAtStartPar
str

\end{description}\end{quote}

\end{fulllineitems}

\index{padding (NS\_Airfoil.NS\_Airfoil attribute)@\spxentry{padding}\spxextra{NS\_Airfoil.NS\_Airfoil attribute}}

\begin{fulllineitems}
\phantomsection\label{\detokenize{autoapi/NS_Airfoil/index:NS_Airfoil.NS_Airfoil.padding}}\pysigline{\sphinxbfcode{\sphinxupquote{padding}}}
\sphinxAtStartPar
number of ghost nodes on boundary
\begin{quote}\begin{description}
\item[{Type}] \leavevmode
\sphinxAtStartPar
int

\end{description}\end{quote}

\end{fulllineitems}

\index{local\_timestepping (NS\_Airfoil.NS\_Airfoil attribute)@\spxentry{local\_timestepping}\spxextra{NS\_Airfoil.NS\_Airfoil attribute}}

\begin{fulllineitems}
\phantomsection\label{\detokenize{autoapi/NS_Airfoil/index:NS_Airfoil.NS_Airfoil.local_timestepping}}\pysigline{\sphinxbfcode{\sphinxupquote{local\_timestepping}}}
\sphinxAtStartPar
parameter for use in calculating stable time step
\begin{quote}\begin{description}
\item[{Type}] \leavevmode
\sphinxAtStartPar
float

\end{description}\end{quote}

\end{fulllineitems}

\index{bc (NS\_Airfoil.NS\_Airfoil attribute)@\spxentry{bc}\spxextra{NS\_Airfoil.NS\_Airfoil attribute}}

\begin{fulllineitems}
\phantomsection\label{\detokenize{autoapi/NS_Airfoil/index:NS_Airfoil.NS_Airfoil.bc}}\pysigline{\sphinxbfcode{\sphinxupquote{bc}}}
\sphinxAtStartPar
some parameter

\end{fulllineitems}

\index{update\_physics() (NS\_Airfoil.NS\_Airfoil method)@\spxentry{update\_physics()}\spxextra{NS\_Airfoil.NS\_Airfoil method}}

\begin{fulllineitems}
\phantomsection\label{\detokenize{autoapi/NS_Airfoil/index:NS_Airfoil.NS_Airfoil.update_physics}}\pysiglinewithargsret{\sphinxbfcode{\sphinxupquote{update\_physics}}}{\emph{\DUrole{n}{self}}, \emph{\DUrole{n}{model}}, \emph{\DUrole{n}{workspace}}, \emph{\DUrole{n}{state}}}{}
\sphinxAtStartPar
updates the turbulent viscocity for calculation of boundary conditions
\begin{quote}\begin{description}
\item[{Parameters}] \leavevmode\begin{itemize}
\item {} 
\sphinxAtStartPar
\sphinxstyleliteralstrong{\sphinxupquote{model}} ({\hyperref[\detokenize{autoapi/NavierStokes/index:NavierStokes.NavierStokes}]{\sphinxcrossref{\sphinxstyleliteralemphasis{\sphinxupquote{NavierStokes}}}}}) \textendash{} physics model

\item {} 
\sphinxAtStartPar
\sphinxstyleliteralstrong{\sphinxupquote{workspace}} ({\hyperref[\detokenize{autoapi/Workspace/index:Workspace.Workspace}]{\sphinxcrossref{\sphinxstyleliteralemphasis{\sphinxupquote{Workspace}}}}}) \textendash{} contains the relevant fields

\item {} 
\sphinxAtStartPar
\sphinxstyleliteralstrong{\sphinxupquote{state}} ({\hyperref[\detokenize{autoapi/Field/index:Field.Field}]{\sphinxcrossref{\sphinxstyleliteralemphasis{\sphinxupquote{Field}}}}}) \textendash{} current state of the system (density, momentum, energy)

\end{itemize}

\end{description}\end{quote}

\end{fulllineitems}

\index{update\_stability() (NS\_Airfoil.NS\_Airfoil method)@\spxentry{update\_stability()}\spxextra{NS\_Airfoil.NS\_Airfoil method}}

\begin{fulllineitems}
\phantomsection\label{\detokenize{autoapi/NS_Airfoil/index:NS_Airfoil.NS_Airfoil.update_stability}}\pysiglinewithargsret{\sphinxbfcode{\sphinxupquote{update\_stability}}}{\emph{\DUrole{n}{self}}, \emph{\DUrole{n}{model}}, \emph{\DUrole{n}{workspace}}, \emph{\DUrole{n}{state}}}{}
\sphinxAtStartPar
updates stability parameters for time step calculations
\begin{quote}\begin{description}
\item[{Parameters}] \leavevmode\begin{itemize}
\item {} 
\sphinxAtStartPar
\sphinxstyleliteralstrong{\sphinxupquote{model}} ({\hyperref[\detokenize{autoapi/NavierStokes/index:NavierStokes.NavierStokes}]{\sphinxcrossref{\sphinxstyleliteralemphasis{\sphinxupquote{NavierStokes}}}}}) \textendash{} physics model

\item {} 
\sphinxAtStartPar
\sphinxstyleliteralstrong{\sphinxupquote{workspace}} ({\hyperref[\detokenize{autoapi/Workspace/index:Workspace.Workspace}]{\sphinxcrossref{\sphinxstyleliteralemphasis{\sphinxupquote{Workspace}}}}}) \textendash{} contains the relevant fields

\item {} 
\sphinxAtStartPar
\sphinxstyleliteralstrong{\sphinxupquote{state}} ({\hyperref[\detokenize{autoapi/Field/index:Field.Field}]{\sphinxcrossref{\sphinxstyleliteralemphasis{\sphinxupquote{Field}}}}}) \textendash{} current state of the system (density, momentum, energy)

\end{itemize}

\end{description}\end{quote}

\end{fulllineitems}

\index{bc\_far() (NS\_Airfoil.NS\_Airfoil method)@\spxentry{bc\_far()}\spxextra{NS\_Airfoil.NS\_Airfoil method}}

\begin{fulllineitems}
\phantomsection\label{\detokenize{autoapi/NS_Airfoil/index:NS_Airfoil.NS_Airfoil.bc_far}}\pysiglinewithargsret{\sphinxbfcode{\sphinxupquote{bc\_far}}}{\emph{\DUrole{n}{self}}, \emph{\DUrole{n}{model}}, \emph{\DUrole{n}{workspace}}, \emph{\DUrole{n}{state}}}{}
\sphinxAtStartPar
apply boundary condition in the far field
\begin{quote}\begin{description}
\item[{Parameters}] \leavevmode\begin{itemize}
\item {} 
\sphinxAtStartPar
\sphinxstyleliteralstrong{\sphinxupquote{model}} ({\hyperref[\detokenize{autoapi/NavierStokes/index:NavierStokes.NavierStokes}]{\sphinxcrossref{\sphinxstyleliteralemphasis{\sphinxupquote{NavierStokes}}}}}) \textendash{} physics model

\item {} 
\sphinxAtStartPar
\sphinxstyleliteralstrong{\sphinxupquote{workspace}} ({\hyperref[\detokenize{autoapi/Workspace/index:Workspace.Workspace}]{\sphinxcrossref{\sphinxstyleliteralemphasis{\sphinxupquote{Workspace}}}}}) \textendash{} contains the relevant fields

\item {} 
\sphinxAtStartPar
\sphinxstyleliteralstrong{\sphinxupquote{state}} ({\hyperref[\detokenize{autoapi/Field/index:Field.Field}]{\sphinxcrossref{\sphinxstyleliteralemphasis{\sphinxupquote{Field}}}}}) \textendash{} current state of the system (density, momentum, energy)

\end{itemize}

\end{description}\end{quote}

\end{fulllineitems}

\index{bc\_wall() (NS\_Airfoil.NS\_Airfoil method)@\spxentry{bc\_wall()}\spxextra{NS\_Airfoil.NS\_Airfoil method}}

\begin{fulllineitems}
\phantomsection\label{\detokenize{autoapi/NS_Airfoil/index:NS_Airfoil.NS_Airfoil.bc_wall}}\pysiglinewithargsret{\sphinxbfcode{\sphinxupquote{bc\_wall}}}{\emph{\DUrole{n}{self}}, \emph{\DUrole{n}{model}}, \emph{\DUrole{n}{workspace}}, \emph{\DUrole{n}{state}}}{}
\sphinxAtStartPar
apply boundary condition along the wall
\begin{quote}\begin{description}
\item[{Parameters}] \leavevmode\begin{itemize}
\item {} 
\sphinxAtStartPar
\sphinxstyleliteralstrong{\sphinxupquote{model}} ({\hyperref[\detokenize{autoapi/NavierStokes/index:NavierStokes.NavierStokes}]{\sphinxcrossref{\sphinxstyleliteralemphasis{\sphinxupquote{NavierStokes}}}}}) \textendash{} physics model

\item {} 
\sphinxAtStartPar
\sphinxstyleliteralstrong{\sphinxupquote{workspace}} ({\hyperref[\detokenize{autoapi/Workspace/index:Workspace.Workspace}]{\sphinxcrossref{\sphinxstyleliteralemphasis{\sphinxupquote{Workspace}}}}}) \textendash{} contains the relevant fields

\item {} 
\sphinxAtStartPar
\sphinxstyleliteralstrong{\sphinxupquote{state}} ({\hyperref[\detokenize{autoapi/Field/index:Field.Field}]{\sphinxcrossref{\sphinxstyleliteralemphasis{\sphinxupquote{Field}}}}}) \textendash{} current state of the system (density, momentum, energy)

\end{itemize}

\end{description}\end{quote}

\end{fulllineitems}

\index{halo() (NS\_Airfoil.NS\_Airfoil method)@\spxentry{halo()}\spxextra{NS\_Airfoil.NS\_Airfoil method}}

\begin{fulllineitems}
\phantomsection\label{\detokenize{autoapi/NS_Airfoil/index:NS_Airfoil.NS_Airfoil.halo}}\pysiglinewithargsret{\sphinxbfcode{\sphinxupquote{halo}}}{\emph{\DUrole{n}{self}}, \emph{\DUrole{n}{model}}, \emph{\DUrole{n}{workspace}}, \emph{\DUrole{n}{state}}}{}
\sphinxAtStartPar
set the values in the ghost cells
\begin{quote}\begin{description}
\item[{Parameters}] \leavevmode\begin{itemize}
\item {} 
\sphinxAtStartPar
\sphinxstyleliteralstrong{\sphinxupquote{model}} ({\hyperref[\detokenize{autoapi/NavierStokes/index:NavierStokes.NavierStokes}]{\sphinxcrossref{\sphinxstyleliteralemphasis{\sphinxupquote{NavierStokes}}}}}) \textendash{} physics model

\item {} 
\sphinxAtStartPar
\sphinxstyleliteralstrong{\sphinxupquote{workspace}} ({\hyperref[\detokenize{autoapi/Workspace/index:Workspace.Workspace}]{\sphinxcrossref{\sphinxstyleliteralemphasis{\sphinxupquote{Workspace}}}}}) \textendash{} contains the relevant fields

\item {} 
\sphinxAtStartPar
\sphinxstyleliteralstrong{\sphinxupquote{state}} ({\hyperref[\detokenize{autoapi/Field/index:Field.Field}]{\sphinxcrossref{\sphinxstyleliteralemphasis{\sphinxupquote{Field}}}}}) \textendash{} current state of the system (density, momentum, energy)

\end{itemize}

\end{description}\end{quote}

\end{fulllineitems}

\index{bc\_all() (NS\_Airfoil.NS\_Airfoil method)@\spxentry{bc\_all()}\spxextra{NS\_Airfoil.NS\_Airfoil method}}

\begin{fulllineitems}
\phantomsection\label{\detokenize{autoapi/NS_Airfoil/index:NS_Airfoil.NS_Airfoil.bc_all}}\pysiglinewithargsret{\sphinxbfcode{\sphinxupquote{bc\_all}}}{\emph{\DUrole{n}{self}}, \emph{\DUrole{n}{model}}, \emph{\DUrole{n}{workspace}}, \emph{\DUrole{n}{state}}}{}
\sphinxAtStartPar
do wall boundaries, far field and set halo values at once
\begin{quote}\begin{description}
\item[{Parameters}] \leavevmode\begin{itemize}
\item {} 
\sphinxAtStartPar
\sphinxstyleliteralstrong{\sphinxupquote{model}} ({\hyperref[\detokenize{autoapi/NavierStokes/index:NavierStokes.NavierStokes}]{\sphinxcrossref{\sphinxstyleliteralemphasis{\sphinxupquote{NavierStokes}}}}}) \textendash{} physics model

\item {} 
\sphinxAtStartPar
\sphinxstyleliteralstrong{\sphinxupquote{workspace}} ({\hyperref[\detokenize{autoapi/Workspace/index:Workspace.Workspace}]{\sphinxcrossref{\sphinxstyleliteralemphasis{\sphinxupquote{Workspace}}}}}) \textendash{} contains the relevant fields

\item {} 
\sphinxAtStartPar
\sphinxstyleliteralstrong{\sphinxupquote{state}} ({\hyperref[\detokenize{autoapi/Field/index:Field.Field}]{\sphinxcrossref{\sphinxstyleliteralemphasis{\sphinxupquote{Field}}}}}) \textendash{} current state of the system (density, momentum, energy)

\end{itemize}

\end{description}\end{quote}

\end{fulllineitems}

\index{transfer\_down() (NS\_Airfoil.NS\_Airfoil method)@\spxentry{transfer\_down()}\spxextra{NS\_Airfoil.NS\_Airfoil method}}

\begin{fulllineitems}
\phantomsection\label{\detokenize{autoapi/NS_Airfoil/index:NS_Airfoil.NS_Airfoil.transfer_down}}\pysiglinewithargsret{\sphinxbfcode{\sphinxupquote{transfer\_down}}}{\emph{\DUrole{n}{self}}, \emph{\DUrole{n}{model}}, \emph{\DUrole{n}{workspace1}}, \emph{\DUrole{n}{workspace2}}}{}
\sphinxAtStartPar
transfer workspace to coarser mesh
\begin{quote}\begin{description}
\item[{Parameters}] \leavevmode\begin{itemize}
\item {} 
\sphinxAtStartPar
\sphinxstyleliteralstrong{\sphinxupquote{workspace1}} ({\hyperref[\detokenize{autoapi/Workspace/index:Workspace.Workspace}]{\sphinxcrossref{\sphinxstyleliteralemphasis{\sphinxupquote{Workspace}}}}}) \textendash{} on fine mesh

\item {} 
\sphinxAtStartPar
\sphinxstyleliteralstrong{\sphinxupquote{workspace2}} ({\hyperref[\detokenize{autoapi/Workspace/index:Workspace.Workspace}]{\sphinxcrossref{\sphinxstyleliteralemphasis{\sphinxupquote{Workspace}}}}}) \textendash{} on coarse mesh

\end{itemize}

\end{description}\end{quote}

\end{fulllineitems}

\index{get\_pori() (NS\_Airfoil.NS\_Airfoil method)@\spxentry{get\_pori()}\spxextra{NS\_Airfoil.NS\_Airfoil method}}

\begin{fulllineitems}
\phantomsection\label{\detokenize{autoapi/NS_Airfoil/index:NS_Airfoil.NS_Airfoil.get_pori}}\pysiglinewithargsret{\sphinxbfcode{\sphinxupquote{get\_pori}}}{\emph{\DUrole{n}{self}}, \emph{\DUrole{n}{workspace}}}{}
\sphinxAtStartPar
grab porosity in i direction from the workspace
\begin{quote}\begin{description}
\item[{Parameters}] \leavevmode
\sphinxAtStartPar
\sphinxstyleliteralstrong{\sphinxupquote{workspace}} ({\hyperref[\detokenize{autoapi/Workspace/index:Workspace.Workspace}]{\sphinxcrossref{\sphinxstyleliteralemphasis{\sphinxupquote{Workspace}}}}}) \textendash{} has pori

\item[{Returns}] \leavevmode
\sphinxAtStartPar
porosity in i direction

\item[{Return type}] \leavevmode
\sphinxAtStartPar
pori ({\hyperref[\detokenize{autoapi/Field/index:Field.Field}]{\sphinxcrossref{Field}}})

\end{description}\end{quote}

\end{fulllineitems}

\index{get\_porj() (NS\_Airfoil.NS\_Airfoil method)@\spxentry{get\_porj()}\spxextra{NS\_Airfoil.NS\_Airfoil method}}

\begin{fulllineitems}
\phantomsection\label{\detokenize{autoapi/NS_Airfoil/index:NS_Airfoil.NS_Airfoil.get_porj}}\pysiglinewithargsret{\sphinxbfcode{\sphinxupquote{get\_porj}}}{\emph{\DUrole{n}{self}}, \emph{\DUrole{n}{workspace}}}{}
\sphinxAtStartPar
grab porosity in j direction from the workspace
\begin{quote}\begin{description}
\item[{Parameters}] \leavevmode
\sphinxAtStartPar
\sphinxstyleliteralstrong{\sphinxupquote{workspace}} ({\hyperref[\detokenize{autoapi/Workspace/index:Workspace.Workspace}]{\sphinxcrossref{\sphinxstyleliteralemphasis{\sphinxupquote{Workspace}}}}}) \textendash{} has porj

\item[{Returns}] \leavevmode
\sphinxAtStartPar
porosity in j direction

\item[{Return type}] \leavevmode
\sphinxAtStartPar
porj ({\hyperref[\detokenize{autoapi/Field/index:Field.Field}]{\sphinxcrossref{Field}}})

\end{description}\end{quote}

\end{fulllineitems}

\index{halo\_geom() (NS\_Airfoil.NS\_Airfoil method)@\spxentry{halo\_geom()}\spxextra{NS\_Airfoil.NS\_Airfoil method}}

\begin{fulllineitems}
\phantomsection\label{\detokenize{autoapi/NS_Airfoil/index:NS_Airfoil.NS_Airfoil.halo_geom}}\pysiglinewithargsret{\sphinxbfcode{\sphinxupquote{halo\_geom}}}{\emph{\DUrole{n}{self}}, \emph{\DUrole{n}{model}}, \emph{\DUrole{n}{workspace}}}{}
\sphinxAtStartPar
set values in the ghost cells
\begin{quote}\begin{description}
\item[{Parameters}] \leavevmode\begin{itemize}
\item {} 
\sphinxAtStartPar
\sphinxstyleliteralstrong{\sphinxupquote{model}} ({\hyperref[\detokenize{autoapi/NavierStokes/index:NavierStokes.NavierStokes}]{\sphinxcrossref{\sphinxstyleliteralemphasis{\sphinxupquote{NavierStokes}}}}}) \textendash{} physics model

\item {} 
\sphinxAtStartPar
\sphinxstyleliteralstrong{\sphinxupquote{workspace}} ({\hyperref[\detokenize{autoapi/Workspace/index:Workspace.Workspace}]{\sphinxcrossref{\sphinxstyleliteralemphasis{\sphinxupquote{Workspace}}}}}) \textendash{} contains the relevant fields

\end{itemize}

\end{description}\end{quote}

\end{fulllineitems}

\index{\_\_check\_vars() (NS\_Airfoil.NS\_Airfoil method)@\spxentry{\_\_check\_vars()}\spxextra{NS\_Airfoil.NS\_Airfoil method}}

\begin{fulllineitems}
\phantomsection\label{\detokenize{autoapi/NS_Airfoil/index:NS_Airfoil.NS_Airfoil.__check_vars}}\pysiglinewithargsret{\sphinxbfcode{\sphinxupquote{\_\_check\_vars}}}{\emph{\DUrole{n}{self}}, \emph{\DUrole{n}{workspace}}}{}
\end{fulllineitems}

\index{\_\_init\_vars() (NS\_Airfoil.NS\_Airfoil method)@\spxentry{\_\_init\_vars()}\spxextra{NS\_Airfoil.NS\_Airfoil method}}

\begin{fulllineitems}
\phantomsection\label{\detokenize{autoapi/NS_Airfoil/index:NS_Airfoil.NS_Airfoil.__init_vars}}\pysiglinewithargsret{\sphinxbfcode{\sphinxupquote{\_\_init\_vars}}}{\emph{\DUrole{n}{self}}, \emph{\DUrole{n}{workspace}}}{}
\end{fulllineitems}

\index{\_\_set\_porosity() (NS\_Airfoil.NS\_Airfoil method)@\spxentry{\_\_set\_porosity()}\spxextra{NS\_Airfoil.NS\_Airfoil method}}

\begin{fulllineitems}
\phantomsection\label{\detokenize{autoapi/NS_Airfoil/index:NS_Airfoil.NS_Airfoil.__set_porosity}}\pysiglinewithargsret{\sphinxbfcode{\sphinxupquote{\_\_set\_porosity}}}{\emph{\DUrole{n}{self}}, \emph{\DUrole{n}{workspace}}}{}
\end{fulllineitems}


\end{fulllineitems}



\section{\sphinxstyleliteralintitle{\sphinxupquote{Workspace}}}
\label{\detokenize{autoapi/Workspace/index:module-Workspace}}\label{\detokenize{autoapi/Workspace/index:workspace}}\label{\detokenize{autoapi/Workspace/index::doc}}\index{module@\spxentry{module}!Workspace@\spxentry{Workspace}}\index{Workspace@\spxentry{Workspace}!module@\spxentry{module}}
\sphinxAtStartPar
This module is the ABC for workspace.
Libraries/Modules:
\begin{quote}

\sphinxAtStartPar
abc

\sphinxAtStartPar
numpy

\sphinxAtStartPar
Field
\end{quote}


\subsection{Module Contents}
\label{\detokenize{autoapi/Workspace/index:module-contents}}

\subsubsection{Classes}
\label{\detokenize{autoapi/Workspace/index:classes}}

\begin{savenotes}\sphinxatlongtablestart\begin{longtable}[c]{\X{1}{2}\X{1}{2}}
\hline

\endfirsthead

\multicolumn{2}{c}%
{\makebox[0pt]{\sphinxtablecontinued{\tablename\ \thetable{} \textendash{} continued from previous page}}}\\
\hline

\endhead

\hline
\multicolumn{2}{r}{\makebox[0pt][r]{\sphinxtablecontinued{continues on next page}}}\\
\endfoot

\endlastfoot

\sphinxAtStartPar
{\hyperref[\detokenize{autoapi/Workspace/index:Workspace.Workspace}]{\sphinxcrossref{\sphinxcode{\sphinxupquote{Workspace}}}}}
&
\sphinxAtStartPar
Abstract base class for workspaces.
\\
\hline
\end{longtable}\sphinxatlongtableend\end{savenotes}
\index{Workspace (class in Workspace)@\spxentry{Workspace}\spxextra{class in Workspace}}

\begin{fulllineitems}
\phantomsection\label{\detokenize{autoapi/Workspace/index:Workspace.Workspace}}\pysiglinewithargsret{\sphinxbfcode{\sphinxupquote{class }}\sphinxcode{\sphinxupquote{Workspace.}}\sphinxbfcode{\sphinxupquote{Workspace}}}{\emph{\DUrole{n}{grid}}, \emph{\DUrole{n}{isFinest}\DUrole{o}{=}\DUrole{default_value}{True}}}{}
\sphinxAtStartPar
Bases: \sphinxcode{\sphinxupquote{abc.ABC}}

\sphinxAtStartPar
Abstract base class for workspaces.
Constructor initializes fields array with Grid fields
\index{grid (Workspace.Workspace.self attribute)@\spxentry{grid}\spxextra{Workspace.Workspace.self attribute}}

\begin{fulllineitems}
\phantomsection\label{\detokenize{autoapi/Workspace/index:Workspace.Workspace.self.grid}}\pysigline{\sphinxcode{\sphinxupquote{self.}}\sphinxbfcode{\sphinxupquote{grid}}}
\sphinxAtStartPar
Inputted grid

\end{fulllineitems}

\index{flds (Workspace.Workspace.self attribute)@\spxentry{flds}\spxextra{Workspace.Workspace.self attribute}}

\begin{fulllineitems}
\phantomsection\label{\detokenize{autoapi/Workspace/index:Workspace.Workspace.self.flds}}\pysigline{\sphinxcode{\sphinxupquote{self.}}\sphinxbfcode{\sphinxupquote{flds}}}
\sphinxAtStartPar
Dictionary of fields for grid

\end{fulllineitems}


\sphinxAtStartPar
Notes: Would not be directly implemented (ABC)
\index{make\_new() (Workspace.Workspace method)@\spxentry{make\_new()}\spxextra{Workspace.Workspace method}}

\begin{fulllineitems}
\phantomsection\label{\detokenize{autoapi/Workspace/index:Workspace.Workspace.make_new}}\pysiglinewithargsret{\sphinxbfcode{\sphinxupquote{abstract }}\sphinxbfcode{\sphinxupquote{make\_new}}}{\emph{\DUrole{n}{self}}, \emph{\DUrole{n}{grid}}}{}
\sphinxAtStartPar
Returns another instance of a workspace

\end{fulllineitems}

\index{get\_grid() (Workspace.Workspace method)@\spxentry{get\_grid()}\spxextra{Workspace.Workspace method}}

\begin{fulllineitems}
\phantomsection\label{\detokenize{autoapi/Workspace/index:Workspace.Workspace.get_grid}}\pysiglinewithargsret{\sphinxbfcode{\sphinxupquote{get\_grid}}}{\emph{\DUrole{n}{self}}}{}
\sphinxAtStartPar
Returns grid object

\end{fulllineitems}

\index{get\_dims() (Workspace.Workspace method)@\spxentry{get\_dims()}\spxextra{Workspace.Workspace method}}

\begin{fulllineitems}
\phantomsection\label{\detokenize{autoapi/Workspace/index:Workspace.Workspace.get_dims}}\pysiglinewithargsret{\sphinxbfcode{\sphinxupquote{get\_dims}}}{\emph{\DUrole{n}{self}}}{}
\sphinxAtStartPar
Returns grid\sphinxhyphen{}level\sphinxhyphen{}specific geometry info

\end{fulllineitems}

\index{get\_geometry() (Workspace.Workspace method)@\spxentry{get\_geometry()}\spxextra{Workspace.Workspace method}}

\begin{fulllineitems}
\phantomsection\label{\detokenize{autoapi/Workspace/index:Workspace.Workspace.get_geometry}}\pysiglinewithargsret{\sphinxbfcode{\sphinxupquote{get\_geometry}}}{\emph{\DUrole{n}{self}}}{}
\sphinxAtStartPar
Returns geometry info

\end{fulllineitems}

\index{add\_field() (Workspace.Workspace method)@\spxentry{add\_field()}\spxextra{Workspace.Workspace method}}

\begin{fulllineitems}
\phantomsection\label{\detokenize{autoapi/Workspace/index:Workspace.Workspace.add_field}}\pysiglinewithargsret{\sphinxbfcode{\sphinxupquote{add\_field}}}{\emph{\DUrole{n}{self}}, \emph{\DUrole{n}{new\_field}}, \emph{\DUrole{n}{fieldName}}, \emph{\DUrole{n}{className}\DUrole{o}{=}\DUrole{default_value}{\textquotesingle{}Grid\textquotesingle{}}}}{}
\sphinxAtStartPar
Add field method for grid. Checks if already an instance.
Raises an error if already in list, adds new field if not.

\end{fulllineitems}

\index{get\_field() (Workspace.Workspace method)@\spxentry{get\_field()}\spxextra{Workspace.Workspace method}}

\begin{fulllineitems}
\phantomsection\label{\detokenize{autoapi/Workspace/index:Workspace.Workspace.get_field}}\pysiglinewithargsret{\sphinxbfcode{\sphinxupquote{get\_field}}}{\emph{\DUrole{n}{self}}, \emph{\DUrole{n}{fieldName}}, \emph{\DUrole{n}{className}\DUrole{o}{=}\DUrole{default_value}{\textquotesingle{}Grid\textquotesingle{}}}}{}
\sphinxAtStartPar
Returns the field. Checks if field aleady exists.
If already exists, raises error.
If does not exist, creats a field.

\end{fulllineitems}

\index{has\_dict() (Workspace.Workspace method)@\spxentry{has\_dict()}\spxextra{Workspace.Workspace method}}

\begin{fulllineitems}
\phantomsection\label{\detokenize{autoapi/Workspace/index:Workspace.Workspace.has_dict}}\pysiglinewithargsret{\sphinxbfcode{\sphinxupquote{has\_dict}}}{\emph{\DUrole{n}{self}}, \emph{\DUrole{n}{className}}}{}
\sphinxAtStartPar
Check if a class dictionary exists.

\end{fulllineitems}

\index{exists() (Workspace.Workspace method)@\spxentry{exists()}\spxextra{Workspace.Workspace method}}

\begin{fulllineitems}
\phantomsection\label{\detokenize{autoapi/Workspace/index:Workspace.Workspace.exists}}\pysiglinewithargsret{\sphinxbfcode{\sphinxupquote{exists}}}{\emph{\DUrole{n}{self}}, \emph{\DUrole{n}{fieldName}}, \emph{\DUrole{n}{className}\DUrole{o}{=}\DUrole{default_value}{\textquotesingle{}Grid\textquotesingle{}}}}{}
\sphinxAtStartPar
Checks if a field exists in a class’s dictionary.
Exists/returns true if the it does have a dictionary,
and if there is a fieldName in self,flds.

\end{fulllineitems}

\index{init\_vars() (Workspace.Workspace method)@\spxentry{init\_vars()}\spxextra{Workspace.Workspace method}}

\begin{fulllineitems}
\phantomsection\label{\detokenize{autoapi/Workspace/index:Workspace.Workspace.init_vars}}\pysiglinewithargsret{\sphinxbfcode{\sphinxupquote{init\_vars}}}{\emph{\DUrole{n}{self}}, \emph{\DUrole{n}{className}}, \emph{\DUrole{n}{vars}}}{}
\sphinxAtStartPar
Initialize class’s stored fields,
must give a dictionary of variables with the following structure:
\sphinxhyphen{} keys are the variable name as a string (e.g. “w”)
\sphinxhyphen{} values are an array of {[}field\_dimensions, state\_dim{]}
\begin{itemize}
\item {} 
\sphinxAtStartPar
field\_dim is usually the grid size in {[}nx, ny{]} format

\item {} 
\sphinxAtStartPar
state\_dim is 1 for scalars

\end{itemize}

\sphinxAtStartPar
Creates a class dictionary.
Creats fields and stores in dictionary.

\end{fulllineitems}

\index{is\_finest() (Workspace.Workspace method)@\spxentry{is\_finest()}\spxextra{Workspace.Workspace method}}

\begin{fulllineitems}
\phantomsection\label{\detokenize{autoapi/Workspace/index:Workspace.Workspace.is_finest}}\pysiglinewithargsret{\sphinxbfcode{\sphinxupquote{is\_finest}}}{\emph{\DUrole{n}{self}}}{}
\sphinxAtStartPar
Checks if finest level of mesh.

\end{fulllineitems}

\index{field\_size() (Workspace.Workspace method)@\spxentry{field\_size()}\spxextra{Workspace.Workspace method}}

\begin{fulllineitems}
\phantomsection\label{\detokenize{autoapi/Workspace/index:Workspace.Workspace.field_size}}\pysiglinewithargsret{\sphinxbfcode{\sphinxupquote{abstract }}\sphinxbfcode{\sphinxupquote{field\_size}}}{\emph{\DUrole{n}{self}}}{}
\sphinxAtStartPar
Returns dimensions of field (\# of control volumes)

\end{fulllineitems}

\index{grid\_size() (Workspace.Workspace method)@\spxentry{grid\_size()}\spxextra{Workspace.Workspace method}}

\begin{fulllineitems}
\phantomsection\label{\detokenize{autoapi/Workspace/index:Workspace.Workspace.grid_size}}\pysiglinewithargsret{\sphinxbfcode{\sphinxupquote{grid\_size}}}{\emph{\DUrole{n}{self}}}{}
\sphinxAtStartPar
Returns dimensions of grid (\# of vertices)

\end{fulllineitems}

\index{edges() (Workspace.Workspace method)@\spxentry{edges()}\spxextra{Workspace.Workspace method}}

\begin{fulllineitems}
\phantomsection\label{\detokenize{autoapi/Workspace/index:Workspace.Workspace.edges}}\pysiglinewithargsret{\sphinxbfcode{\sphinxupquote{abstract }}\sphinxbfcode{\sphinxupquote{edges}}}{\emph{\DUrole{n}{self}}, \emph{\DUrole{n}{i}}, \emph{\DUrole{n}{j}}, \emph{\DUrole{n}{side}}}{}
\sphinxAtStartPar
Returns a Field containing the edge vectors
\begin{quote}\begin{description}
\item[{Parameters}] \leavevmode
\sphinxAtStartPar
\sphinxstyleliteralstrong{\sphinxupquote{dim}} (\sphinxstyleliteralemphasis{\sphinxupquote{0}}\sphinxstyleliteralemphasis{\sphinxupquote{ or }}\sphinxstyleliteralemphasis{\sphinxupquote{1}}) \textendash{} Which edges will be returned (0 for i, 1 for j edges)

\end{description}\end{quote}

\end{fulllineitems}

\index{edge\_normals() (Workspace.Workspace method)@\spxentry{edge\_normals()}\spxextra{Workspace.Workspace method}}

\begin{fulllineitems}
\phantomsection\label{\detokenize{autoapi/Workspace/index:Workspace.Workspace.edge_normals}}\pysiglinewithargsret{\sphinxbfcode{\sphinxupquote{abstract }}\sphinxbfcode{\sphinxupquote{edge\_normals}}}{\emph{\DUrole{n}{self}}, \emph{\DUrole{n}{i}}, \emph{\DUrole{n}{j}}, \emph{\DUrole{n}{side}}}{}
\sphinxAtStartPar
Returns a Field containing the unit normal vectors to the edges along a given dimension
\begin{quote}\begin{description}
\item[{Parameters}] \leavevmode
\sphinxAtStartPar
\sphinxstyleliteralstrong{\sphinxupquote{dim}} (\sphinxstyleliteralemphasis{\sphinxupquote{0}}\sphinxstyleliteralemphasis{\sphinxupquote{ or }}\sphinxstyleliteralemphasis{\sphinxupquote{1}}) \textendash{} Which edges normals will be returned for (0 for i, 1 for j edges)

\end{description}\end{quote}

\end{fulllineitems}


\end{fulllineitems}



\section{\sphinxstyleliteralintitle{\sphinxupquote{BaldwinLomax}}}
\label{\detokenize{autoapi/BaldwinLomax/index:module-BaldwinLomax}}\label{\detokenize{autoapi/BaldwinLomax/index:baldwinlomax}}\label{\detokenize{autoapi/BaldwinLomax/index::doc}}\index{module@\spxentry{module}!BaldwinLomax@\spxentry{BaldwinLomax}}\index{BaldwinLomax@\spxentry{BaldwinLomax}!module@\spxentry{module}}
\sphinxAtStartPar
This module calculates turbulent viscosity at the cell faces.
\begin{description}
\item[{Libraries/Modules:}] \leavevmode
\sphinxAtStartPar
numpy

\end{description}


\subsection{Module Contents}
\label{\detokenize{autoapi/BaldwinLomax/index:module-contents}}

\subsubsection{Functions}
\label{\detokenize{autoapi/BaldwinLomax/index:functions}}

\begin{savenotes}\sphinxatlongtablestart\begin{longtable}[c]{\X{1}{2}\X{1}{2}}
\hline

\endfirsthead

\multicolumn{2}{c}%
{\makebox[0pt]{\sphinxtablecontinued{\tablename\ \thetable{} \textendash{} continued from previous page}}}\\
\hline

\endhead

\hline
\multicolumn{2}{r}{\makebox[0pt][r]{\sphinxtablecontinued{continues on next page}}}\\
\endfoot

\endlastfoot

\sphinxAtStartPar
{\hyperref[\detokenize{autoapi/BaldwinLomax/index:BaldwinLomax.turbulent_viscosity}]{\sphinxcrossref{\sphinxcode{\sphinxupquote{turbulent\_viscosity}}}}}(model, ws, state)
&
\sphinxAtStartPar
Baldwin\sphinxhyphen{}lomax turbulence model:  modtur = 2.
\\
\hline
\end{longtable}\sphinxatlongtableend\end{savenotes}
\index{turbulent\_viscosity() (in module BaldwinLomax)@\spxentry{turbulent\_viscosity()}\spxextra{in module BaldwinLomax}}

\begin{fulllineitems}
\phantomsection\label{\detokenize{autoapi/BaldwinLomax/index:BaldwinLomax.turbulent_viscosity}}\pysiglinewithargsret{\sphinxcode{\sphinxupquote{BaldwinLomax.}}\sphinxbfcode{\sphinxupquote{turbulent\_viscosity}}}{\emph{\DUrole{n}{model}}, \emph{\DUrole{n}{ws}}, \emph{\DUrole{n}{state}}}{}
\sphinxAtStartPar
Baldwin\sphinxhyphen{}lomax turbulence model:  modtur = 2.
Calculates turbulent viscosity at the cell faces.
Averages to obtain cell center values fully vectorized routine.                                         *
Calculates eddy viscosity, vorticity, total velocity, normal distance.
Also calculates outer and innner eddy viscosity.
\index{rev (in module BaldwinLomax)@\spxentry{rev}\spxextra{in module BaldwinLomax}}

\begin{fulllineitems}
\phantomsection\label{\detokenize{autoapi/BaldwinLomax/index:BaldwinLomax.rev}}\pysigline{\sphinxcode{\sphinxupquote{BaldwinLomax.}}\sphinxbfcode{\sphinxupquote{rev}}}
\sphinxAtStartPar
eddy viscocity

\end{fulllineitems}

\index{ylen (in module BaldwinLomax)@\spxentry{ylen}\spxextra{in module BaldwinLomax}}

\begin{fulllineitems}
\phantomsection\label{\detokenize{autoapi/BaldwinLomax/index:BaldwinLomax.ylen}}\pysigline{\sphinxcode{\sphinxupquote{BaldwinLomax.}}\sphinxbfcode{\sphinxupquote{ylen}}}
\sphinxAtStartPar
normal distance

\end{fulllineitems}

\index{vor (in module BaldwinLomax)@\spxentry{vor}\spxextra{in module BaldwinLomax}}

\begin{fulllineitems}
\phantomsection\label{\detokenize{autoapi/BaldwinLomax/index:BaldwinLomax.vor}}\pysigline{\sphinxcode{\sphinxupquote{BaldwinLomax.}}\sphinxbfcode{\sphinxupquote{vor}}}
\sphinxAtStartPar
vorticity

\end{fulllineitems}

\index{vol (in module BaldwinLomax)@\spxentry{vol}\spxextra{in module BaldwinLomax}}

\begin{fulllineitems}
\phantomsection\label{\detokenize{autoapi/BaldwinLomax/index:BaldwinLomax.vol}}\pysigline{\sphinxcode{\sphinxupquote{BaldwinLomax.}}\sphinxbfcode{\sphinxupquote{vol}}}
\sphinxAtStartPar
control volume

\end{fulllineitems}

\index{amuto (in module BaldwinLomax)@\spxentry{amuto}\spxextra{in module BaldwinLomax}}

\begin{fulllineitems}
\phantomsection\label{\detokenize{autoapi/BaldwinLomax/index:BaldwinLomax.amuto}}\pysigline{\sphinxcode{\sphinxupquote{BaldwinLomax.}}\sphinxbfcode{\sphinxupquote{amuto}}}
\sphinxAtStartPar
outer eddy viscosity

\end{fulllineitems}

\index{amuti (in module BaldwinLomax)@\spxentry{amuti}\spxextra{in module BaldwinLomax}}

\begin{fulllineitems}
\phantomsection\label{\detokenize{autoapi/BaldwinLomax/index:BaldwinLomax.amuti}}\pysigline{\sphinxcode{\sphinxupquote{BaldwinLomax.}}\sphinxbfcode{\sphinxupquote{amuti}}}
\sphinxAtStartPar
inner eddy viscosity

\end{fulllineitems}

\subsubsection*{Notes}

\sphinxAtStartPar
Adapted from subroutine turb2.f

\end{fulllineitems}



\section{\sphinxstyleliteralintitle{\sphinxupquote{bcfar}}}
\label{\detokenize{autoapi/bcfar/index:module-bcfar}}\label{\detokenize{autoapi/bcfar/index:bcfar}}\label{\detokenize{autoapi/bcfar/index::doc}}\index{module@\spxentry{module}!bcfar@\spxentry{bcfar}}\index{bcfar@\spxentry{bcfar}!module@\spxentry{module}}

\subsection{Module Contents}
\label{\detokenize{autoapi/bcfar/index:module-contents}}

\subsubsection{Functions}
\label{\detokenize{autoapi/bcfar/index:functions}}

\begin{savenotes}\sphinxatlongtablestart\begin{longtable}[c]{\X{1}{2}\X{1}{2}}
\hline

\endfirsthead

\multicolumn{2}{c}%
{\makebox[0pt]{\sphinxtablecontinued{\tablename\ \thetable{} \textendash{} continued from previous page}}}\\
\hline

\endhead

\hline
\multicolumn{2}{r}{\makebox[0pt][r]{\sphinxtablecontinued{continues on next page}}}\\
\endfoot

\endlastfoot

\sphinxAtStartPar
{\hyperref[\detokenize{autoapi/bcfar/index:bcfar.far_field}]{\sphinxcrossref{\sphinxcode{\sphinxupquote{far\_field}}}}}(bcmodel, model, workspace, state)
&
\sphinxAtStartPar
set values in the far field
\\
\hline
\end{longtable}\sphinxatlongtableend\end{savenotes}
\index{far\_field() (in module bcfar)@\spxentry{far\_field()}\spxextra{in module bcfar}}

\begin{fulllineitems}
\phantomsection\label{\detokenize{autoapi/bcfar/index:bcfar.far_field}}\pysiglinewithargsret{\sphinxcode{\sphinxupquote{bcfar.}}\sphinxbfcode{\sphinxupquote{far\_field}}}{\emph{\DUrole{n}{bcmodel}}, \emph{\DUrole{n}{model}}, \emph{\DUrole{n}{workspace}}, \emph{\DUrole{n}{state}}}{}
\sphinxAtStartPar
set values in the far field
\begin{quote}\begin{description}
\item[{Parameters}] \leavevmode\begin{itemize}
\item {} 
\sphinxAtStartPar
\sphinxstyleliteralstrong{\sphinxupquote{bcmodel}} (\sphinxstyleliteralemphasis{\sphinxupquote{NS\_Arifoil}}) \textendash{} boundary condition object

\item {} 
\sphinxAtStartPar
\sphinxstyleliteralstrong{\sphinxupquote{model}} ({\hyperref[\detokenize{autoapi/NavierStokes/index:NavierStokes.NavierStokes}]{\sphinxcrossref{\sphinxstyleliteralemphasis{\sphinxupquote{NavierStokes}}}}}) \textendash{} physics model

\item {} 
\sphinxAtStartPar
\sphinxstyleliteralstrong{\sphinxupquote{workspace}} ({\hyperref[\detokenize{autoapi/Workspace/index:Workspace.Workspace}]{\sphinxcrossref{\sphinxstyleliteralemphasis{\sphinxupquote{Workspace}}}}}) \textendash{} the relevant fields

\item {} 
\sphinxAtStartPar
\sphinxstyleliteralstrong{\sphinxupquote{state}} ({\hyperref[\detokenize{autoapi/Field/index:Field.Field}]{\sphinxcrossref{\sphinxstyleliteralemphasis{\sphinxupquote{Field}}}}}) \textendash{} containing the density, x\sphinxhyphen{}momentum, y\sphinxhyphen{}momentum, and energy

\end{itemize}

\end{description}\end{quote}

\end{fulllineitems}



\section{\sphinxstyleliteralintitle{\sphinxupquote{bcwall}}}
\label{\detokenize{autoapi/bcwall/index:module-bcwall}}\label{\detokenize{autoapi/bcwall/index:bcwall}}\label{\detokenize{autoapi/bcwall/index::doc}}\index{module@\spxentry{module}!bcwall@\spxentry{bcwall}}\index{bcwall@\spxentry{bcwall}!module@\spxentry{module}}

\subsection{Module Contents}
\label{\detokenize{autoapi/bcwall/index:module-contents}}

\subsubsection{Functions}
\label{\detokenize{autoapi/bcwall/index:functions}}

\begin{savenotes}\sphinxatlongtablestart\begin{longtable}[c]{\X{1}{2}\X{1}{2}}
\hline

\endfirsthead

\multicolumn{2}{c}%
{\makebox[0pt]{\sphinxtablecontinued{\tablename\ \thetable{} \textendash{} continued from previous page}}}\\
\hline

\endhead

\hline
\multicolumn{2}{r}{\makebox[0pt][r]{\sphinxtablecontinued{continues on next page}}}\\
\endfoot

\endlastfoot

\sphinxAtStartPar
{\hyperref[\detokenize{autoapi/bcwall/index:bcwall.wall}]{\sphinxcrossref{\sphinxcode{\sphinxupquote{wall}}}}}(bcmodel, model, workspace, state)
&
\sphinxAtStartPar
set values at the wall
\\
\hline
\end{longtable}\sphinxatlongtableend\end{savenotes}
\index{wall() (in module bcwall)@\spxentry{wall()}\spxextra{in module bcwall}}

\begin{fulllineitems}
\phantomsection\label{\detokenize{autoapi/bcwall/index:bcwall.wall}}\pysiglinewithargsret{\sphinxcode{\sphinxupquote{bcwall.}}\sphinxbfcode{\sphinxupquote{wall}}}{\emph{\DUrole{n}{bcmodel}}, \emph{\DUrole{n}{model}}, \emph{\DUrole{n}{workspace}}, \emph{\DUrole{n}{state}}}{}
\sphinxAtStartPar
set values at the wall
\begin{quote}\begin{description}
\item[{Parameters}] \leavevmode\begin{itemize}
\item {} 
\sphinxAtStartPar
\sphinxstyleliteralstrong{\sphinxupquote{bcmodel}} (\sphinxstyleliteralemphasis{\sphinxupquote{NS\_Arifoil}}) \textendash{} boundary condition object

\item {} 
\sphinxAtStartPar
\sphinxstyleliteralstrong{\sphinxupquote{model}} ({\hyperref[\detokenize{autoapi/NavierStokes/index:NavierStokes.NavierStokes}]{\sphinxcrossref{\sphinxstyleliteralemphasis{\sphinxupquote{NavierStokes}}}}}) \textendash{} physics model

\item {} 
\sphinxAtStartPar
\sphinxstyleliteralstrong{\sphinxupquote{workspace}} ({\hyperref[\detokenize{autoapi/Workspace/index:Workspace.Workspace}]{\sphinxcrossref{\sphinxstyleliteralemphasis{\sphinxupquote{Workspace}}}}}) \textendash{} the relevant fields

\item {} 
\sphinxAtStartPar
\sphinxstyleliteralstrong{\sphinxupquote{state}} ({\hyperref[\detokenize{autoapi/Field/index:Field.Field}]{\sphinxcrossref{\sphinxstyleliteralemphasis{\sphinxupquote{Field}}}}}) \textendash{} containing the density, x\sphinxhyphen{}momentum, y\sphinxhyphen{}momentum, and energy

\end{itemize}

\end{description}\end{quote}

\end{fulllineitems}



\section{\sphinxstyleliteralintitle{\sphinxupquote{bc\_metric}}}
\label{\detokenize{autoapi/bc_metric/index:module-bc_metric}}\label{\detokenize{autoapi/bc_metric/index:bc-metric}}\label{\detokenize{autoapi/bc_metric/index::doc}}\index{module@\spxentry{module}!bc\_metric@\spxentry{bc\_metric}}\index{bc\_metric@\spxentry{bc\_metric}!module@\spxentry{module}}

\subsection{Module Contents}
\label{\detokenize{autoapi/bc_metric/index:module-contents}}

\subsubsection{Functions}
\label{\detokenize{autoapi/bc_metric/index:functions}}

\begin{savenotes}\sphinxatlongtablestart\begin{longtable}[c]{\X{1}{2}\X{1}{2}}
\hline

\endfirsthead

\multicolumn{2}{c}%
{\makebox[0pt]{\sphinxtablecontinued{\tablename\ \thetable{} \textendash{} continued from previous page}}}\\
\hline

\endhead

\hline
\multicolumn{2}{r}{\makebox[0pt][r]{\sphinxtablecontinued{continues on next page}}}\\
\endfoot

\endlastfoot

\sphinxAtStartPar
{\hyperref[\detokenize{autoapi/bc_metric/index:bc_metric.halo_geom}]{\sphinxcrossref{\sphinxcode{\sphinxupquote{halo\_geom}}}}}(self, model, workspace)
&
\sphinxAtStartPar
Sets the geometry values in the halo
\\
\hline
\end{longtable}\sphinxatlongtableend\end{savenotes}
\index{halo\_geom() (in module bc\_metric)@\spxentry{halo\_geom()}\spxextra{in module bc\_metric}}

\begin{fulllineitems}
\phantomsection\label{\detokenize{autoapi/bc_metric/index:bc_metric.halo_geom}}\pysiglinewithargsret{\sphinxcode{\sphinxupquote{bc\_metric.}}\sphinxbfcode{\sphinxupquote{halo\_geom}}}{\emph{\DUrole{n}{self}}, \emph{\DUrole{n}{model}}, \emph{\DUrole{n}{workspace}}}{}
\sphinxAtStartPar
Sets the geometry values in the halo
\begin{quote}\begin{description}
\item[{Parameters}] \leavevmode\begin{itemize}
\item {} 
\sphinxAtStartPar
\sphinxstyleliteralstrong{\sphinxupquote{model}} \textendash{} The physics model

\item {} 
\sphinxAtStartPar
\sphinxstyleliteralstrong{\sphinxupquote{workspace}} \textendash{} The Workspace

\end{itemize}

\end{description}\end{quote}

\end{fulllineitems}



\section{\sphinxstyleliteralintitle{\sphinxupquote{bc\_transfer}}}
\label{\detokenize{autoapi/bc_transfer/index:module-bc_transfer}}\label{\detokenize{autoapi/bc_transfer/index:bc-transfer}}\label{\detokenize{autoapi/bc_transfer/index::doc}}\index{module@\spxentry{module}!bc\_transfer@\spxentry{bc\_transfer}}\index{bc\_transfer@\spxentry{bc\_transfer}!module@\spxentry{module}}

\subsection{Module Contents}
\label{\detokenize{autoapi/bc_transfer/index:module-contents}}

\subsubsection{Functions}
\label{\detokenize{autoapi/bc_transfer/index:functions}}

\begin{savenotes}\sphinxatlongtablestart\begin{longtable}[c]{\X{1}{2}\X{1}{2}}
\hline

\endfirsthead

\multicolumn{2}{c}%
{\makebox[0pt]{\sphinxtablecontinued{\tablename\ \thetable{} \textendash{} continued from previous page}}}\\
\hline

\endhead

\hline
\multicolumn{2}{r}{\makebox[0pt][r]{\sphinxtablecontinued{continues on next page}}}\\
\endfoot

\endlastfoot

\sphinxAtStartPar
{\hyperref[\detokenize{autoapi/bc_transfer/index:bc_transfer.transfer_down}]{\sphinxcrossref{\sphinxcode{\sphinxupquote{transfer\_down}}}}}(self, model, workspace1, workspace2)
&
\sphinxAtStartPar
Sets the geometry values in the halo
\\
\hline
\end{longtable}\sphinxatlongtableend\end{savenotes}
\index{transfer\_down() (in module bc\_transfer)@\spxentry{transfer\_down()}\spxextra{in module bc\_transfer}}

\begin{fulllineitems}
\phantomsection\label{\detokenize{autoapi/bc_transfer/index:bc_transfer.transfer_down}}\pysiglinewithargsret{\sphinxcode{\sphinxupquote{bc\_transfer.}}\sphinxbfcode{\sphinxupquote{transfer\_down}}}{\emph{\DUrole{n}{self}}, \emph{\DUrole{n}{model}}, \emph{\DUrole{n}{workspace1}}, \emph{\DUrole{n}{workspace2}}}{}
\sphinxAtStartPar
Sets the geometry values in the halo
\begin{quote}\begin{description}
\item[{Parameters}] \leavevmode\begin{itemize}
\item {} 
\sphinxAtStartPar
\sphinxstyleliteralstrong{\sphinxupquote{model}} \textendash{} The physics model

\item {} 
\sphinxAtStartPar
\sphinxstyleliteralstrong{\sphinxupquote{workspace1}} \textendash{} The finer Workspace

\item {} 
\sphinxAtStartPar
\sphinxstyleliteralstrong{\sphinxupquote{workspace2}} \textendash{} The coarser Workspace

\end{itemize}

\end{description}\end{quote}

\end{fulllineitems}



\section{\sphinxstyleliteralintitle{\sphinxupquote{BoundaryThickness}}}
\label{\detokenize{autoapi/BoundaryThickness/index:module-BoundaryThickness}}\label{\detokenize{autoapi/BoundaryThickness/index:boundarythickness}}\label{\detokenize{autoapi/BoundaryThickness/index::doc}}\index{module@\spxentry{module}!BoundaryThickness@\spxentry{BoundaryThickness}}\index{BoundaryThickness@\spxentry{BoundaryThickness}!module@\spxentry{module}}
\sphinxAtStartPar
This module calculates boundary layer thickness for viscosity.
\begin{description}
\item[{Libraries/Modules:}] \leavevmode
\sphinxAtStartPar
numpy

\end{description}


\subsection{Module Contents}
\label{\detokenize{autoapi/BoundaryThickness/index:module-contents}}

\subsubsection{Functions}
\label{\detokenize{autoapi/BoundaryThickness/index:functions}}

\begin{savenotes}\sphinxatlongtablestart\begin{longtable}[c]{\X{1}{2}\X{1}{2}}
\hline

\endfirsthead

\multicolumn{2}{c}%
{\makebox[0pt]{\sphinxtablecontinued{\tablename\ \thetable{} \textendash{} continued from previous page}}}\\
\hline

\endhead

\hline
\multicolumn{2}{r}{\makebox[0pt][r]{\sphinxtablecontinued{continues on next page}}}\\
\endfoot

\endlastfoot

\sphinxAtStartPar
{\hyperref[\detokenize{autoapi/BoundaryThickness/index:BoundaryThickness.boundary_thickness}]{\sphinxcrossref{\sphinxcode{\sphinxupquote{boundary\_thickness}}}}}(model, ws, state)
&
\sphinxAtStartPar
Calculates the boundary layer thickness.
\\
\hline
\end{longtable}\sphinxatlongtableend\end{savenotes}
\index{boundary\_thickness() (in module BoundaryThickness)@\spxentry{boundary\_thickness()}\spxextra{in module BoundaryThickness}}

\begin{fulllineitems}
\phantomsection\label{\detokenize{autoapi/BoundaryThickness/index:BoundaryThickness.boundary_thickness}}\pysiglinewithargsret{\sphinxcode{\sphinxupquote{BoundaryThickness.}}\sphinxbfcode{\sphinxupquote{boundary\_thickness}}}{\emph{\DUrole{n}{model}}, \emph{\DUrole{n}{ws}}, \emph{\DUrole{n}{state}}}{}
\sphinxAtStartPar
Calculates the boundary layer thickness.
\subsubsection*{Notes}

\sphinxAtStartPar
Adapted from subroutine delt

\end{fulllineitems}



\section{\sphinxstyleliteralintitle{\sphinxupquote{dflux}}}
\label{\detokenize{autoapi/dflux/index:module-dflux}}\label{\detokenize{autoapi/dflux/index:dflux}}\label{\detokenize{autoapi/dflux/index::doc}}\index{module@\spxentry{module}!dflux@\spxentry{dflux}}\index{dflux@\spxentry{dflux}!module@\spxentry{module}}

\subsection{Module Contents}
\label{\detokenize{autoapi/dflux/index:module-contents}}

\subsubsection{Functions}
\label{\detokenize{autoapi/dflux/index:functions}}

\begin{savenotes}\sphinxatlongtablestart\begin{longtable}[c]{\X{1}{2}\X{1}{2}}
\hline

\endfirsthead

\multicolumn{2}{c}%
{\makebox[0pt]{\sphinxtablecontinued{\tablename\ \thetable{} \textendash{} continued from previous page}}}\\
\hline

\endhead

\hline
\multicolumn{2}{r}{\makebox[0pt][r]{\sphinxtablecontinued{continues on next page}}}\\
\endfoot

\endlastfoot

\sphinxAtStartPar
{\hyperref[\detokenize{autoapi/dflux/index:dflux.dflux}]{\sphinxcrossref{\sphinxcode{\sphinxupquote{dflux}}}}}(model, ws, state, dw, rfil)
&
\sphinxAtStartPar
calculate artificial dissipation fluxes on finest mesh using blended first and
\\
\hline
\end{longtable}\sphinxatlongtableend\end{savenotes}
\index{dflux() (in module dflux)@\spxentry{dflux()}\spxextra{in module dflux}}

\begin{fulllineitems}
\phantomsection\label{\detokenize{autoapi/dflux/index:dflux.dflux}}\pysiglinewithargsret{\sphinxcode{\sphinxupquote{dflux.}}\sphinxbfcode{\sphinxupquote{dflux}}}{\emph{\DUrole{n}{model}}, \emph{\DUrole{n}{ws}}, \emph{\DUrole{n}{state}}, \emph{\DUrole{n}{dw}}, \emph{\DUrole{n}{rfil}}}{}
\sphinxAtStartPar
calculate artificial dissipation fluxes on finest mesh using blended first and
third order fluxes
\begin{quote}\begin{description}
\item[{Parameters}] \leavevmode\begin{itemize}
\item {} 
\sphinxAtStartPar
\sphinxstyleliteralstrong{\sphinxupquote{model}} ({\hyperref[\detokenize{autoapi/NavierStokes/index:NavierStokes.NavierStokes}]{\sphinxcrossref{\sphinxstyleliteralemphasis{\sphinxupquote{NavierStokes}}}}}) \textendash{} physics model

\item {} 
\sphinxAtStartPar
\sphinxstyleliteralstrong{\sphinxupquote{workspace}} ({\hyperref[\detokenize{autoapi/Workspace/index:Workspace.Workspace}]{\sphinxcrossref{\sphinxstyleliteralemphasis{\sphinxupquote{Workspace}}}}}) \textendash{} contains the relevant Fields

\item {} 
\sphinxAtStartPar
\sphinxstyleliteralstrong{\sphinxupquote{state}} ({\hyperref[\detokenize{autoapi/Field/index:Field.Field}]{\sphinxcrossref{\sphinxstyleliteralemphasis{\sphinxupquote{Field}}}}}) \textendash{} density, x\sphinxhyphen{}momentum, y\sphinxhyphen{}momentum, and energy

\item {} 
\sphinxAtStartPar
\sphinxstyleliteralstrong{\sphinxupquote{dw}} ({\hyperref[\detokenize{autoapi/Field/index:Field.Field}]{\sphinxcrossref{\sphinxstyleliteralemphasis{\sphinxupquote{Field}}}}}) \textendash{} to store new residuals after completing fluxes

\item {} 
\sphinxAtStartPar
\sphinxstyleliteralstrong{\sphinxupquote{rfil}} (\sphinxstyleliteralemphasis{\sphinxupquote{float}}) \textendash{} relaxation factor determining balance between viscous and artificial dissipation fluxes

\end{itemize}

\end{description}\end{quote}

\end{fulllineitems}



\section{\sphinxstyleliteralintitle{\sphinxupquote{dfluxc}}}
\label{\detokenize{autoapi/dfluxc/index:module-dfluxc}}\label{\detokenize{autoapi/dfluxc/index:dfluxc}}\label{\detokenize{autoapi/dfluxc/index::doc}}\index{module@\spxentry{module}!dfluxc@\spxentry{dfluxc}}\index{dfluxc@\spxentry{dfluxc}!module@\spxentry{module}}

\subsection{Module Contents}
\label{\detokenize{autoapi/dfluxc/index:module-contents}}

\subsubsection{Functions}
\label{\detokenize{autoapi/dfluxc/index:functions}}

\begin{savenotes}\sphinxatlongtablestart\begin{longtable}[c]{\X{1}{2}\X{1}{2}}
\hline

\endfirsthead

\multicolumn{2}{c}%
{\makebox[0pt]{\sphinxtablecontinued{\tablename\ \thetable{} \textendash{} continued from previous page}}}\\
\hline

\endhead

\hline
\multicolumn{2}{r}{\makebox[0pt][r]{\sphinxtablecontinued{continues on next page}}}\\
\endfoot

\endlastfoot

\sphinxAtStartPar
{\hyperref[\detokenize{autoapi/dfluxc/index:dfluxc.dfluxc}]{\sphinxcrossref{\sphinxcode{\sphinxupquote{dfluxc}}}}}(model, ws, state, dw, rfil)
&
\sphinxAtStartPar
calculate artificial dissipation fluxes on coarse meshes using blended first order
\\
\hline
\end{longtable}\sphinxatlongtableend\end{savenotes}
\index{dfluxc() (in module dfluxc)@\spxentry{dfluxc()}\spxextra{in module dfluxc}}

\begin{fulllineitems}
\phantomsection\label{\detokenize{autoapi/dfluxc/index:dfluxc.dfluxc}}\pysiglinewithargsret{\sphinxcode{\sphinxupquote{dfluxc.}}\sphinxbfcode{\sphinxupquote{dfluxc}}}{\emph{\DUrole{n}{model}}, \emph{\DUrole{n}{ws}}, \emph{\DUrole{n}{state}}, \emph{\DUrole{n}{dw}}, \emph{\DUrole{n}{rfil}}}{}
\sphinxAtStartPar
calculate artificial dissipation fluxes on coarse meshes using blended first order
fluxes scaled to spectral radius
\begin{quote}\begin{description}
\item[{Parameters}] \leavevmode\begin{itemize}
\item {} 
\sphinxAtStartPar
\sphinxstyleliteralstrong{\sphinxupquote{model}} ({\hyperref[\detokenize{autoapi/NavierStokes/index:NavierStokes.NavierStokes}]{\sphinxcrossref{\sphinxstyleliteralemphasis{\sphinxupquote{NavierStokes}}}}}) \textendash{} physics model

\item {} 
\sphinxAtStartPar
\sphinxstyleliteralstrong{\sphinxupquote{workspace}} ({\hyperref[\detokenize{autoapi/Workspace/index:Workspace.Workspace}]{\sphinxcrossref{\sphinxstyleliteralemphasis{\sphinxupquote{Workspace}}}}}) \textendash{} contains the relevant Fields

\item {} 
\sphinxAtStartPar
\sphinxstyleliteralstrong{\sphinxupquote{state}} ({\hyperref[\detokenize{autoapi/Field/index:Field.Field}]{\sphinxcrossref{\sphinxstyleliteralemphasis{\sphinxupquote{Field}}}}}) \textendash{} density, x\sphinxhyphen{}momentum, y\sphinxhyphen{}momentum, and energy

\item {} 
\sphinxAtStartPar
\sphinxstyleliteralstrong{\sphinxupquote{dw}} ({\hyperref[\detokenize{autoapi/Field/index:Field.Field}]{\sphinxcrossref{\sphinxstyleliteralemphasis{\sphinxupquote{Field}}}}}) \textendash{} to store new residuals after completing fluxes

\item {} 
\sphinxAtStartPar
\sphinxstyleliteralstrong{\sphinxupquote{rfil}} (\sphinxstyleliteralemphasis{\sphinxupquote{float}}) \textendash{} relaxation factor determining balance between viscous and artificial dissipation fluxes

\end{itemize}

\end{description}\end{quote}

\end{fulllineitems}



\section{\sphinxstyleliteralintitle{\sphinxupquote{eflux}}}
\label{\detokenize{autoapi/eflux/index:module-eflux}}\label{\detokenize{autoapi/eflux/index:eflux}}\label{\detokenize{autoapi/eflux/index::doc}}\index{module@\spxentry{module}!eflux@\spxentry{eflux}}\index{eflux@\spxentry{eflux}!module@\spxentry{module}}

\subsection{Module Contents}
\label{\detokenize{autoapi/eflux/index:module-contents}}

\subsubsection{Functions}
\label{\detokenize{autoapi/eflux/index:functions}}

\begin{savenotes}\sphinxatlongtablestart\begin{longtable}[c]{\X{1}{2}\X{1}{2}}
\hline

\endfirsthead

\multicolumn{2}{c}%
{\makebox[0pt]{\sphinxtablecontinued{\tablename\ \thetable{} \textendash{} continued from previous page}}}\\
\hline

\endhead

\hline
\multicolumn{2}{r}{\makebox[0pt][r]{\sphinxtablecontinued{continues on next page}}}\\
\endfoot

\endlastfoot

\sphinxAtStartPar
{\hyperref[\detokenize{autoapi/eflux/index:eflux.eflux}]{\sphinxcrossref{\sphinxcode{\sphinxupquote{eflux}}}}}(model, ws, state, dw)
&
\sphinxAtStartPar
calculate convective fluxes
\\
\hline
\end{longtable}\sphinxatlongtableend\end{savenotes}
\index{eflux() (in module eflux)@\spxentry{eflux()}\spxextra{in module eflux}}

\begin{fulllineitems}
\phantomsection\label{\detokenize{autoapi/eflux/index:eflux.eflux}}\pysiglinewithargsret{\sphinxcode{\sphinxupquote{eflux.}}\sphinxbfcode{\sphinxupquote{eflux}}}{\emph{\DUrole{n}{model}}, \emph{\DUrole{n}{ws}}, \emph{\DUrole{n}{state}}, \emph{\DUrole{n}{dw}}}{}
\sphinxAtStartPar
calculate convective fluxes
\begin{quote}\begin{description}
\item[{Parameters}] \leavevmode\begin{itemize}
\item {} 
\sphinxAtStartPar
\sphinxstyleliteralstrong{\sphinxupquote{model}} ({\hyperref[\detokenize{autoapi/NavierStokes/index:NavierStokes.NavierStokes}]{\sphinxcrossref{\sphinxstyleliteralemphasis{\sphinxupquote{NavierStokes}}}}}) \textendash{} physics model

\item {} 
\sphinxAtStartPar
\sphinxstyleliteralstrong{\sphinxupquote{workspace}} ({\hyperref[\detokenize{autoapi/Workspace/index:Workspace.Workspace}]{\sphinxcrossref{\sphinxstyleliteralemphasis{\sphinxupquote{Workspace}}}}}) \textendash{} the relevant fields

\item {} 
\sphinxAtStartPar
\sphinxstyleliteralstrong{\sphinxupquote{state}} ({\hyperref[\detokenize{autoapi/Field/index:Field.Field}]{\sphinxcrossref{\sphinxstyleliteralemphasis{\sphinxupquote{Field}}}}}) \textendash{} containing the density, x\sphinxhyphen{}momentum, y\sphinxhyphen{}momentum, and energy

\item {} 
\sphinxAtStartPar
\sphinxstyleliteralstrong{\sphinxupquote{dw}} ({\hyperref[\detokenize{autoapi/Field/index:Field.Field}]{\sphinxcrossref{\sphinxstyleliteralemphasis{\sphinxupquote{Field}}}}}) \textendash{} to store new residuals after completing fluxes

\end{itemize}

\end{description}\end{quote}

\end{fulllineitems}



\section{\sphinxstyleliteralintitle{\sphinxupquote{halo}}}
\label{\detokenize{autoapi/halo/index:module-halo}}\label{\detokenize{autoapi/halo/index:halo}}\label{\detokenize{autoapi/halo/index::doc}}\index{module@\spxentry{module}!halo@\spxentry{halo}}\index{halo@\spxentry{halo}!module@\spxentry{module}}

\subsection{Module Contents}
\label{\detokenize{autoapi/halo/index:module-contents}}

\subsubsection{Functions}
\label{\detokenize{autoapi/halo/index:functions}}

\begin{savenotes}\sphinxatlongtablestart\begin{longtable}[c]{\X{1}{2}\X{1}{2}}
\hline

\endfirsthead

\multicolumn{2}{c}%
{\makebox[0pt]{\sphinxtablecontinued{\tablename\ \thetable{} \textendash{} continued from previous page}}}\\
\hline

\endhead

\hline
\multicolumn{2}{r}{\makebox[0pt][r]{\sphinxtablecontinued{continues on next page}}}\\
\endfoot

\endlastfoot

\sphinxAtStartPar
{\hyperref[\detokenize{autoapi/halo/index:halo.halo}]{\sphinxcrossref{\sphinxcode{\sphinxupquote{halo}}}}}(bcmodel, model, workspace, state)
&
\sphinxAtStartPar
assign values in the ghost cells
\\
\hline
\end{longtable}\sphinxatlongtableend\end{savenotes}
\index{halo() (in module halo)@\spxentry{halo()}\spxextra{in module halo}}

\begin{fulllineitems}
\phantomsection\label{\detokenize{autoapi/halo/index:halo.halo}}\pysiglinewithargsret{\sphinxcode{\sphinxupquote{halo.}}\sphinxbfcode{\sphinxupquote{halo}}}{\emph{\DUrole{n}{bcmodel}}, \emph{\DUrole{n}{model}}, \emph{\DUrole{n}{workspace}}, \emph{\DUrole{n}{state}}}{}
\sphinxAtStartPar
assign values in the ghost cells
\begin{quote}\begin{description}
\item[{Parameters}] \leavevmode\begin{itemize}
\item {} 
\sphinxAtStartPar
\sphinxstyleliteralstrong{\sphinxupquote{bcmodel}} (\sphinxstyleliteralemphasis{\sphinxupquote{NS\_Arifoil}}) \textendash{} boundary condition object

\item {} 
\sphinxAtStartPar
\sphinxstyleliteralstrong{\sphinxupquote{model}} ({\hyperref[\detokenize{autoapi/NavierStokes/index:NavierStokes.NavierStokes}]{\sphinxcrossref{\sphinxstyleliteralemphasis{\sphinxupquote{NavierStokes}}}}}) \textendash{} physics model

\item {} 
\sphinxAtStartPar
\sphinxstyleliteralstrong{\sphinxupquote{workspace}} ({\hyperref[\detokenize{autoapi/Workspace/index:Workspace.Workspace}]{\sphinxcrossref{\sphinxstyleliteralemphasis{\sphinxupquote{Workspace}}}}}) \textendash{} the relevant fields

\item {} 
\sphinxAtStartPar
\sphinxstyleliteralstrong{\sphinxupquote{state}} ({\hyperref[\detokenize{autoapi/Field/index:Field.Field}]{\sphinxcrossref{\sphinxstyleliteralemphasis{\sphinxupquote{Field}}}}}) \textendash{} containing the density, x\sphinxhyphen{}momentum, y\sphinxhyphen{}momentum, and energy

\end{itemize}

\end{description}\end{quote}

\end{fulllineitems}



\section{\sphinxstyleliteralintitle{\sphinxupquote{stability\_fast}}}
\label{\detokenize{autoapi/stability_fast/index:module-stability_fast}}\label{\detokenize{autoapi/stability_fast/index:stability-fast}}\label{\detokenize{autoapi/stability_fast/index::doc}}\index{module@\spxentry{module}!stability\_fast@\spxentry{stability\_fast}}\index{stability\_fast@\spxentry{stability\_fast}!module@\spxentry{module}}

\subsection{Module Contents}
\label{\detokenize{autoapi/stability_fast/index:module-contents}}

\subsubsection{Functions}
\label{\detokenize{autoapi/stability_fast/index:functions}}

\begin{savenotes}\sphinxatlongtablestart\begin{longtable}[c]{\X{1}{2}\X{1}{2}}
\hline

\endfirsthead

\multicolumn{2}{c}%
{\makebox[0pt]{\sphinxtablecontinued{\tablename\ \thetable{} \textendash{} continued from previous page}}}\\
\hline

\endhead

\hline
\multicolumn{2}{r}{\makebox[0pt][r]{\sphinxtablecontinued{continues on next page}}}\\
\endfoot

\endlastfoot

\sphinxAtStartPar
{\hyperref[\detokenize{autoapi/stability_fast/index:stability_fast.stability}]{\sphinxcrossref{\sphinxcode{\sphinxupquote{stability}}}}}(self, model, workspace, state)
&
\sphinxAtStartPar
Calculates timestep limits to maintain stability
\\
\hline
\end{longtable}\sphinxatlongtableend\end{savenotes}
\index{stability() (in module stability\_fast)@\spxentry{stability()}\spxextra{in module stability\_fast}}

\begin{fulllineitems}
\phantomsection\label{\detokenize{autoapi/stability_fast/index:stability_fast.stability}}\pysiglinewithargsret{\sphinxcode{\sphinxupquote{stability\_fast.}}\sphinxbfcode{\sphinxupquote{stability}}}{\emph{\DUrole{n}{self}}, \emph{\DUrole{n}{model}}, \emph{\DUrole{n}{workspace}}, \emph{\DUrole{n}{state}}}{}
\sphinxAtStartPar
Calculates timestep limits to maintain stability
\begin{quote}\begin{description}
\item[{Parameters}] \leavevmode\begin{itemize}
\item {} 
\sphinxAtStartPar
\sphinxstyleliteralstrong{\sphinxupquote{model}} ({\hyperref[\detokenize{autoapi/Model/index:Model.Model}]{\sphinxcrossref{\sphinxstyleliteralemphasis{\sphinxupquote{Model}}}}}) \textendash{} The physics model

\item {} 
\sphinxAtStartPar
\sphinxstyleliteralstrong{\sphinxupquote{workspace}} ({\hyperref[\detokenize{autoapi/Workspace/index:Workspace.Workspace}]{\sphinxcrossref{\sphinxstyleliteralemphasis{\sphinxupquote{Workspace}}}}}) \textendash{} The current Workspace

\item {} 
\sphinxAtStartPar
\sphinxstyleliteralstrong{\sphinxupquote{state}} ({\hyperref[\detokenize{autoapi/Field/index:Field.Field}]{\sphinxcrossref{\sphinxstyleliteralemphasis{\sphinxupquote{Field}}}}}) \textendash{} Field containing current state

\end{itemize}

\end{description}\end{quote}

\end{fulllineitems}



\section{\sphinxstyleliteralintitle{\sphinxupquote{Viscosity}}}
\label{\detokenize{autoapi/Viscosity/index:module-Viscosity}}\label{\detokenize{autoapi/Viscosity/index:viscosity}}\label{\detokenize{autoapi/Viscosity/index::doc}}\index{module@\spxentry{module}!Viscosity@\spxentry{Viscosity}}\index{Viscosity@\spxentry{Viscosity}!module@\spxentry{module}}
\sphinxAtStartPar
This module computes viscosity coefficients

\sphinxAtStartPar
Libraries/Modules:
numpy

\sphinxAtStartPar
BaldwinLomax

\sphinxAtStartPar
BoundaryThickness


\subsection{Module Contents}
\label{\detokenize{autoapi/Viscosity/index:module-contents}}

\subsubsection{Functions}
\label{\detokenize{autoapi/Viscosity/index:functions}}

\begin{savenotes}\sphinxatlongtablestart\begin{longtable}[c]{\X{1}{2}\X{1}{2}}
\hline

\endfirsthead

\multicolumn{2}{c}%
{\makebox[0pt]{\sphinxtablecontinued{\tablename\ \thetable{} \textendash{} continued from previous page}}}\\
\hline

\endhead

\hline
\multicolumn{2}{r}{\makebox[0pt][r]{\sphinxtablecontinued{continues on next page}}}\\
\endfoot

\endlastfoot

\sphinxAtStartPar
{\hyperref[\detokenize{autoapi/Viscosity/index:Viscosity.compute_viscosity}]{\sphinxcrossref{\sphinxcode{\sphinxupquote{compute\_viscosity}}}}}(model, ws, state)
&
\sphinxAtStartPar
Computes viscosity coefficients.
\\
\hline
\end{longtable}\sphinxatlongtableend\end{savenotes}
\index{compute\_viscosity() (in module Viscosity)@\spxentry{compute\_viscosity()}\spxextra{in module Viscosity}}

\begin{fulllineitems}
\phantomsection\label{\detokenize{autoapi/Viscosity/index:Viscosity.compute_viscosity}}\pysiglinewithargsret{\sphinxcode{\sphinxupquote{Viscosity.}}\sphinxbfcode{\sphinxupquote{compute\_viscosity}}}{\emph{\DUrole{n}{model}}, \emph{\DUrole{n}{ws}}, \emph{\DUrole{n}{state}}}{}~\begin{description}
\item[{Computes viscosity coefficients.}] \leavevmode
\sphinxAtStartPar
First, computes the molecular viscosity.
Then continues for turbulent, Baldwin Lomax Model
or runs the RNG algebraic model.
Next, calculates the boundary layer thickness
Solves for the eddy viscosity.

\end{description}
\index{rlv (in module Viscosity)@\spxentry{rlv}\spxextra{in module Viscosity}}

\begin{fulllineitems}
\phantomsection\label{\detokenize{autoapi/Viscosity/index:Viscosity.rlv}}\pysigline{\sphinxcode{\sphinxupquote{Viscosity.}}\sphinxbfcode{\sphinxupquote{rlv}}}
\sphinxAtStartPar
laminar viscosity

\end{fulllineitems}

\index{rev (in module Viscosity)@\spxentry{rev}\spxextra{in module Viscosity}}

\begin{fulllineitems}
\phantomsection\label{\detokenize{autoapi/Viscosity/index:Viscosity.rev}}\pysigline{\sphinxcode{\sphinxupquote{Viscosity.}}\sphinxbfcode{\sphinxupquote{rev}}}
\sphinxAtStartPar
eddy viscosity

\end{fulllineitems}

\subsubsection*{Notes}

\sphinxAtStartPar
Adapted from subroutine viscf.f

\end{fulllineitems}



\section{\sphinxstyleliteralintitle{\sphinxupquote{bcfar\_wrap}}}
\label{\detokenize{autoapi/bcfar_wrap/index:module-bcfar_wrap}}\label{\detokenize{autoapi/bcfar_wrap/index:bcfar-wrap}}\label{\detokenize{autoapi/bcfar_wrap/index::doc}}\index{module@\spxentry{module}!bcfar\_wrap@\spxentry{bcfar\_wrap}}\index{bcfar\_wrap@\spxentry{bcfar\_wrap}!module@\spxentry{module}}

\subsection{Module Contents}
\label{\detokenize{autoapi/bcfar_wrap/index:module-contents}}

\subsubsection{Functions}
\label{\detokenize{autoapi/bcfar_wrap/index:functions}}

\begin{savenotes}\sphinxatlongtablestart\begin{longtable}[c]{\X{1}{2}\X{1}{2}}
\hline

\endfirsthead

\multicolumn{2}{c}%
{\makebox[0pt]{\sphinxtablecontinued{\tablename\ \thetable{} \textendash{} continued from previous page}}}\\
\hline

\endhead

\hline
\multicolumn{2}{r}{\makebox[0pt][r]{\sphinxtablecontinued{continues on next page}}}\\
\endfoot

\endlastfoot

\sphinxAtStartPar
{\hyperref[\detokenize{autoapi/bcfar_wrap/index:bcfar_wrap.bc_far}]{\sphinxcrossref{\sphinxcode{\sphinxupquote{bc\_far}}}}}(self, model, workspace, state)
&
\sphinxAtStartPar

\\
\hline
\end{longtable}\sphinxatlongtableend\end{savenotes}
\index{bc\_far() (in module bcfar\_wrap)@\spxentry{bc\_far()}\spxextra{in module bcfar\_wrap}}

\begin{fulllineitems}
\phantomsection\label{\detokenize{autoapi/bcfar_wrap/index:bcfar_wrap.bc_far}}\pysiglinewithargsret{\sphinxcode{\sphinxupquote{bcfar\_wrap.}}\sphinxbfcode{\sphinxupquote{bc\_far}}}{\emph{\DUrole{n}{self}}, \emph{\DUrole{n}{model}}, \emph{\DUrole{n}{workspace}}, \emph{\DUrole{n}{state}}}{}
\end{fulllineitems}



\section{\sphinxstyleliteralintitle{\sphinxupquote{bcwall\_wrap}}}
\label{\detokenize{autoapi/bcwall_wrap/index:module-bcwall_wrap}}\label{\detokenize{autoapi/bcwall_wrap/index:bcwall-wrap}}\label{\detokenize{autoapi/bcwall_wrap/index::doc}}\index{module@\spxentry{module}!bcwall\_wrap@\spxentry{bcwall\_wrap}}\index{bcwall\_wrap@\spxentry{bcwall\_wrap}!module@\spxentry{module}}

\subsection{Module Contents}
\label{\detokenize{autoapi/bcwall_wrap/index:module-contents}}

\subsubsection{Functions}
\label{\detokenize{autoapi/bcwall_wrap/index:functions}}

\begin{savenotes}\sphinxatlongtablestart\begin{longtable}[c]{\X{1}{2}\X{1}{2}}
\hline

\endfirsthead

\multicolumn{2}{c}%
{\makebox[0pt]{\sphinxtablecontinued{\tablename\ \thetable{} \textendash{} continued from previous page}}}\\
\hline

\endhead

\hline
\multicolumn{2}{r}{\makebox[0pt][r]{\sphinxtablecontinued{continues on next page}}}\\
\endfoot

\endlastfoot

\sphinxAtStartPar
{\hyperref[\detokenize{autoapi/bcwall_wrap/index:bcwall_wrap.bc_wall}]{\sphinxcrossref{\sphinxcode{\sphinxupquote{bc\_wall}}}}}(self, model, workspace, state)
&
\sphinxAtStartPar

\\
\hline
\end{longtable}\sphinxatlongtableend\end{savenotes}
\index{bc\_wall() (in module bcwall\_wrap)@\spxentry{bc\_wall()}\spxextra{in module bcwall\_wrap}}

\begin{fulllineitems}
\phantomsection\label{\detokenize{autoapi/bcwall_wrap/index:bcwall_wrap.bc_wall}}\pysiglinewithargsret{\sphinxcode{\sphinxupquote{bcwall\_wrap.}}\sphinxbfcode{\sphinxupquote{bc\_wall}}}{\emph{\DUrole{n}{self}}, \emph{\DUrole{n}{model}}, \emph{\DUrole{n}{workspace}}, \emph{\DUrole{n}{state}}}{}
\end{fulllineitems}



\section{\sphinxstyleliteralintitle{\sphinxupquote{dfluxc\_wrap}}}
\label{\detokenize{autoapi/dfluxc_wrap/index:module-dfluxc_wrap}}\label{\detokenize{autoapi/dfluxc_wrap/index:dfluxc-wrap}}\label{\detokenize{autoapi/dfluxc_wrap/index::doc}}\index{module@\spxentry{module}!dfluxc\_wrap@\spxentry{dfluxc\_wrap}}\index{dfluxc\_wrap@\spxentry{dfluxc\_wrap}!module@\spxentry{module}}

\subsection{Module Contents}
\label{\detokenize{autoapi/dfluxc_wrap/index:module-contents}}

\subsubsection{Functions}
\label{\detokenize{autoapi/dfluxc_wrap/index:functions}}

\begin{savenotes}\sphinxatlongtablestart\begin{longtable}[c]{\X{1}{2}\X{1}{2}}
\hline

\endfirsthead

\multicolumn{2}{c}%
{\makebox[0pt]{\sphinxtablecontinued{\tablename\ \thetable{} \textendash{} continued from previous page}}}\\
\hline

\endhead

\hline
\multicolumn{2}{r}{\makebox[0pt][r]{\sphinxtablecontinued{continues on next page}}}\\
\endfoot

\endlastfoot

\sphinxAtStartPar
{\hyperref[\detokenize{autoapi/dfluxc_wrap/index:dfluxc_wrap.dfluxc}]{\sphinxcrossref{\sphinxcode{\sphinxupquote{dfluxc}}}}}(model, ws, w, dw, fw, rfil)
&
\sphinxAtStartPar

\\
\hline
\end{longtable}\sphinxatlongtableend\end{savenotes}
\index{dfluxc() (in module dfluxc\_wrap)@\spxentry{dfluxc()}\spxextra{in module dfluxc\_wrap}}

\begin{fulllineitems}
\phantomsection\label{\detokenize{autoapi/dfluxc_wrap/index:dfluxc_wrap.dfluxc}}\pysiglinewithargsret{\sphinxcode{\sphinxupquote{dfluxc\_wrap.}}\sphinxbfcode{\sphinxupquote{dfluxc}}}{\emph{\DUrole{n}{model}}, \emph{\DUrole{n}{ws}}, \emph{\DUrole{n}{w}}, \emph{\DUrole{n}{dw}}, \emph{\DUrole{n}{fw}}, \emph{\DUrole{n}{rfil}}}{}
\end{fulllineitems}



\section{\sphinxstyleliteralintitle{\sphinxupquote{dflux\_wrap}}}
\label{\detokenize{autoapi/dflux_wrap/index:module-dflux_wrap}}\label{\detokenize{autoapi/dflux_wrap/index:dflux-wrap}}\label{\detokenize{autoapi/dflux_wrap/index::doc}}\index{module@\spxentry{module}!dflux\_wrap@\spxentry{dflux\_wrap}}\index{dflux\_wrap@\spxentry{dflux\_wrap}!module@\spxentry{module}}

\subsection{Module Contents}
\label{\detokenize{autoapi/dflux_wrap/index:module-contents}}

\subsubsection{Functions}
\label{\detokenize{autoapi/dflux_wrap/index:functions}}

\begin{savenotes}\sphinxatlongtablestart\begin{longtable}[c]{\X{1}{2}\X{1}{2}}
\hline

\endfirsthead

\multicolumn{2}{c}%
{\makebox[0pt]{\sphinxtablecontinued{\tablename\ \thetable{} \textendash{} continued from previous page}}}\\
\hline

\endhead

\hline
\multicolumn{2}{r}{\makebox[0pt][r]{\sphinxtablecontinued{continues on next page}}}\\
\endfoot

\endlastfoot

\sphinxAtStartPar
{\hyperref[\detokenize{autoapi/dflux_wrap/index:dflux_wrap.dflux}]{\sphinxcrossref{\sphinxcode{\sphinxupquote{dflux}}}}}(model, ws, w, dw, rfil)
&
\sphinxAtStartPar

\\
\hline
\end{longtable}\sphinxatlongtableend\end{savenotes}
\index{dflux() (in module dflux\_wrap)@\spxentry{dflux()}\spxextra{in module dflux\_wrap}}

\begin{fulllineitems}
\phantomsection\label{\detokenize{autoapi/dflux_wrap/index:dflux_wrap.dflux}}\pysiglinewithargsret{\sphinxcode{\sphinxupquote{dflux\_wrap.}}\sphinxbfcode{\sphinxupquote{dflux}}}{\emph{\DUrole{n}{model}}, \emph{\DUrole{n}{ws}}, \emph{\DUrole{n}{w}}, \emph{\DUrole{n}{dw}}, \emph{\DUrole{n}{rfil}}}{}
\end{fulllineitems}



\section{\sphinxstyleliteralintitle{\sphinxupquote{eflux\_wrap}}}
\label{\detokenize{autoapi/eflux_wrap/index:module-eflux_wrap}}\label{\detokenize{autoapi/eflux_wrap/index:eflux-wrap}}\label{\detokenize{autoapi/eflux_wrap/index::doc}}\index{module@\spxentry{module}!eflux\_wrap@\spxentry{eflux\_wrap}}\index{eflux\_wrap@\spxentry{eflux\_wrap}!module@\spxentry{module}}

\subsection{Module Contents}
\label{\detokenize{autoapi/eflux_wrap/index:module-contents}}

\subsubsection{Functions}
\label{\detokenize{autoapi/eflux_wrap/index:functions}}

\begin{savenotes}\sphinxatlongtablestart\begin{longtable}[c]{\X{1}{2}\X{1}{2}}
\hline

\endfirsthead

\multicolumn{2}{c}%
{\makebox[0pt]{\sphinxtablecontinued{\tablename\ \thetable{} \textendash{} continued from previous page}}}\\
\hline

\endhead

\hline
\multicolumn{2}{r}{\makebox[0pt][r]{\sphinxtablecontinued{continues on next page}}}\\
\endfoot

\endlastfoot

\sphinxAtStartPar
{\hyperref[\detokenize{autoapi/eflux_wrap/index:eflux_wrap.eflux}]{\sphinxcrossref{\sphinxcode{\sphinxupquote{eflux}}}}}(self, ws, w, dw)
&
\sphinxAtStartPar

\\
\hline
\end{longtable}\sphinxatlongtableend\end{savenotes}
\index{eflux() (in module eflux\_wrap)@\spxentry{eflux()}\spxextra{in module eflux\_wrap}}

\begin{fulllineitems}
\phantomsection\label{\detokenize{autoapi/eflux_wrap/index:eflux_wrap.eflux}}\pysiglinewithargsret{\sphinxcode{\sphinxupquote{eflux\_wrap.}}\sphinxbfcode{\sphinxupquote{eflux}}}{\emph{\DUrole{n}{self}}, \emph{\DUrole{n}{ws}}, \emph{\DUrole{n}{w}}, \emph{\DUrole{n}{dw}}}{}
\end{fulllineitems}



\section{\sphinxstyleliteralintitle{\sphinxupquote{halo\_wrap}}}
\label{\detokenize{autoapi/halo_wrap/index:module-halo_wrap}}\label{\detokenize{autoapi/halo_wrap/index:halo-wrap}}\label{\detokenize{autoapi/halo_wrap/index::doc}}\index{module@\spxentry{module}!halo\_wrap@\spxentry{halo\_wrap}}\index{halo\_wrap@\spxentry{halo\_wrap}!module@\spxentry{module}}

\subsection{Module Contents}
\label{\detokenize{autoapi/halo_wrap/index:module-contents}}

\subsubsection{Functions}
\label{\detokenize{autoapi/halo_wrap/index:functions}}

\begin{savenotes}\sphinxatlongtablestart\begin{longtable}[c]{\X{1}{2}\X{1}{2}}
\hline

\endfirsthead

\multicolumn{2}{c}%
{\makebox[0pt]{\sphinxtablecontinued{\tablename\ \thetable{} \textendash{} continued from previous page}}}\\
\hline

\endhead

\hline
\multicolumn{2}{r}{\makebox[0pt][r]{\sphinxtablecontinued{continues on next page}}}\\
\endfoot

\endlastfoot

\sphinxAtStartPar
{\hyperref[\detokenize{autoapi/halo_wrap/index:halo_wrap.halo}]{\sphinxcrossref{\sphinxcode{\sphinxupquote{halo}}}}}(self, model, workspace, state)
&
\sphinxAtStartPar

\\
\hline
\end{longtable}\sphinxatlongtableend\end{savenotes}
\index{halo() (in module halo\_wrap)@\spxentry{halo()}\spxextra{in module halo\_wrap}}

\begin{fulllineitems}
\phantomsection\label{\detokenize{autoapi/halo_wrap/index:halo_wrap.halo}}\pysiglinewithargsret{\sphinxcode{\sphinxupquote{halo\_wrap.}}\sphinxbfcode{\sphinxupquote{halo}}}{\emph{\DUrole{n}{self}}, \emph{\DUrole{n}{model}}, \emph{\DUrole{n}{workspace}}, \emph{\DUrole{n}{state}}}{}
\end{fulllineitems}



\section{\sphinxstyleliteralintitle{\sphinxupquote{nsflux\_wrap}}}
\label{\detokenize{autoapi/nsflux_wrap/index:module-nsflux_wrap}}\label{\detokenize{autoapi/nsflux_wrap/index:nsflux-wrap}}\label{\detokenize{autoapi/nsflux_wrap/index::doc}}\index{module@\spxentry{module}!nsflux\_wrap@\spxentry{nsflux\_wrap}}\index{nsflux\_wrap@\spxentry{nsflux\_wrap}!module@\spxentry{module}}

\subsection{Module Contents}
\label{\detokenize{autoapi/nsflux_wrap/index:module-contents}}

\subsubsection{Functions}
\label{\detokenize{autoapi/nsflux_wrap/index:functions}}

\begin{savenotes}\sphinxatlongtablestart\begin{longtable}[c]{\X{1}{2}\X{1}{2}}
\hline

\endfirsthead

\multicolumn{2}{c}%
{\makebox[0pt]{\sphinxtablecontinued{\tablename\ \thetable{} \textendash{} continued from previous page}}}\\
\hline

\endhead

\hline
\multicolumn{2}{r}{\makebox[0pt][r]{\sphinxtablecontinued{continues on next page}}}\\
\endfoot

\endlastfoot

\sphinxAtStartPar
{\hyperref[\detokenize{autoapi/nsflux_wrap/index:nsflux_wrap.nsflux}]{\sphinxcrossref{\sphinxcode{\sphinxupquote{nsflux}}}}}(self, ws, w, vw, rfil)
&
\sphinxAtStartPar

\\
\hline
\end{longtable}\sphinxatlongtableend\end{savenotes}
\index{nsflux() (in module nsflux\_wrap)@\spxentry{nsflux()}\spxextra{in module nsflux\_wrap}}

\begin{fulllineitems}
\phantomsection\label{\detokenize{autoapi/nsflux_wrap/index:nsflux_wrap.nsflux}}\pysiglinewithargsret{\sphinxcode{\sphinxupquote{nsflux\_wrap.}}\sphinxbfcode{\sphinxupquote{nsflux}}}{\emph{\DUrole{n}{self}}, \emph{\DUrole{n}{ws}}, \emph{\DUrole{n}{w}}, \emph{\DUrole{n}{vw}}, \emph{\DUrole{n}{rfil}}}{}
\end{fulllineitems}



\section{\sphinxstyleliteralintitle{\sphinxupquote{stability}}}
\label{\detokenize{autoapi/stability/index:module-stability}}\label{\detokenize{autoapi/stability/index:stability}}\label{\detokenize{autoapi/stability/index::doc}}\index{module@\spxentry{module}!stability@\spxentry{stability}}\index{stability@\spxentry{stability}!module@\spxentry{module}}

\subsection{Module Contents}
\label{\detokenize{autoapi/stability/index:module-contents}}

\subsubsection{Functions}
\label{\detokenize{autoapi/stability/index:functions}}

\begin{savenotes}\sphinxatlongtablestart\begin{longtable}[c]{\X{1}{2}\X{1}{2}}
\hline

\endfirsthead

\multicolumn{2}{c}%
{\makebox[0pt]{\sphinxtablecontinued{\tablename\ \thetable{} \textendash{} continued from previous page}}}\\
\hline

\endhead

\hline
\multicolumn{2}{r}{\makebox[0pt][r]{\sphinxtablecontinued{continues on next page}}}\\
\endfoot

\endlastfoot

\sphinxAtStartPar
{\hyperref[\detokenize{autoapi/stability/index:stability.stability}]{\sphinxcrossref{\sphinxcode{\sphinxupquote{stability}}}}}(self, model, workspace, state)
&
\sphinxAtStartPar
Calculates timestep limits to maintain stability
\\
\hline
\sphinxAtStartPar
{\hyperref[\detokenize{autoapi/stability/index:stability.edge}]{\sphinxcrossref{\sphinxcode{\sphinxupquote{edge}}}}}(workspace, i, j, side)
&
\sphinxAtStartPar

\\
\hline
\end{longtable}\sphinxatlongtableend\end{savenotes}
\index{stability() (in module stability)@\spxentry{stability()}\spxextra{in module stability}}

\begin{fulllineitems}
\phantomsection\label{\detokenize{autoapi/stability/index:stability.stability}}\pysiglinewithargsret{\sphinxcode{\sphinxupquote{stability.}}\sphinxbfcode{\sphinxupquote{stability}}}{\emph{\DUrole{n}{self}}, \emph{\DUrole{n}{model}}, \emph{\DUrole{n}{workspace}}, \emph{\DUrole{n}{state}}}{}
\sphinxAtStartPar
Calculates timestep limits to maintain stability
\begin{quote}\begin{description}
\item[{Parameters}] \leavevmode\begin{itemize}
\item {} 
\sphinxAtStartPar
\sphinxstyleliteralstrong{\sphinxupquote{model}} ({\hyperref[\detokenize{autoapi/Model/index:Model.Model}]{\sphinxcrossref{\sphinxstyleliteralemphasis{\sphinxupquote{Model}}}}}) \textendash{} The physics model

\item {} 
\sphinxAtStartPar
\sphinxstyleliteralstrong{\sphinxupquote{workspace}} ({\hyperref[\detokenize{autoapi/Workspace/index:Workspace.Workspace}]{\sphinxcrossref{\sphinxstyleliteralemphasis{\sphinxupquote{Workspace}}}}}) \textendash{} The current Workspace

\item {} 
\sphinxAtStartPar
\sphinxstyleliteralstrong{\sphinxupquote{state}} ({\hyperref[\detokenize{autoapi/Field/index:Field.Field}]{\sphinxcrossref{\sphinxstyleliteralemphasis{\sphinxupquote{Field}}}}}) \textendash{} Field containing current state

\end{itemize}

\end{description}\end{quote}

\end{fulllineitems}

\index{edge() (in module stability)@\spxentry{edge()}\spxextra{in module stability}}

\begin{fulllineitems}
\phantomsection\label{\detokenize{autoapi/stability/index:stability.edge}}\pysiglinewithargsret{\sphinxcode{\sphinxupquote{stability.}}\sphinxbfcode{\sphinxupquote{edge}}}{\emph{\DUrole{n}{workspace}}, \emph{\DUrole{n}{i}}, \emph{\DUrole{n}{j}}, \emph{\DUrole{n}{side}}}{}
\end{fulllineitems}



\section{\sphinxstyleliteralintitle{\sphinxupquote{airfoil\_map}}}
\label{\detokenize{autoapi/airfoil_map/index:module-airfoil_map}}\label{\detokenize{autoapi/airfoil_map/index:airfoil-map}}\label{\detokenize{autoapi/airfoil_map/index::doc}}\index{module@\spxentry{module}!airfoil\_map@\spxentry{airfoil\_map}}\index{airfoil\_map@\spxentry{airfoil\_map}!module@\spxentry{module}}
\sphinxAtStartPar
This module has functions that perform the conformal mapping and the coarser grid objects
\begin{description}
\item[{Libraries/Modules:}] \leavevmode
\sphinxAtStartPar
numpy

\sphinxAtStartPar
Field

\sphinxAtStartPar
Contractinator

\sphinxAtStartPar
dims\_funs

\sphinxAtStartPar
coord\_strch\_func

\sphinxAtStartPar
geom\_func

\sphinxAtStartPar
sangho\_func

\sphinxAtStartPar
metric\_func

\sphinxAtStartPar
plot\_mesh\_func

\end{description}


\subsection{Module Contents}
\label{\detokenize{autoapi/airfoil_map/index:module-contents}}

\subsubsection{Functions}
\label{\detokenize{autoapi/airfoil_map/index:functions}}

\begin{savenotes}\sphinxatlongtablestart\begin{longtable}[c]{\X{1}{2}\X{1}{2}}
\hline

\endfirsthead

\multicolumn{2}{c}%
{\makebox[0pt]{\sphinxtablecontinued{\tablename\ \thetable{} \textendash{} continued from previous page}}}\\
\hline

\endhead

\hline
\multicolumn{2}{r}{\makebox[0pt][r]{\sphinxtablecontinued{continues on next page}}}\\
\endfoot

\endlastfoot

\sphinxAtStartPar
{\hyperref[\detokenize{autoapi/airfoil_map/index:airfoil_map.init_from_file}]{\sphinxcrossref{\sphinxcode{\sphinxupquote{init\_from\_file}}}}}(self, grid\_dim, input)
&
\sphinxAtStartPar
Performs conformal mapping and finds x,xc and vol values in physical space.
\\
\hline
\sphinxAtStartPar
{\hyperref[\detokenize{autoapi/airfoil_map/index:airfoil_map.init_from_grid}]{\sphinxcrossref{\sphinxcode{\sphinxupquote{init\_from\_grid}}}}}(newGrid, grid)
&
\sphinxAtStartPar
Makes a grid coarser and finds new x,xc and vol values on the coarse grid
\\
\hline
\end{longtable}\sphinxatlongtableend\end{savenotes}
\index{init\_from\_file() (in module airfoil\_map)@\spxentry{init\_from\_file()}\spxextra{in module airfoil\_map}}

\begin{fulllineitems}
\phantomsection\label{\detokenize{autoapi/airfoil_map/index:airfoil_map.init_from_file}}\pysiglinewithargsret{\sphinxcode{\sphinxupquote{airfoil\_map.}}\sphinxbfcode{\sphinxupquote{init\_from\_file}}}{\emph{\DUrole{n}{self}}, \emph{\DUrole{n}{grid\_dim}}, \emph{\DUrole{n}{input}}}{}
\sphinxAtStartPar
Performs conformal mapping and finds x,xc and vol values in physical space.

\sphinxAtStartPar
Also plots the c\sphinxhyphen{}mesh/grid in phyiscal space
\begin{quote}
\begin{description}
\item[{Args:}] \leavevmode
\sphinxAtStartPar
grid\_dim (list):Number of cells in the x and y directions.
input (dict):Dictionary containing data\sphinxhyphen{}file values

\end{description}
\end{quote}

\end{fulllineitems}

\index{init\_from\_grid() (in module airfoil\_map)@\spxentry{init\_from\_grid()}\spxextra{in module airfoil\_map}}

\begin{fulllineitems}
\phantomsection\label{\detokenize{autoapi/airfoil_map/index:airfoil_map.init_from_grid}}\pysiglinewithargsret{\sphinxcode{\sphinxupquote{airfoil\_map.}}\sphinxbfcode{\sphinxupquote{init\_from\_grid}}}{\emph{\DUrole{n}{newGrid}}, \emph{\DUrole{n}{grid}}}{}
\sphinxAtStartPar
Makes a grid coarser and finds new x,xc and vol values on the coarse grid

\sphinxAtStartPar
Also plots the coarser grid in physical space
\begin{quote}
\begin{description}
\item[{Args:}] \leavevmode
\sphinxAtStartPar
grid (obj):finer input AirfoilMap object
newGrid (obj):coarser output new AirfoilMap object

\end{description}
\end{quote}

\end{fulllineitems}



\section{\sphinxstyleliteralintitle{\sphinxupquote{coord\_strch\_func}}}
\label{\detokenize{autoapi/coord_strch_func/index:module-coord_strch_func}}\label{\detokenize{autoapi/coord_strch_func/index:coord-strch-func}}\label{\detokenize{autoapi/coord_strch_func/index::doc}}\index{module@\spxentry{module}!coord\_strch\_func@\spxentry{coord\_strch\_func}}\index{coord\_strch\_func@\spxentry{coord\_strch\_func}!module@\spxentry{module}}
\sphinxAtStartPar
This module creates x and y array in computational space.
\begin{description}
\item[{Libraries/Modules:}] \leavevmode
\sphinxAtStartPar
numpy

\end{description}


\subsection{Module Contents}
\label{\detokenize{autoapi/coord_strch_func/index:module-contents}}

\subsubsection{Functions}
\label{\detokenize{autoapi/coord_strch_func/index:functions}}

\begin{savenotes}\sphinxatlongtablestart\begin{longtable}[c]{\X{1}{2}\X{1}{2}}
\hline

\endfirsthead

\multicolumn{2}{c}%
{\makebox[0pt]{\sphinxtablecontinued{\tablename\ \thetable{} \textendash{} continued from previous page}}}\\
\hline

\endhead

\hline
\multicolumn{2}{r}{\makebox[0pt][r]{\sphinxtablecontinued{continues on next page}}}\\
\endfoot

\endlastfoot

\sphinxAtStartPar
{\hyperref[\detokenize{autoapi/coord_strch_func/index:coord_strch_func.coord_stretch}]{\sphinxcrossref{\sphinxcode{\sphinxupquote{coord\_stretch}}}}}(self)
&
\sphinxAtStartPar
It create array a0 and b0 in x and y dirn computationally with spacing such that when it is mapped back to physica domain it points on the mesh are evenly spaced for a given i/j direction.
\\
\hline
\end{longtable}\sphinxatlongtableend\end{savenotes}
\index{coord\_stretch() (in module coord\_strch\_func)@\spxentry{coord\_stretch()}\spxextra{in module coord\_strch\_func}}

\begin{fulllineitems}
\phantomsection\label{\detokenize{autoapi/coord_strch_func/index:coord_strch_func.coord_stretch}}\pysiglinewithargsret{\sphinxcode{\sphinxupquote{coord\_strch\_func.}}\sphinxbfcode{\sphinxupquote{coord\_stretch}}}{\emph{\DUrole{n}{self}}}{}
\sphinxAtStartPar
It create array a0 and b0 in x and y dirn computationally with spacing such that when it is mapped back to physica domain it points on the mesh are evenly spaced for a given i/j direction.

\end{fulllineitems}



\section{\sphinxstyleliteralintitle{\sphinxupquote{dims\_func}}}
\label{\detokenize{autoapi/dims_func/index:module-dims_func}}\label{\detokenize{autoapi/dims_func/index:dims-func}}\label{\detokenize{autoapi/dims_func/index::doc}}\index{module@\spxentry{module}!dims\_func@\spxentry{dims\_func}}\index{dims\_func@\spxentry{dims\_func}!module@\spxentry{module}}
\sphinxAtStartPar
This module unpacks the dims dict and sets up other dimesnsions and airfoil constraints
\begin{description}
\item[{Libraries/Modules:}] \leavevmode
\sphinxAtStartPar
numpy

\end{description}


\subsection{Module Contents}
\label{\detokenize{autoapi/dims_func/index:module-contents}}

\subsubsection{Functions}
\label{\detokenize{autoapi/dims_func/index:functions}}

\begin{savenotes}\sphinxatlongtablestart\begin{longtable}[c]{\X{1}{2}\X{1}{2}}
\hline

\endfirsthead

\multicolumn{2}{c}%
{\makebox[0pt]{\sphinxtablecontinued{\tablename\ \thetable{} \textendash{} continued from previous page}}}\\
\hline

\endhead

\hline
\multicolumn{2}{r}{\makebox[0pt][r]{\sphinxtablecontinued{continues on next page}}}\\
\endfoot

\endlastfoot

\sphinxAtStartPar
{\hyperref[\detokenize{autoapi/dims_func/index:dims_func.set_dims}]{\sphinxcrossref{\sphinxcode{\sphinxupquote{set\_dims}}}}}(self)
&
\sphinxAtStartPar
Sets mesh dimensions,limits of the airfoil and updates dims dict
\\
\hline
\end{longtable}\sphinxatlongtableend\end{savenotes}
\index{set\_dims() (in module dims\_func)@\spxentry{set\_dims()}\spxextra{in module dims\_func}}

\begin{fulllineitems}
\phantomsection\label{\detokenize{autoapi/dims_func/index:dims_func.set_dims}}\pysiglinewithargsret{\sphinxcode{\sphinxupquote{dims\_func.}}\sphinxbfcode{\sphinxupquote{set\_dims}}}{\emph{\DUrole{n}{self}}}{}
\sphinxAtStartPar
Sets mesh dimensions,limits of the airfoil and updates dims dict

\end{fulllineitems}



\section{\sphinxstyleliteralintitle{\sphinxupquote{geom\_func}}}
\label{\detokenize{autoapi/geom_func/index:module-geom_func}}\label{\detokenize{autoapi/geom_func/index:geom-func}}\label{\detokenize{autoapi/geom_func/index::doc}}\index{module@\spxentry{module}!geom\_func@\spxentry{geom\_func}}\index{geom\_func@\spxentry{geom\_func}!module@\spxentry{module}}
\sphinxAtStartPar
This module maps physical airfoil geometry to computational space
\begin{description}
\item[{Libraries/Modules:}] \leavevmode
\sphinxAtStartPar
numpy

\sphinxAtStartPar
scipy.interpolate

\end{description}


\subsection{Module Contents}
\label{\detokenize{autoapi/geom_func/index:module-contents}}

\subsubsection{Functions}
\label{\detokenize{autoapi/geom_func/index:functions}}

\begin{savenotes}\sphinxatlongtablestart\begin{longtable}[c]{\X{1}{2}\X{1}{2}}
\hline

\endfirsthead

\multicolumn{2}{c}%
{\makebox[0pt]{\sphinxtablecontinued{\tablename\ \thetable{} \textendash{} continued from previous page}}}\\
\hline

\endhead

\hline
\multicolumn{2}{r}{\makebox[0pt][r]{\sphinxtablecontinued{continues on next page}}}\\
\endfoot

\endlastfoot

\sphinxAtStartPar
{\hyperref[\detokenize{autoapi/geom_func/index:geom_func.geom}]{\sphinxcrossref{\sphinxcode{\sphinxupquote{geom}}}}}(self)
&
\sphinxAtStartPar
This function performs conformal mapping to computational domain
\\
\hline
\end{longtable}\sphinxatlongtableend\end{savenotes}
\index{geom() (in module geom\_func)@\spxentry{geom()}\spxextra{in module geom\_func}}

\begin{fulllineitems}
\phantomsection\label{\detokenize{autoapi/geom_func/index:geom_func.geom}}\pysiglinewithargsret{\sphinxcode{\sphinxupquote{geom\_func.}}\sphinxbfcode{\sphinxupquote{geom}}}{\emph{\DUrole{n}{self}}}{}
\sphinxAtStartPar
This function performs conformal mapping to computational domain

\sphinxAtStartPar
It maps the xn and yn physical coordinates of the airfoil to xs and ys
coordinates in the computational domain

\end{fulllineitems}



\section{\sphinxstyleliteralintitle{\sphinxupquote{mesh\_func}}}
\label{\detokenize{autoapi/mesh_func/index:module-mesh_func}}\label{\detokenize{autoapi/mesh_func/index:mesh-func}}\label{\detokenize{autoapi/mesh_func/index::doc}}\index{module@\spxentry{module}!mesh\_func@\spxentry{mesh\_func}}\index{mesh\_func@\spxentry{mesh\_func}!module@\spxentry{module}}
\sphinxAtStartPar
This module creates the c\sphinxhyphen{}mesh in physical space after conformal mapping
\begin{description}
\item[{Libraries/Modules:}] \leavevmode
\sphinxAtStartPar
numpy

\sphinxAtStartPar
Field

\end{description}


\subsection{Module Contents}
\label{\detokenize{autoapi/mesh_func/index:module-contents}}

\subsubsection{Functions}
\label{\detokenize{autoapi/mesh_func/index:functions}}

\begin{savenotes}\sphinxatlongtablestart\begin{longtable}[c]{\X{1}{2}\X{1}{2}}
\hline

\endfirsthead

\multicolumn{2}{c}%
{\makebox[0pt]{\sphinxtablecontinued{\tablename\ \thetable{} \textendash{} continued from previous page}}}\\
\hline

\endhead

\hline
\multicolumn{2}{r}{\makebox[0pt][r]{\sphinxtablecontinued{continues on next page}}}\\
\endfoot

\endlastfoot

\sphinxAtStartPar
{\hyperref[\detokenize{autoapi/mesh_func/index:mesh_func.mesh}]{\sphinxcrossref{\sphinxcode{\sphinxupquote{mesh}}}}}(self)
&
\sphinxAtStartPar
This function maps back to physical space to create the c\sphinxhyphen{}mesh
\\
\hline
\end{longtable}\sphinxatlongtableend\end{savenotes}
\index{mesh() (in module mesh\_func)@\spxentry{mesh()}\spxextra{in module mesh\_func}}

\begin{fulllineitems}
\phantomsection\label{\detokenize{autoapi/mesh_func/index:mesh_func.mesh}}\pysiglinewithargsret{\sphinxcode{\sphinxupquote{mesh\_func.}}\sphinxbfcode{\sphinxupquote{mesh}}}{\emph{\DUrole{n}{self}}}{}
\sphinxAtStartPar
This function maps back to physical space to create the c\sphinxhyphen{}mesh

\sphinxAtStartPar
First a cubic spline interpolation is performed to make sure that points on airfoil geometry line up with a0 points

\sphinxAtStartPar
Then a0,b0,xs and ys are mapped to an s0 array and this is used to creat the x
array which contains the vertices of the right hand side corners of all cells
in the physical domain.

\end{fulllineitems}



\section{\sphinxstyleliteralintitle{\sphinxupquote{metric\_func}}}
\label{\detokenize{autoapi/metric_func/index:module-metric_func}}\label{\detokenize{autoapi/metric_func/index:metric-func}}\label{\detokenize{autoapi/metric_func/index::doc}}\index{module@\spxentry{module}!metric\_func@\spxentry{metric\_func}}\index{metric\_func@\spxentry{metric\_func}!module@\spxentry{module}}
\sphinxAtStartPar
This module calculates cell centers xc and cell volumes vol in physical space
\begin{description}
\item[{Libraries/Modules:}] \leavevmode
\sphinxAtStartPar
Field

\end{description}


\subsection{Module Contents}
\label{\detokenize{autoapi/metric_func/index:module-contents}}

\subsubsection{Functions}
\label{\detokenize{autoapi/metric_func/index:functions}}

\begin{savenotes}\sphinxatlongtablestart\begin{longtable}[c]{\X{1}{2}\X{1}{2}}
\hline

\endfirsthead

\multicolumn{2}{c}%
{\makebox[0pt]{\sphinxtablecontinued{\tablename\ \thetable{} \textendash{} continued from previous page}}}\\
\hline

\endhead

\hline
\multicolumn{2}{r}{\makebox[0pt][r]{\sphinxtablecontinued{continues on next page}}}\\
\endfoot

\endlastfoot

\sphinxAtStartPar
{\hyperref[\detokenize{autoapi/metric_func/index:metric_func.metric}]{\sphinxcrossref{\sphinxcode{\sphinxupquote{metric}}}}}(self)
&
\sphinxAtStartPar
This function calculates cell centers and cell volume
\\
\hline
\end{longtable}\sphinxatlongtableend\end{savenotes}
\index{metric() (in module metric\_func)@\spxentry{metric()}\spxextra{in module metric\_func}}

\begin{fulllineitems}
\phantomsection\label{\detokenize{autoapi/metric_func/index:metric_func.metric}}\pysiglinewithargsret{\sphinxcode{\sphinxupquote{metric\_func.}}\sphinxbfcode{\sphinxupquote{metric}}}{\emph{\DUrole{n}{self}}}{}
\sphinxAtStartPar
This function calculates cell centers and cell volume

\sphinxAtStartPar
It uses the vertices from x to calculate xc (centers) and vol(volme)
of cells in physical space.

\end{fulllineitems}



\section{\sphinxstyleliteralintitle{\sphinxupquote{plot\_mesh\_func}}}
\label{\detokenize{autoapi/plot_mesh_func/index:module-plot_mesh_func}}\label{\detokenize{autoapi/plot_mesh_func/index:plot-mesh-func}}\label{\detokenize{autoapi/plot_mesh_func/index::doc}}\index{module@\spxentry{module}!plot\_mesh\_func@\spxentry{plot\_mesh\_func}}\index{plot\_mesh\_func@\spxentry{plot\_mesh\_func}!module@\spxentry{module}}
\sphinxAtStartPar
This module plots the c\sphinxhyphen{}mesh in physical space
\begin{description}
\item[{Libraries/Modules:}] \leavevmode
\sphinxAtStartPar
nump

\sphinxAtStartPar
matplotlib.pyplot

\end{description}


\subsection{Module Contents}
\label{\detokenize{autoapi/plot_mesh_func/index:module-contents}}

\subsubsection{Functions}
\label{\detokenize{autoapi/plot_mesh_func/index:functions}}

\begin{savenotes}\sphinxatlongtablestart\begin{longtable}[c]{\X{1}{2}\X{1}{2}}
\hline

\endfirsthead

\multicolumn{2}{c}%
{\makebox[0pt]{\sphinxtablecontinued{\tablename\ \thetable{} \textendash{} continued from previous page}}}\\
\hline

\endhead

\hline
\multicolumn{2}{r}{\makebox[0pt][r]{\sphinxtablecontinued{continues on next page}}}\\
\endfoot

\endlastfoot

\sphinxAtStartPar
{\hyperref[\detokenize{autoapi/plot_mesh_func/index:plot_mesh_func.plot_mesh}]{\sphinxcrossref{\sphinxcode{\sphinxupquote{plot\_mesh}}}}}(self)
&
\sphinxAtStartPar
Plots the c\sphinxhyphen{}mesh in physical space after conformal mapping
\\
\hline
\end{longtable}\sphinxatlongtableend\end{savenotes}
\index{plot\_mesh() (in module plot\_mesh\_func)@\spxentry{plot\_mesh()}\spxextra{in module plot\_mesh\_func}}

\begin{fulllineitems}
\phantomsection\label{\detokenize{autoapi/plot_mesh_func/index:plot_mesh_func.plot_mesh}}\pysiglinewithargsret{\sphinxcode{\sphinxupquote{plot\_mesh\_func.}}\sphinxbfcode{\sphinxupquote{plot\_mesh}}}{\emph{\DUrole{n}{self}}}{}
\sphinxAtStartPar
Plots the c\sphinxhyphen{}mesh in physical space after conformal mapping

\sphinxAtStartPar
Uses the vertices from x to plot the mesh.

\end{fulllineitems}



\section{\sphinxstyleliteralintitle{\sphinxupquote{sangho\_func}}}
\label{\detokenize{autoapi/sangho_func/index:module-sangho_func}}\label{\detokenize{autoapi/sangho_func/index:sangho-func}}\label{\detokenize{autoapi/sangho_func/index::doc}}\index{module@\spxentry{module}!sangho\_func@\spxentry{sangho\_func}}\index{sangho\_func@\spxentry{sangho\_func}!module@\spxentry{module}}
\sphinxAtStartPar
This module scales the values of the vertices
\begin{description}
\item[{Libraries/Modules:}] \leavevmode
\sphinxAtStartPar
numpy

\end{description}


\subsection{Module Contents}
\label{\detokenize{autoapi/sangho_func/index:module-contents}}

\subsubsection{Functions}
\label{\detokenize{autoapi/sangho_func/index:functions}}

\begin{savenotes}\sphinxatlongtablestart\begin{longtable}[c]{\X{1}{2}\X{1}{2}}
\hline

\endfirsthead

\multicolumn{2}{c}%
{\makebox[0pt]{\sphinxtablecontinued{\tablename\ \thetable{} \textendash{} continued from previous page}}}\\
\hline

\endhead

\hline
\multicolumn{2}{r}{\makebox[0pt][r]{\sphinxtablecontinued{continues on next page}}}\\
\endfoot

\endlastfoot

\sphinxAtStartPar
{\hyperref[\detokenize{autoapi/sangho_func/index:sangho_func.sangho}]{\sphinxcrossref{\sphinxcode{\sphinxupquote{sangho}}}}}(self)
&
\sphinxAtStartPar
Performs sanghos non\sphinxhyphen{}dimensionalisation method
\\
\hline
\end{longtable}\sphinxatlongtableend\end{savenotes}
\index{sangho() (in module sangho\_func)@\spxentry{sangho()}\spxextra{in module sangho\_func}}

\begin{fulllineitems}
\phantomsection\label{\detokenize{autoapi/sangho_func/index:sangho_func.sangho}}\pysiglinewithargsret{\sphinxcode{\sphinxupquote{sangho\_func.}}\sphinxbfcode{\sphinxupquote{sangho}}}{\emph{\DUrole{n}{self}}}{}
\sphinxAtStartPar
Performs sanghos non\sphinxhyphen{}dimensionalisation method

\sphinxAtStartPar
Scales the vertices in x to wrap around the airfoil in physical space

\end{fulllineitems}



\section{\sphinxstyleliteralintitle{\sphinxupquote{tests}}}
\label{\detokenize{autoapi/tests/index:module-tests}}\label{\detokenize{autoapi/tests/index:tests}}\label{\detokenize{autoapi/tests/index::doc}}\index{module@\spxentry{module}!tests@\spxentry{tests}}\index{tests@\spxentry{tests}!module@\spxentry{module}}

\subsection{Submodules}
\label{\detokenize{autoapi/tests/index:submodules}}

\subsubsection{\sphinxstyleliteralintitle{\sphinxupquote{tests.test\_AirfoilMap}}}
\label{\detokenize{autoapi/tests/test_AirfoilMap/index:module-tests.test_AirfoilMap}}\label{\detokenize{autoapi/tests/test_AirfoilMap/index:tests-test-airfoilmap}}\label{\detokenize{autoapi/tests/test_AirfoilMap/index::doc}}\index{module@\spxentry{module}!tests.test\_AirfoilMap@\spxentry{tests.test\_AirfoilMap}}\index{tests.test\_AirfoilMap@\spxentry{tests.test\_AirfoilMap}!module@\spxentry{module}}
\sphinxAtStartPar
Tests the AirfoilMap object to make sure all the outputs are as expected
\begin{description}
\item[{Libraries/Modules:}] \leavevmode
\sphinxAtStartPar
pytest

\sphinxAtStartPar
AerfoilMap

\end{description}
\subsubsection*{Notes}

\sphinxAtStartPar
Runs the following tests:
\begin{enumerate}
\sphinxsetlistlabels{\arabic}{enumi}{enumii}{}{.}%
\item {} 
\sphinxAtStartPar
Checks that x,xc and vol are Field objects

\item {} 
\sphinxAtStartPar
Checks that vol is positive values

\end{enumerate}


\paragraph{Module Contents}
\label{\detokenize{autoapi/tests/test_AirfoilMap/index:module-contents}}

\subparagraph{Functions}
\label{\detokenize{autoapi/tests/test_AirfoilMap/index:functions}}

\begin{savenotes}\sphinxatlongtablestart\begin{longtable}[c]{\X{1}{2}\X{1}{2}}
\hline

\endfirsthead

\multicolumn{2}{c}%
{\makebox[0pt]{\sphinxtablecontinued{\tablename\ \thetable{} \textendash{} continued from previous page}}}\\
\hline

\endhead

\hline
\multicolumn{2}{r}{\makebox[0pt][r]{\sphinxtablecontinued{continues on next page}}}\\
\endfoot

\endlastfoot

\sphinxAtStartPar
{\hyperref[\detokenize{autoapi/tests/test_AirfoilMap/index:tests.test_AirfoilMap.test_if_field_object}]{\sphinxcrossref{\sphinxcode{\sphinxupquote{test\_if\_field\_object}}}}}()
&
\sphinxAtStartPar
Assert that x,xc and vol are Field
\\
\hline
\sphinxAtStartPar
{\hyperref[\detokenize{autoapi/tests/test_AirfoilMap/index:tests.test_AirfoilMap.test_vol_postive}]{\sphinxcrossref{\sphinxcode{\sphinxupquote{test\_vol\_postive}}}}}()
&
\sphinxAtStartPar
Assert that vol is all non\sphinxhyphen{}negative values
\\
\hline
\end{longtable}\sphinxatlongtableend\end{savenotes}


\subparagraph{Attributes}
\label{\detokenize{autoapi/tests/test_AirfoilMap/index:attributes}}

\begin{savenotes}\sphinxatlongtablestart\begin{longtable}[c]{\X{1}{2}\X{1}{2}}
\hline

\endfirsthead

\multicolumn{2}{c}%
{\makebox[0pt]{\sphinxtablecontinued{\tablename\ \thetable{} \textendash{} continued from previous page}}}\\
\hline

\endhead

\hline
\multicolumn{2}{r}{\makebox[0pt][r]{\sphinxtablecontinued{continues on next page}}}\\
\endfoot

\endlastfoot

\sphinxAtStartPar
{\hyperref[\detokenize{autoapi/tests/test_AirfoilMap/index:tests.test_AirfoilMap.input}]{\sphinxcrossref{\sphinxcode{\sphinxupquote{input}}}}}
&
\sphinxAtStartPar

\\
\hline
\sphinxAtStartPar
{\hyperref[\detokenize{autoapi/tests/test_AirfoilMap/index:tests.test_AirfoilMap.gridInput}]{\sphinxcrossref{\sphinxcode{\sphinxupquote{gridInput}}}}}
&
\sphinxAtStartPar

\\
\hline
\sphinxAtStartPar
{\hyperref[\detokenize{autoapi/tests/test_AirfoilMap/index:tests.test_AirfoilMap.grid_dim}]{\sphinxcrossref{\sphinxcode{\sphinxupquote{grid\_dim}}}}}
&
\sphinxAtStartPar

\\
\hline
\sphinxAtStartPar
{\hyperref[\detokenize{autoapi/tests/test_AirfoilMap/index:tests.test_AirfoilMap.grid}]{\sphinxcrossref{\sphinxcode{\sphinxupquote{grid}}}}}
&
\sphinxAtStartPar

\\
\hline
\sphinxAtStartPar
{\hyperref[\detokenize{autoapi/tests/test_AirfoilMap/index:tests.test_AirfoilMap.x}]{\sphinxcrossref{\sphinxcode{\sphinxupquote{x}}}}}
&
\sphinxAtStartPar

\\
\hline
\sphinxAtStartPar
{\hyperref[\detokenize{autoapi/tests/test_AirfoilMap/index:tests.test_AirfoilMap.xc}]{\sphinxcrossref{\sphinxcode{\sphinxupquote{xc}}}}}
&
\sphinxAtStartPar

\\
\hline
\sphinxAtStartPar
{\hyperref[\detokenize{autoapi/tests/test_AirfoilMap/index:tests.test_AirfoilMap.vol}]{\sphinxcrossref{\sphinxcode{\sphinxupquote{vol}}}}}
&
\sphinxAtStartPar

\\
\hline
\end{longtable}\sphinxatlongtableend\end{savenotes}
\index{input (in module tests.test\_AirfoilMap)@\spxentry{input}\spxextra{in module tests.test\_AirfoilMap}}

\begin{fulllineitems}
\phantomsection\label{\detokenize{autoapi/tests/test_AirfoilMap/index:tests.test_AirfoilMap.input}}\pysigline{\sphinxcode{\sphinxupquote{tests.test\_AirfoilMap.}}\sphinxbfcode{\sphinxupquote{input}}}
\end{fulllineitems}

\index{gridInput (in module tests.test\_AirfoilMap)@\spxentry{gridInput}\spxextra{in module tests.test\_AirfoilMap}}

\begin{fulllineitems}
\phantomsection\label{\detokenize{autoapi/tests/test_AirfoilMap/index:tests.test_AirfoilMap.gridInput}}\pysigline{\sphinxcode{\sphinxupquote{tests.test\_AirfoilMap.}}\sphinxbfcode{\sphinxupquote{gridInput}}}
\end{fulllineitems}

\index{grid\_dim (in module tests.test\_AirfoilMap)@\spxentry{grid\_dim}\spxextra{in module tests.test\_AirfoilMap}}

\begin{fulllineitems}
\phantomsection\label{\detokenize{autoapi/tests/test_AirfoilMap/index:tests.test_AirfoilMap.grid_dim}}\pysigline{\sphinxcode{\sphinxupquote{tests.test\_AirfoilMap.}}\sphinxbfcode{\sphinxupquote{grid\_dim}}}
\end{fulllineitems}

\index{grid (in module tests.test\_AirfoilMap)@\spxentry{grid}\spxextra{in module tests.test\_AirfoilMap}}

\begin{fulllineitems}
\phantomsection\label{\detokenize{autoapi/tests/test_AirfoilMap/index:tests.test_AirfoilMap.grid}}\pysigline{\sphinxcode{\sphinxupquote{tests.test\_AirfoilMap.}}\sphinxbfcode{\sphinxupquote{grid}}}
\end{fulllineitems}

\index{x (in module tests.test\_AirfoilMap)@\spxentry{x}\spxextra{in module tests.test\_AirfoilMap}}

\begin{fulllineitems}
\phantomsection\label{\detokenize{autoapi/tests/test_AirfoilMap/index:tests.test_AirfoilMap.x}}\pysigline{\sphinxcode{\sphinxupquote{tests.test\_AirfoilMap.}}\sphinxbfcode{\sphinxupquote{x}}}
\end{fulllineitems}

\index{xc (in module tests.test\_AirfoilMap)@\spxentry{xc}\spxextra{in module tests.test\_AirfoilMap}}

\begin{fulllineitems}
\phantomsection\label{\detokenize{autoapi/tests/test_AirfoilMap/index:tests.test_AirfoilMap.xc}}\pysigline{\sphinxcode{\sphinxupquote{tests.test\_AirfoilMap.}}\sphinxbfcode{\sphinxupquote{xc}}}
\end{fulllineitems}

\index{vol (in module tests.test\_AirfoilMap)@\spxentry{vol}\spxextra{in module tests.test\_AirfoilMap}}

\begin{fulllineitems}
\phantomsection\label{\detokenize{autoapi/tests/test_AirfoilMap/index:tests.test_AirfoilMap.vol}}\pysigline{\sphinxcode{\sphinxupquote{tests.test\_AirfoilMap.}}\sphinxbfcode{\sphinxupquote{vol}}}
\end{fulllineitems}

\index{test\_if\_field\_object() (in module tests.test\_AirfoilMap)@\spxentry{test\_if\_field\_object()}\spxextra{in module tests.test\_AirfoilMap}}

\begin{fulllineitems}
\phantomsection\label{\detokenize{autoapi/tests/test_AirfoilMap/index:tests.test_AirfoilMap.test_if_field_object}}\pysiglinewithargsret{\sphinxcode{\sphinxupquote{tests.test\_AirfoilMap.}}\sphinxbfcode{\sphinxupquote{test\_if\_field\_object}}}{}{}
\sphinxAtStartPar
Assert that x,xc and vol are Field
objects

\end{fulllineitems}

\index{test\_vol\_postive() (in module tests.test\_AirfoilMap)@\spxentry{test\_vol\_postive()}\spxextra{in module tests.test\_AirfoilMap}}

\begin{fulllineitems}
\phantomsection\label{\detokenize{autoapi/tests/test_AirfoilMap/index:tests.test_AirfoilMap.test_vol_postive}}\pysiglinewithargsret{\sphinxcode{\sphinxupquote{tests.test\_AirfoilMap.}}\sphinxbfcode{\sphinxupquote{test\_vol\_postive}}}{}{}
\sphinxAtStartPar
Assert that vol is all non\sphinxhyphen{}negative values

\end{fulllineitems}



\subsubsection{\sphinxstyleliteralintitle{\sphinxupquote{tests.test\_Contractinator}}}
\label{\detokenize{autoapi/tests/test_Contractinator/index:module-tests.test_Contractinator}}\label{\detokenize{autoapi/tests/test_Contractinator/index:tests-test-contractinator}}\label{\detokenize{autoapi/tests/test_Contractinator/index::doc}}\index{module@\spxentry{module}!tests.test\_Contractinator@\spxentry{tests.test\_Contractinator}}\index{tests.test\_Contractinator@\spxentry{tests.test\_Contractinator}!module@\spxentry{module}}
\sphinxAtStartPar
Description

\sphinxAtStartPar
Tests the contractinator.

\sphinxAtStartPar
Libraries/Modules

\sphinxAtStartPar
\sphinxhyphen{}pytest

\sphinxAtStartPar
\sphinxhyphen{}Field

\sphinxAtStartPar
\sphinxhyphen{}numpy

\sphinxAtStartPar
Notes


\paragraph{Module Contents}
\label{\detokenize{autoapi/tests/test_Contractinator/index:module-contents}}

\subparagraph{Functions}
\label{\detokenize{autoapi/tests/test_Contractinator/index:functions}}

\begin{savenotes}\sphinxatlongtablestart\begin{longtable}[c]{\X{1}{2}\X{1}{2}}
\hline

\endfirsthead

\multicolumn{2}{c}%
{\makebox[0pt]{\sphinxtablecontinued{\tablename\ \thetable{} \textendash{} continued from previous page}}}\\
\hline

\endhead

\hline
\multicolumn{2}{r}{\makebox[0pt][r]{\sphinxtablecontinued{continues on next page}}}\\
\endfoot

\endlastfoot

\sphinxAtStartPar
{\hyperref[\detokenize{autoapi/tests/test_Contractinator/index:tests.test_Contractinator.test_simple}]{\sphinxcrossref{\sphinxcode{\sphinxupquote{test\_simple}}}}}()
&
\sphinxAtStartPar
Tests the Contractinator.py ‘simple’ method.
\\
\hline
\sphinxAtStartPar
{\hyperref[\detokenize{autoapi/tests/test_Contractinator/index:tests.test_Contractinator.test_sum4way}]{\sphinxcrossref{\sphinxcode{\sphinxupquote{test\_sum4way}}}}}()
&
\sphinxAtStartPar
Tests the Contractinator.py ‘sum4way’ method.
\\
\hline
\sphinxAtStartPar
{\hyperref[\detokenize{autoapi/tests/test_Contractinator/index:tests.test_Contractinator.test_conservative4way}]{\sphinxcrossref{\sphinxcode{\sphinxupquote{test\_conservative4way}}}}}()
&
\sphinxAtStartPar
Tests the Contractinator.py ‘conservative4way’ method. Note: does not test weighted averaging.
\\
\hline
\end{longtable}\sphinxatlongtableend\end{savenotes}
\index{test\_simple() (in module tests.test\_Contractinator)@\spxentry{test\_simple()}\spxextra{in module tests.test\_Contractinator}}

\begin{fulllineitems}
\phantomsection\label{\detokenize{autoapi/tests/test_Contractinator/index:tests.test_Contractinator.test_simple}}\pysiglinewithargsret{\sphinxcode{\sphinxupquote{tests.test\_Contractinator.}}\sphinxbfcode{\sphinxupquote{test\_simple}}}{}{}
\sphinxAtStartPar
Tests the Contractinator.py ‘simple’ method.

\sphinxAtStartPar
Args:

\sphinxAtStartPar
Returns
\begin{description}
\item[{:}] \leavevmode
\sphinxAtStartPar
Nothing, but asserts if ‘simple’ deletes items from the Field as it should.

\end{description}

\end{fulllineitems}

\index{test\_sum4way() (in module tests.test\_Contractinator)@\spxentry{test\_sum4way()}\spxextra{in module tests.test\_Contractinator}}

\begin{fulllineitems}
\phantomsection\label{\detokenize{autoapi/tests/test_Contractinator/index:tests.test_Contractinator.test_sum4way}}\pysiglinewithargsret{\sphinxcode{\sphinxupquote{tests.test\_Contractinator.}}\sphinxbfcode{\sphinxupquote{test\_sum4way}}}{}{}
\sphinxAtStartPar
Tests the Contractinator.py ‘sum4way’ method.

\sphinxAtStartPar
Args:

\sphinxAtStartPar
Returns
\begin{description}
\item[{:}] \leavevmode
\sphinxAtStartPar
Nothing, but asserts if ‘sum4way’ properly sums items from the Field as it should.

\end{description}

\end{fulllineitems}

\index{test\_conservative4way() (in module tests.test\_Contractinator)@\spxentry{test\_conservative4way()}\spxextra{in module tests.test\_Contractinator}}

\begin{fulllineitems}
\phantomsection\label{\detokenize{autoapi/tests/test_Contractinator/index:tests.test_Contractinator.test_conservative4way}}\pysiglinewithargsret{\sphinxcode{\sphinxupquote{tests.test\_Contractinator.}}\sphinxbfcode{\sphinxupquote{test\_conservative4way}}}{}{}
\sphinxAtStartPar
Tests the Contractinator.py ‘conservative4way’ method. Note: does not test weighted averaging.

\sphinxAtStartPar
Args:

\sphinxAtStartPar
Returns
\begin{description}
\item[{:}] \leavevmode
\sphinxAtStartPar
Nothing, but asserts if ‘conservative4way’ properly averages items from the Field as it should.

\end{description}

\end{fulllineitems}



\subsubsection{\sphinxstyleliteralintitle{\sphinxupquote{tests.test\_Expandinator}}}
\label{\detokenize{autoapi/tests/test_Expandinator/index:module-tests.test_Expandinator}}\label{\detokenize{autoapi/tests/test_Expandinator/index:tests-test-expandinator}}\label{\detokenize{autoapi/tests/test_Expandinator/index::doc}}\index{module@\spxentry{module}!tests.test\_Expandinator@\spxentry{tests.test\_Expandinator}}\index{tests.test\_Expandinator@\spxentry{tests.test\_Expandinator}!module@\spxentry{module}}
\sphinxAtStartPar
Description

\sphinxAtStartPar
Tests the expandinator.

\sphinxAtStartPar
Libraries/Modules

\sphinxAtStartPar
\sphinxhyphen{}pytest

\sphinxAtStartPar
\sphinxhyphen{}Field

\sphinxAtStartPar
\sphinxhyphen{}numpy

\sphinxAtStartPar
\sphinxhyphen{}scipy

\sphinxAtStartPar
Notes


\paragraph{Module Contents}
\label{\detokenize{autoapi/tests/test_Expandinator/index:module-contents}}

\subparagraph{Functions}
\label{\detokenize{autoapi/tests/test_Expandinator/index:functions}}

\begin{savenotes}\sphinxatlongtablestart\begin{longtable}[c]{\X{1}{2}\X{1}{2}}
\hline

\endfirsthead

\multicolumn{2}{c}%
{\makebox[0pt]{\sphinxtablecontinued{\tablename\ \thetable{} \textendash{} continued from previous page}}}\\
\hline

\endhead

\hline
\multicolumn{2}{r}{\makebox[0pt][r]{\sphinxtablecontinued{continues on next page}}}\\
\endfoot

\endlastfoot

\sphinxAtStartPar
{\hyperref[\detokenize{autoapi/tests/test_Expandinator/index:tests.test_Expandinator.test_bilinear4way}]{\sphinxcrossref{\sphinxcode{\sphinxupquote{test\_bilinear4way}}}}}()
&
\sphinxAtStartPar
Tests the Expandinator.py ‘bilinear4way’ method. Does not currently assert (test) anything
\\
\hline
\end{longtable}\sphinxatlongtableend\end{savenotes}
\index{test\_bilinear4way() (in module tests.test\_Expandinator)@\spxentry{test\_bilinear4way()}\spxextra{in module tests.test\_Expandinator}}

\begin{fulllineitems}
\phantomsection\label{\detokenize{autoapi/tests/test_Expandinator/index:tests.test_Expandinator.test_bilinear4way}}\pysiglinewithargsret{\sphinxcode{\sphinxupquote{tests.test\_Expandinator.}}\sphinxbfcode{\sphinxupquote{test\_bilinear4way}}}{}{}
\sphinxAtStartPar
Tests the Expandinator.py ‘bilinear4way’ method. Does not currently assert (test) anything
as Expandinator.py does not currently work.

\sphinxAtStartPar
Args:

\sphinxAtStartPar
Returns
\begin{description}
\item[{:}] \leavevmode
\sphinxAtStartPar
Nothing, but asserts if ‘bilinear4way’ expands a Field properly.

\end{description}

\end{fulllineitems}



\subsubsection{\sphinxstyleliteralintitle{\sphinxupquote{tests.test\_Field}}}
\label{\detokenize{autoapi/tests/test_Field/index:module-tests.test_Field}}\label{\detokenize{autoapi/tests/test_Field/index:tests-test-field}}\label{\detokenize{autoapi/tests/test_Field/index::doc}}\index{module@\spxentry{module}!tests.test\_Field@\spxentry{tests.test\_Field}}\index{tests.test\_Field@\spxentry{tests.test\_Field}!module@\spxentry{module}}
\sphinxAtStartPar
Description

\sphinxAtStartPar
Tests the Field object to see if it works as expected

\sphinxAtStartPar
Libraries/Modules

\sphinxAtStartPar
\sphinxhyphen{}pytest

\sphinxAtStartPar
\sphinxhyphen{}Field

\sphinxAtStartPar
Notes

\sphinxAtStartPar
Runs the following tests:
\begin{enumerate}
\sphinxsetlistlabels{\arabic}{enumi}{enumii}{}{.}%
\item {} 
\sphinxAtStartPar
Checks that a 2D Field can be created

\item {} 
\sphinxAtStartPar
Checks that a 3D field can be created

\item {} 
\sphinxAtStartPar
Tests that we can set a whole field

\item {} 
\sphinxAtStartPar
Tests that we can set an individual elements

\item {} 
\sphinxAtStartPar
Tests + operation

\item {} 
\sphinxAtStartPar
Tests \sphinxhyphen{} operation

\item {} 
\sphinxAtStartPar
Tests * operation

\item {} 
\sphinxAtStartPar
Tests / operation

\end{enumerate}

\sphinxAtStartPar
Author(s)

\sphinxAtStartPar
Andy Rothstein


\paragraph{Module Contents}
\label{\detokenize{autoapi/tests/test_Field/index:module-contents}}

\subparagraph{Functions}
\label{\detokenize{autoapi/tests/test_Field/index:functions}}

\begin{savenotes}\sphinxatlongtablestart\begin{longtable}[c]{\X{1}{2}\X{1}{2}}
\hline

\endfirsthead

\multicolumn{2}{c}%
{\makebox[0pt]{\sphinxtablecontinued{\tablename\ \thetable{} \textendash{} continued from previous page}}}\\
\hline

\endhead

\hline
\multicolumn{2}{r}{\makebox[0pt][r]{\sphinxtablecontinued{continues on next page}}}\\
\endfoot

\endlastfoot

\sphinxAtStartPar
{\hyperref[\detokenize{autoapi/tests/test_Field/index:tests.test_Field.rng}]{\sphinxcrossref{\sphinxcode{\sphinxupquote{rng}}}}}(dim)
&
\sphinxAtStartPar

\\
\hline
\sphinxAtStartPar
{\hyperref[\detokenize{autoapi/tests/test_Field/index:tests.test_Field.test_constructor}]{\sphinxcrossref{\sphinxcode{\sphinxupquote{test\_constructor}}}}}()
&
\sphinxAtStartPar
Asserts that we can create a 8x8 field
\\
\hline
\sphinxAtStartPar
{\hyperref[\detokenize{autoapi/tests/test_Field/index:tests.test_Field.test_isfinite}]{\sphinxcrossref{\sphinxcode{\sphinxupquote{test\_isfinite}}}}}()
&
\sphinxAtStartPar
Asserts that we can create a 8x8 field
\\
\hline
\sphinxAtStartPar
{\hyperref[\detokenize{autoapi/tests/test_Field/index:tests.test_Field.test_constructor_1d}]{\sphinxcrossref{\sphinxcode{\sphinxupquote{test\_constructor\_1d}}}}}()
&
\sphinxAtStartPar
Asserts that we can create a 8x1 field
\\
\hline
\sphinxAtStartPar
{\hyperref[\detokenize{autoapi/tests/test_Field/index:tests.test_Field.test_constructor_2d}]{\sphinxcrossref{\sphinxcode{\sphinxupquote{test\_constructor\_2d}}}}}()
&
\sphinxAtStartPar
Asserts that we can create a 8x8 field
\\
\hline
\sphinxAtStartPar
{\hyperref[\detokenize{autoapi/tests/test_Field/index:tests.test_Field.test_constructor_3d}]{\sphinxcrossref{\sphinxcode{\sphinxupquote{test\_constructor\_3d}}}}}()
&
\sphinxAtStartPar
Asserts we can 8x8 field with state dimension of 4
\\
\hline
\sphinxAtStartPar
{\hyperref[\detokenize{autoapi/tests/test_Field/index:tests.test_Field.test_constructor_wrap_2arg}]{\sphinxcrossref{\sphinxcode{\sphinxupquote{test\_constructor\_wrap\_2arg}}}}}()
&
\sphinxAtStartPar
Asserts we wrap a numpy array with state dimension of 4
\\
\hline
\sphinxAtStartPar
{\hyperref[\detokenize{autoapi/tests/test_Field/index:tests.test_Field.test_constructor_wrap_1arg}]{\sphinxcrossref{\sphinxcode{\sphinxupquote{test\_constructor\_wrap\_1arg}}}}}()
&
\sphinxAtStartPar
Asserts we wrap a numpy array with state dimension of 4
\\
\hline
\sphinxAtStartPar
{\hyperref[\detokenize{autoapi/tests/test_Field/index:tests.test_Field.test_set_item}]{\sphinxcrossref{\sphinxcode{\sphinxupquote{test\_set\_item}}}}}()
&
\sphinxAtStartPar
Asserts that we can change a single element in a field
\\
\hline
\sphinxAtStartPar
{\hyperref[\detokenize{autoapi/tests/test_Field/index:tests.test_Field.test_add_func}]{\sphinxcrossref{\sphinxcode{\sphinxupquote{test\_add\_func}}}}}()
&
\sphinxAtStartPar
Tests that the add function works out
\\
\hline
\sphinxAtStartPar
{\hyperref[\detokenize{autoapi/tests/test_Field/index:tests.test_Field.test_difference_func}]{\sphinxcrossref{\sphinxcode{\sphinxupquote{test\_difference\_func}}}}}()
&
\sphinxAtStartPar
Tests that the difference function works out
\\
\hline
\sphinxAtStartPar
{\hyperref[\detokenize{autoapi/tests/test_Field/index:tests.test_Field.test_product_func}]{\sphinxcrossref{\sphinxcode{\sphinxupquote{test\_product\_func}}}}}()
&
\sphinxAtStartPar
Tests that the product function works out
\\
\hline
\sphinxAtStartPar
{\hyperref[\detokenize{autoapi/tests/test_Field/index:tests.test_Field.test_quotient_func}]{\sphinxcrossref{\sphinxcode{\sphinxupquote{test\_quotient\_func}}}}}()
&
\sphinxAtStartPar
Tests that the add function works out
\\
\hline
\sphinxAtStartPar
{\hyperref[\detokenize{autoapi/tests/test_Field/index:tests.test_Field.test_dimensional_mean}]{\sphinxcrossref{\sphinxcode{\sphinxupquote{test\_dimensional\_mean}}}}}()
&
\sphinxAtStartPar
Tests that the add function works out
\\
\hline
\sphinxAtStartPar
{\hyperref[\detokenize{autoapi/tests/test_Field/index:tests.test_Field.test_copy}]{\sphinxcrossref{\sphinxcode{\sphinxupquote{test\_copy}}}}}()
&
\sphinxAtStartPar
Tests we change the values of an array using the copy() method
\\
\hline
\sphinxAtStartPar
{\hyperref[\detokenize{autoapi/tests/test_Field/index:tests.test_Field.test_product_2d_3d}]{\sphinxcrossref{\sphinxcode{\sphinxupquote{test\_product\_2d\_3d}}}}}()
&
\sphinxAtStartPar
Tests we can multiply (n,m,p) field by (n,m) field
\\
\hline
\end{longtable}\sphinxatlongtableend\end{savenotes}
\index{rng() (in module tests.test\_Field)@\spxentry{rng()}\spxextra{in module tests.test\_Field}}

\begin{fulllineitems}
\phantomsection\label{\detokenize{autoapi/tests/test_Field/index:tests.test_Field.rng}}\pysiglinewithargsret{\sphinxcode{\sphinxupquote{tests.test\_Field.}}\sphinxbfcode{\sphinxupquote{rng}}}{\emph{\DUrole{n}{dim}}}{}
\end{fulllineitems}

\index{test\_constructor() (in module tests.test\_Field)@\spxentry{test\_constructor()}\spxextra{in module tests.test\_Field}}

\begin{fulllineitems}
\phantomsection\label{\detokenize{autoapi/tests/test_Field/index:tests.test_Field.test_constructor}}\pysiglinewithargsret{\sphinxcode{\sphinxupquote{tests.test\_Field.}}\sphinxbfcode{\sphinxupquote{test\_constructor}}}{}{}
\sphinxAtStartPar
Asserts that we can create a 8x8 field

\end{fulllineitems}

\index{test\_isfinite() (in module tests.test\_Field)@\spxentry{test\_isfinite()}\spxextra{in module tests.test\_Field}}

\begin{fulllineitems}
\phantomsection\label{\detokenize{autoapi/tests/test_Field/index:tests.test_Field.test_isfinite}}\pysiglinewithargsret{\sphinxcode{\sphinxupquote{tests.test\_Field.}}\sphinxbfcode{\sphinxupquote{test\_isfinite}}}{}{}
\sphinxAtStartPar
Asserts that we can create a 8x8 field

\end{fulllineitems}

\index{test\_constructor\_1d() (in module tests.test\_Field)@\spxentry{test\_constructor\_1d()}\spxextra{in module tests.test\_Field}}

\begin{fulllineitems}
\phantomsection\label{\detokenize{autoapi/tests/test_Field/index:tests.test_Field.test_constructor_1d}}\pysiglinewithargsret{\sphinxcode{\sphinxupquote{tests.test\_Field.}}\sphinxbfcode{\sphinxupquote{test\_constructor\_1d}}}{}{}
\sphinxAtStartPar
Asserts that we can create a 8x1 field

\end{fulllineitems}

\index{test\_constructor\_2d() (in module tests.test\_Field)@\spxentry{test\_constructor\_2d()}\spxextra{in module tests.test\_Field}}

\begin{fulllineitems}
\phantomsection\label{\detokenize{autoapi/tests/test_Field/index:tests.test_Field.test_constructor_2d}}\pysiglinewithargsret{\sphinxcode{\sphinxupquote{tests.test\_Field.}}\sphinxbfcode{\sphinxupquote{test\_constructor\_2d}}}{}{}
\sphinxAtStartPar
Asserts that we can create a 8x8 field

\end{fulllineitems}

\index{test\_constructor\_3d() (in module tests.test\_Field)@\spxentry{test\_constructor\_3d()}\spxextra{in module tests.test\_Field}}

\begin{fulllineitems}
\phantomsection\label{\detokenize{autoapi/tests/test_Field/index:tests.test_Field.test_constructor_3d}}\pysiglinewithargsret{\sphinxcode{\sphinxupquote{tests.test\_Field.}}\sphinxbfcode{\sphinxupquote{test\_constructor\_3d}}}{}{}
\sphinxAtStartPar
Asserts we can 8x8 field with state dimension of 4

\end{fulllineitems}

\index{test\_constructor\_wrap\_2arg() (in module tests.test\_Field)@\spxentry{test\_constructor\_wrap\_2arg()}\spxextra{in module tests.test\_Field}}

\begin{fulllineitems}
\phantomsection\label{\detokenize{autoapi/tests/test_Field/index:tests.test_Field.test_constructor_wrap_2arg}}\pysiglinewithargsret{\sphinxcode{\sphinxupquote{tests.test\_Field.}}\sphinxbfcode{\sphinxupquote{test\_constructor\_wrap\_2arg}}}{}{}
\sphinxAtStartPar
Asserts we wrap a numpy array with state dimension of 4

\end{fulllineitems}

\index{test\_constructor\_wrap\_1arg() (in module tests.test\_Field)@\spxentry{test\_constructor\_wrap\_1arg()}\spxextra{in module tests.test\_Field}}

\begin{fulllineitems}
\phantomsection\label{\detokenize{autoapi/tests/test_Field/index:tests.test_Field.test_constructor_wrap_1arg}}\pysiglinewithargsret{\sphinxcode{\sphinxupquote{tests.test\_Field.}}\sphinxbfcode{\sphinxupquote{test\_constructor\_wrap\_1arg}}}{}{}
\sphinxAtStartPar
Asserts we wrap a numpy array with state dimension of 4

\end{fulllineitems}

\index{test\_set\_item() (in module tests.test\_Field)@\spxentry{test\_set\_item()}\spxextra{in module tests.test\_Field}}

\begin{fulllineitems}
\phantomsection\label{\detokenize{autoapi/tests/test_Field/index:tests.test_Field.test_set_item}}\pysiglinewithargsret{\sphinxcode{\sphinxupquote{tests.test\_Field.}}\sphinxbfcode{\sphinxupquote{test\_set\_item}}}{}{}
\sphinxAtStartPar
Asserts that we can change a single element in a field

\end{fulllineitems}

\index{test\_add\_func() (in module tests.test\_Field)@\spxentry{test\_add\_func()}\spxextra{in module tests.test\_Field}}

\begin{fulllineitems}
\phantomsection\label{\detokenize{autoapi/tests/test_Field/index:tests.test_Field.test_add_func}}\pysiglinewithargsret{\sphinxcode{\sphinxupquote{tests.test\_Field.}}\sphinxbfcode{\sphinxupquote{test\_add\_func}}}{}{}
\sphinxAtStartPar
Tests that the add function works out

\end{fulllineitems}

\index{test\_difference\_func() (in module tests.test\_Field)@\spxentry{test\_difference\_func()}\spxextra{in module tests.test\_Field}}

\begin{fulllineitems}
\phantomsection\label{\detokenize{autoapi/tests/test_Field/index:tests.test_Field.test_difference_func}}\pysiglinewithargsret{\sphinxcode{\sphinxupquote{tests.test\_Field.}}\sphinxbfcode{\sphinxupquote{test\_difference\_func}}}{}{}
\sphinxAtStartPar
Tests that the difference function works out

\end{fulllineitems}

\index{test\_product\_func() (in module tests.test\_Field)@\spxentry{test\_product\_func()}\spxextra{in module tests.test\_Field}}

\begin{fulllineitems}
\phantomsection\label{\detokenize{autoapi/tests/test_Field/index:tests.test_Field.test_product_func}}\pysiglinewithargsret{\sphinxcode{\sphinxupquote{tests.test\_Field.}}\sphinxbfcode{\sphinxupquote{test\_product\_func}}}{}{}
\sphinxAtStartPar
Tests that the product function works out

\end{fulllineitems}

\index{test\_quotient\_func() (in module tests.test\_Field)@\spxentry{test\_quotient\_func()}\spxextra{in module tests.test\_Field}}

\begin{fulllineitems}
\phantomsection\label{\detokenize{autoapi/tests/test_Field/index:tests.test_Field.test_quotient_func}}\pysiglinewithargsret{\sphinxcode{\sphinxupquote{tests.test\_Field.}}\sphinxbfcode{\sphinxupquote{test\_quotient\_func}}}{}{}
\sphinxAtStartPar
Tests that the add function works out

\end{fulllineitems}

\index{test\_dimensional\_mean() (in module tests.test\_Field)@\spxentry{test\_dimensional\_mean()}\spxextra{in module tests.test\_Field}}

\begin{fulllineitems}
\phantomsection\label{\detokenize{autoapi/tests/test_Field/index:tests.test_Field.test_dimensional_mean}}\pysiglinewithargsret{\sphinxcode{\sphinxupquote{tests.test\_Field.}}\sphinxbfcode{\sphinxupquote{test\_dimensional\_mean}}}{}{}
\sphinxAtStartPar
Tests that the add function works out

\end{fulllineitems}

\index{test\_copy() (in module tests.test\_Field)@\spxentry{test\_copy()}\spxextra{in module tests.test\_Field}}

\begin{fulllineitems}
\phantomsection\label{\detokenize{autoapi/tests/test_Field/index:tests.test_Field.test_copy}}\pysiglinewithargsret{\sphinxcode{\sphinxupquote{tests.test\_Field.}}\sphinxbfcode{\sphinxupquote{test\_copy}}}{}{}
\sphinxAtStartPar
Tests we change the values of an array using the copy() method

\end{fulllineitems}

\index{test\_product\_2d\_3d() (in module tests.test\_Field)@\spxentry{test\_product\_2d\_3d()}\spxextra{in module tests.test\_Field}}

\begin{fulllineitems}
\phantomsection\label{\detokenize{autoapi/tests/test_Field/index:tests.test_Field.test_product_2d_3d}}\pysiglinewithargsret{\sphinxcode{\sphinxupquote{tests.test\_Field.}}\sphinxbfcode{\sphinxupquote{test\_product\_2d\_3d}}}{}{}
\sphinxAtStartPar
Tests we can multiply (n,m,p) field by (n,m) field

\end{fulllineitems}



\subsubsection{\sphinxstyleliteralintitle{\sphinxupquote{tests.test\_Input}}}
\label{\detokenize{autoapi/tests/test_Input/index:module-tests.test_Input}}\label{\detokenize{autoapi/tests/test_Input/index:tests-test-input}}\label{\detokenize{autoapi/tests/test_Input/index::doc}}\index{module@\spxentry{module}!tests.test\_Input@\spxentry{tests.test\_Input}}\index{tests.test\_Input@\spxentry{tests.test\_Input}!module@\spxentry{module}}
\sphinxAtStartPar
Tests the Input object to make sure all inputs as as expected
\begin{description}
\item[{Libraries/Modules:}] \leavevmode
\sphinxAtStartPar
pytest

\sphinxAtStartPar
Input

\end{description}
\subsubsection*{Notes}

\sphinxAtStartPar
Runs the following tests:
\begin{enumerate}
\sphinxsetlistlabels{\arabic}{enumi}{enumii}{}{.}%
\item {} 
\sphinxAtStartPar
Checks that the input airfoil geometry is a closed curve

\item {} 
\sphinxAtStartPar
Checks that the airfoil geomtry is of the right length

\item {} 
\sphinxAtStartPar
Check that all the required dims values are being read into the dims dictionary

\item {} 
\sphinxAtStartPar
Check that all the required solv\_param values are being read into the solv\_param dictionary

\item {} 
\sphinxAtStartPar
Check that all the required flo\_param values are being read into the flo\_param dictionary

\item {} 
\sphinxAtStartPar
Check that all the required geo\_param values are being read into the geo\_param dictionary

\end{enumerate}

\sphinxAtStartPar
Author(s)

\sphinxAtStartPar
Vedin Dewan


\paragraph{Module Contents}
\label{\detokenize{autoapi/tests/test_Input/index:module-contents}}

\subparagraph{Functions}
\label{\detokenize{autoapi/tests/test_Input/index:functions}}

\begin{savenotes}\sphinxatlongtablestart\begin{longtable}[c]{\X{1}{2}\X{1}{2}}
\hline

\endfirsthead

\multicolumn{2}{c}%
{\makebox[0pt]{\sphinxtablecontinued{\tablename\ \thetable{} \textendash{} continued from previous page}}}\\
\hline

\endhead

\hline
\multicolumn{2}{r}{\makebox[0pt][r]{\sphinxtablecontinued{continues on next page}}}\\
\endfoot

\endlastfoot

\sphinxAtStartPar
{\hyperref[\detokenize{autoapi/tests/test_Input/index:tests.test_Input.test_geometry_closed}]{\sphinxcrossref{\sphinxcode{\sphinxupquote{test\_geometry\_closed}}}}}()
&
\sphinxAtStartPar
Asserts that geometry is a closed curve
\\
\hline
\sphinxAtStartPar
{\hyperref[\detokenize{autoapi/tests/test_Input/index:tests.test_Input.test_geom_length}]{\sphinxcrossref{\sphinxcode{\sphinxupquote{test\_geom\_length}}}}}()
&
\sphinxAtStartPar
Asserts that the number of points on the geometry
\\
\hline
\sphinxAtStartPar
{\hyperref[\detokenize{autoapi/tests/test_Input/index:tests.test_Input.test_dims}]{\sphinxcrossref{\sphinxcode{\sphinxupquote{test\_dims}}}}}()
&
\sphinxAtStartPar
Asserts that the all the required values in the
\\
\hline
\sphinxAtStartPar
{\hyperref[\detokenize{autoapi/tests/test_Input/index:tests.test_Input.test_solv_param}]{\sphinxcrossref{\sphinxcode{\sphinxupquote{test\_solv\_param}}}}}()
&
\sphinxAtStartPar
Asserts that the all the required values in the
\\
\hline
\sphinxAtStartPar
{\hyperref[\detokenize{autoapi/tests/test_Input/index:tests.test_Input.test_flo_param}]{\sphinxcrossref{\sphinxcode{\sphinxupquote{test\_flo\_param}}}}}()
&
\sphinxAtStartPar
Asserts that the all the required values in the
\\
\hline
\sphinxAtStartPar
{\hyperref[\detokenize{autoapi/tests/test_Input/index:tests.test_Input.test_geo_param}]{\sphinxcrossref{\sphinxcode{\sphinxupquote{test\_geo\_param}}}}}()
&
\sphinxAtStartPar
Asserts that the all the required values in the
\\
\hline
\end{longtable}\sphinxatlongtableend\end{savenotes}


\subparagraph{Attributes}
\label{\detokenize{autoapi/tests/test_Input/index:attributes}}

\begin{savenotes}\sphinxatlongtablestart\begin{longtable}[c]{\X{1}{2}\X{1}{2}}
\hline

\endfirsthead

\multicolumn{2}{c}%
{\makebox[0pt]{\sphinxtablecontinued{\tablename\ \thetable{} \textendash{} continued from previous page}}}\\
\hline

\endhead

\hline
\multicolumn{2}{r}{\makebox[0pt][r]{\sphinxtablecontinued{continues on next page}}}\\
\endfoot

\endlastfoot

\sphinxAtStartPar
{\hyperref[\detokenize{autoapi/tests/test_Input/index:tests.test_Input.input}]{\sphinxcrossref{\sphinxcode{\sphinxupquote{input}}}}}
&
\sphinxAtStartPar

\\
\hline
\sphinxAtStartPar
{\hyperref[\detokenize{autoapi/tests/test_Input/index:tests.test_Input.xn}]{\sphinxcrossref{\sphinxcode{\sphinxupquote{xn}}}}}
&
\sphinxAtStartPar

\\
\hline
\sphinxAtStartPar
{\hyperref[\detokenize{autoapi/tests/test_Input/index:tests.test_Input.yn}]{\sphinxcrossref{\sphinxcode{\sphinxupquote{yn}}}}}
&
\sphinxAtStartPar

\\
\hline
\end{longtable}\sphinxatlongtableend\end{savenotes}
\index{input (in module tests.test\_Input)@\spxentry{input}\spxextra{in module tests.test\_Input}}

\begin{fulllineitems}
\phantomsection\label{\detokenize{autoapi/tests/test_Input/index:tests.test_Input.input}}\pysigline{\sphinxcode{\sphinxupquote{tests.test\_Input.}}\sphinxbfcode{\sphinxupquote{input}}}
\end{fulllineitems}

\index{xn (in module tests.test\_Input)@\spxentry{xn}\spxextra{in module tests.test\_Input}}

\begin{fulllineitems}
\phantomsection\label{\detokenize{autoapi/tests/test_Input/index:tests.test_Input.xn}}\pysigline{\sphinxcode{\sphinxupquote{tests.test\_Input.}}\sphinxbfcode{\sphinxupquote{xn}}}
\end{fulllineitems}

\index{yn (in module tests.test\_Input)@\spxentry{yn}\spxextra{in module tests.test\_Input}}

\begin{fulllineitems}
\phantomsection\label{\detokenize{autoapi/tests/test_Input/index:tests.test_Input.yn}}\pysigline{\sphinxcode{\sphinxupquote{tests.test\_Input.}}\sphinxbfcode{\sphinxupquote{yn}}}
\end{fulllineitems}

\index{test\_geometry\_closed() (in module tests.test\_Input)@\spxentry{test\_geometry\_closed()}\spxextra{in module tests.test\_Input}}

\begin{fulllineitems}
\phantomsection\label{\detokenize{autoapi/tests/test_Input/index:tests.test_Input.test_geometry_closed}}\pysiglinewithargsret{\sphinxcode{\sphinxupquote{tests.test\_Input.}}\sphinxbfcode{\sphinxupquote{test\_geometry\_closed}}}{}{}
\sphinxAtStartPar
Asserts that geometry is a closed curve

\end{fulllineitems}

\index{test\_geom\_length() (in module tests.test\_Input)@\spxentry{test\_geom\_length()}\spxextra{in module tests.test\_Input}}

\begin{fulllineitems}
\phantomsection\label{\detokenize{autoapi/tests/test_Input/index:tests.test_Input.test_geom_length}}\pysiglinewithargsret{\sphinxcode{\sphinxupquote{tests.test\_Input.}}\sphinxbfcode{\sphinxupquote{test\_geom\_length}}}{}{}
\sphinxAtStartPar
Asserts that the number of points on the geometry
is as expected

\end{fulllineitems}

\index{test\_dims() (in module tests.test\_Input)@\spxentry{test\_dims()}\spxextra{in module tests.test\_Input}}

\begin{fulllineitems}
\phantomsection\label{\detokenize{autoapi/tests/test_Input/index:tests.test_Input.test_dims}}\pysiglinewithargsret{\sphinxcode{\sphinxupquote{tests.test\_Input.}}\sphinxbfcode{\sphinxupquote{test\_dims}}}{}{}
\sphinxAtStartPar
Asserts that the all the required values in the
dims dictionary are being read in

\end{fulllineitems}

\index{test\_solv\_param() (in module tests.test\_Input)@\spxentry{test\_solv\_param()}\spxextra{in module tests.test\_Input}}

\begin{fulllineitems}
\phantomsection\label{\detokenize{autoapi/tests/test_Input/index:tests.test_Input.test_solv_param}}\pysiglinewithargsret{\sphinxcode{\sphinxupquote{tests.test\_Input.}}\sphinxbfcode{\sphinxupquote{test\_solv\_param}}}{}{}
\sphinxAtStartPar
Asserts that the all the required values in the
solv\_param dictionary are being read in

\end{fulllineitems}

\index{test\_flo\_param() (in module tests.test\_Input)@\spxentry{test\_flo\_param()}\spxextra{in module tests.test\_Input}}

\begin{fulllineitems}
\phantomsection\label{\detokenize{autoapi/tests/test_Input/index:tests.test_Input.test_flo_param}}\pysiglinewithargsret{\sphinxcode{\sphinxupquote{tests.test\_Input.}}\sphinxbfcode{\sphinxupquote{test\_flo\_param}}}{}{}
\sphinxAtStartPar
Asserts that the all the required values in the
flo\_param dictionary are being read in

\end{fulllineitems}

\index{test\_geo\_param() (in module tests.test\_Input)@\spxentry{test\_geo\_param()}\spxextra{in module tests.test\_Input}}

\begin{fulllineitems}
\phantomsection\label{\detokenize{autoapi/tests/test_Input/index:tests.test_Input.test_geo_param}}\pysiglinewithargsret{\sphinxcode{\sphinxupquote{tests.test\_Input.}}\sphinxbfcode{\sphinxupquote{test\_geo\_param}}}{}{}
\sphinxAtStartPar
Asserts that the all the required values in the
geo\_param dictionary are being read in

\end{fulllineitems}



\subsubsection{\sphinxstyleliteralintitle{\sphinxupquote{tests.test\_NavierStokes}}}
\label{\detokenize{autoapi/tests/test_NavierStokes/index:module-tests.test_NavierStokes}}\label{\detokenize{autoapi/tests/test_NavierStokes/index:tests-test-navierstokes}}\label{\detokenize{autoapi/tests/test_NavierStokes/index::doc}}\index{module@\spxentry{module}!tests.test\_NavierStokes@\spxentry{tests.test\_NavierStokes}}\index{tests.test\_NavierStokes@\spxentry{tests.test\_NavierStokes}!module@\spxentry{module}}
\sphinxAtStartPar
Tests the NavierStokes object to make sure it behaves as expected
\begin{description}
\item[{Libraries/Modules:}] \leavevmode
\sphinxAtStartPar
pytest
Input
NavierStokes
AirfoilMap
CellCenterWS
NS\_Airfoil

\end{description}
\subsubsection*{Notes}

\sphinxAtStartPar
Runs the following tests:
1. Checks that the constructor works
2. Checks that the airfoil geomtry is of the right length
\begin{enumerate}
\sphinxsetlistlabels{\arabic}{enumi}{enumii}{}{.}%
\setcounter{enumi}{2}
\item {} 
\sphinxAtStartPar
Check that all the required dims values are being read into the dims dictionary

\item {} 
\sphinxAtStartPar
Check that all the required solv\_param values are being read into the solv\_param dictionary

\item {} 
\sphinxAtStartPar
Check that all the required flo\_param values are being read into the flo\_param dictionary

\item {} 
\sphinxAtStartPar
Check that all the required geo\_param values are being read into the geo\_param dictionary

\end{enumerate}

\sphinxAtStartPar
Author(s)

\sphinxAtStartPar
Vedin Dewan


\paragraph{Module Contents}
\label{\detokenize{autoapi/tests/test_NavierStokes/index:module-contents}}

\subparagraph{Functions}
\label{\detokenize{autoapi/tests/test_NavierStokes/index:functions}}

\begin{savenotes}\sphinxatlongtablestart\begin{longtable}[c]{\X{1}{2}\X{1}{2}}
\hline

\endfirsthead

\multicolumn{2}{c}%
{\makebox[0pt]{\sphinxtablecontinued{\tablename\ \thetable{} \textendash{} continued from previous page}}}\\
\hline

\endhead

\hline
\multicolumn{2}{r}{\makebox[0pt][r]{\sphinxtablecontinued{continues on next page}}}\\
\endfoot

\endlastfoot

\sphinxAtStartPar
{\hyperref[\detokenize{autoapi/tests/test_NavierStokes/index:tests.test_NavierStokes.test_constructor}]{\sphinxcrossref{\sphinxcode{\sphinxupquote{test\_constructor}}}}}()
&
\sphinxAtStartPar
Asserts that we can create a NavierStokes object
\\
\hline
\end{longtable}\sphinxatlongtableend\end{savenotes}


\subparagraph{Attributes}
\label{\detokenize{autoapi/tests/test_NavierStokes/index:attributes}}

\begin{savenotes}\sphinxatlongtablestart\begin{longtable}[c]{\X{1}{2}\X{1}{2}}
\hline

\endfirsthead

\multicolumn{2}{c}%
{\makebox[0pt]{\sphinxtablecontinued{\tablename\ \thetable{} \textendash{} continued from previous page}}}\\
\hline

\endhead

\hline
\multicolumn{2}{r}{\makebox[0pt][r]{\sphinxtablecontinued{continues on next page}}}\\
\endfoot

\endlastfoot

\sphinxAtStartPar
{\hyperref[\detokenize{autoapi/tests/test_NavierStokes/index:tests.test_NavierStokes.filename}]{\sphinxcrossref{\sphinxcode{\sphinxupquote{filename}}}}}
&
\sphinxAtStartPar

\\
\hline
\sphinxAtStartPar
{\hyperref[\detokenize{autoapi/tests/test_NavierStokes/index:tests.test_NavierStokes.input}]{\sphinxcrossref{\sphinxcode{\sphinxupquote{input}}}}}
&
\sphinxAtStartPar

\\
\hline
\sphinxAtStartPar
{\hyperref[\detokenize{autoapi/tests/test_NavierStokes/index:tests.test_NavierStokes.gridInput}]{\sphinxcrossref{\sphinxcode{\sphinxupquote{gridInput}}}}}
&
\sphinxAtStartPar

\\
\hline
\sphinxAtStartPar
{\hyperref[\detokenize{autoapi/tests/test_NavierStokes/index:tests.test_NavierStokes.grid_dim}]{\sphinxcrossref{\sphinxcode{\sphinxupquote{grid\_dim}}}}}
&
\sphinxAtStartPar

\\
\hline
\sphinxAtStartPar
{\hyperref[\detokenize{autoapi/tests/test_NavierStokes/index:tests.test_NavierStokes.modelInput}]{\sphinxcrossref{\sphinxcode{\sphinxupquote{modelInput}}}}}
&
\sphinxAtStartPar

\\
\hline
\sphinxAtStartPar
{\hyperref[\detokenize{autoapi/tests/test_NavierStokes/index:tests.test_NavierStokes.grid}]{\sphinxcrossref{\sphinxcode{\sphinxupquote{grid}}}}}
&
\sphinxAtStartPar

\\
\hline
\sphinxAtStartPar
{\hyperref[\detokenize{autoapi/tests/test_NavierStokes/index:tests.test_NavierStokes.workspace}]{\sphinxcrossref{\sphinxcode{\sphinxupquote{workspace}}}}}
&
\sphinxAtStartPar

\\
\hline
\sphinxAtStartPar
{\hyperref[\detokenize{autoapi/tests/test_NavierStokes/index:tests.test_NavierStokes.bcmodel}]{\sphinxcrossref{\sphinxcode{\sphinxupquote{bcmodel}}}}}
&
\sphinxAtStartPar

\\
\hline
\end{longtable}\sphinxatlongtableend\end{savenotes}
\index{filename (in module tests.test\_NavierStokes)@\spxentry{filename}\spxextra{in module tests.test\_NavierStokes}}

\begin{fulllineitems}
\phantomsection\label{\detokenize{autoapi/tests/test_NavierStokes/index:tests.test_NavierStokes.filename}}\pysigline{\sphinxcode{\sphinxupquote{tests.test\_NavierStokes.}}\sphinxbfcode{\sphinxupquote{filename}}\sphinxbfcode{\sphinxupquote{ = rae9\sphinxhyphen{}s1.data}}}
\end{fulllineitems}

\index{input (in module tests.test\_NavierStokes)@\spxentry{input}\spxextra{in module tests.test\_NavierStokes}}

\begin{fulllineitems}
\phantomsection\label{\detokenize{autoapi/tests/test_NavierStokes/index:tests.test_NavierStokes.input}}\pysigline{\sphinxcode{\sphinxupquote{tests.test\_NavierStokes.}}\sphinxbfcode{\sphinxupquote{input}}}
\end{fulllineitems}

\index{gridInput (in module tests.test\_NavierStokes)@\spxentry{gridInput}\spxextra{in module tests.test\_NavierStokes}}

\begin{fulllineitems}
\phantomsection\label{\detokenize{autoapi/tests/test_NavierStokes/index:tests.test_NavierStokes.gridInput}}\pysigline{\sphinxcode{\sphinxupquote{tests.test\_NavierStokes.}}\sphinxbfcode{\sphinxupquote{gridInput}}}
\end{fulllineitems}

\index{grid\_dim (in module tests.test\_NavierStokes)@\spxentry{grid\_dim}\spxextra{in module tests.test\_NavierStokes}}

\begin{fulllineitems}
\phantomsection\label{\detokenize{autoapi/tests/test_NavierStokes/index:tests.test_NavierStokes.grid_dim}}\pysigline{\sphinxcode{\sphinxupquote{tests.test\_NavierStokes.}}\sphinxbfcode{\sphinxupquote{grid\_dim}}}
\end{fulllineitems}

\index{modelInput (in module tests.test\_NavierStokes)@\spxentry{modelInput}\spxextra{in module tests.test\_NavierStokes}}

\begin{fulllineitems}
\phantomsection\label{\detokenize{autoapi/tests/test_NavierStokes/index:tests.test_NavierStokes.modelInput}}\pysigline{\sphinxcode{\sphinxupquote{tests.test\_NavierStokes.}}\sphinxbfcode{\sphinxupquote{modelInput}}}
\end{fulllineitems}

\index{grid (in module tests.test\_NavierStokes)@\spxentry{grid}\spxextra{in module tests.test\_NavierStokes}}

\begin{fulllineitems}
\phantomsection\label{\detokenize{autoapi/tests/test_NavierStokes/index:tests.test_NavierStokes.grid}}\pysigline{\sphinxcode{\sphinxupquote{tests.test\_NavierStokes.}}\sphinxbfcode{\sphinxupquote{grid}}}
\end{fulllineitems}

\index{workspace (in module tests.test\_NavierStokes)@\spxentry{workspace}\spxextra{in module tests.test\_NavierStokes}}

\begin{fulllineitems}
\phantomsection\label{\detokenize{autoapi/tests/test_NavierStokes/index:tests.test_NavierStokes.workspace}}\pysigline{\sphinxcode{\sphinxupquote{tests.test\_NavierStokes.}}\sphinxbfcode{\sphinxupquote{workspace}}}
\end{fulllineitems}

\index{bcmodel (in module tests.test\_NavierStokes)@\spxentry{bcmodel}\spxextra{in module tests.test\_NavierStokes}}

\begin{fulllineitems}
\phantomsection\label{\detokenize{autoapi/tests/test_NavierStokes/index:tests.test_NavierStokes.bcmodel}}\pysigline{\sphinxcode{\sphinxupquote{tests.test\_NavierStokes.}}\sphinxbfcode{\sphinxupquote{bcmodel}}}
\end{fulllineitems}

\index{test\_constructor() (in module tests.test\_NavierStokes)@\spxentry{test\_constructor()}\spxextra{in module tests.test\_NavierStokes}}

\begin{fulllineitems}
\phantomsection\label{\detokenize{autoapi/tests/test_NavierStokes/index:tests.test_NavierStokes.test_constructor}}\pysiglinewithargsret{\sphinxcode{\sphinxupquote{tests.test\_NavierStokes.}}\sphinxbfcode{\sphinxupquote{test\_constructor}}}{}{}
\sphinxAtStartPar
Asserts that we can create a NavierStokes object

\end{fulllineitems}



\subsubsection{\sphinxstyleliteralintitle{\sphinxupquote{tests.test\_Workspace}}}
\label{\detokenize{autoapi/tests/test_Workspace/index:module-tests.test_Workspace}}\label{\detokenize{autoapi/tests/test_Workspace/index:tests-test-workspace}}\label{\detokenize{autoapi/tests/test_Workspace/index::doc}}\index{module@\spxentry{module}!tests.test\_Workspace@\spxentry{tests.test\_Workspace}}\index{tests.test\_Workspace@\spxentry{tests.test\_Workspace}!module@\spxentry{module}}
\sphinxAtStartPar
Tests the Workspace object to see if it works as expected
\begin{description}
\item[{Libraries/Modules:}] \leavevmode
\sphinxAtStartPar
pytest

\sphinxAtStartPar
Workspace

\end{description}
\subsubsection*{Notes}

\sphinxAtStartPar
Runs the following tests:
\begin{enumerate}
\sphinxsetlistlabels{\arabic}{enumi}{enumii}{}{.}%
\item {} 
\sphinxAtStartPar
Verify x, xc, vol can be retrieved

\item {} 
\sphinxAtStartPar
Verify x, xc, vol are non\sphinxhyphen{}zero

\item {} 
\sphinxAtStartPar
Verify that init\_vars works as expected

\item {} 
\sphinxAtStartPar
Checks that has\_dict exists and has\_dict return as expected

\item {} 
\sphinxAtStartPar
Checks is\_finest method

\end{enumerate}


\paragraph{Module Contents}
\label{\detokenize{autoapi/tests/test_Workspace/index:module-contents}}

\subparagraph{Functions}
\label{\detokenize{autoapi/tests/test_Workspace/index:functions}}

\begin{savenotes}\sphinxatlongtablestart\begin{longtable}[c]{\X{1}{2}\X{1}{2}}
\hline

\endfirsthead

\multicolumn{2}{c}%
{\makebox[0pt]{\sphinxtablecontinued{\tablename\ \thetable{} \textendash{} continued from previous page}}}\\
\hline

\endhead

\hline
\multicolumn{2}{r}{\makebox[0pt][r]{\sphinxtablecontinued{continues on next page}}}\\
\endfoot

\endlastfoot

\sphinxAtStartPar
{\hyperref[\detokenize{autoapi/tests/test_Workspace/index:tests.test_Workspace.test_x}]{\sphinxcrossref{\sphinxcode{\sphinxupquote{test\_x}}}}}()
&
\sphinxAtStartPar
Asserts that we can retrieve x and that it is non zero
\\
\hline
\sphinxAtStartPar
{\hyperref[\detokenize{autoapi/tests/test_Workspace/index:tests.test_Workspace.test_xc}]{\sphinxcrossref{\sphinxcode{\sphinxupquote{test\_xc}}}}}()
&
\sphinxAtStartPar
Asserts that we can retrieve xc and that it is non zero
\\
\hline
\sphinxAtStartPar
{\hyperref[\detokenize{autoapi/tests/test_Workspace/index:tests.test_Workspace.test_vol}]{\sphinxcrossref{\sphinxcode{\sphinxupquote{test\_vol}}}}}()
&
\sphinxAtStartPar
Asserts that we can retrieve vol and that it is non zero
\\
\hline
\sphinxAtStartPar
{\hyperref[\detokenize{autoapi/tests/test_Workspace/index:tests.test_Workspace.test_init_vars}]{\sphinxcrossref{\sphinxcode{\sphinxupquote{test\_init\_vars}}}}}()
&
\sphinxAtStartPar
Verify that init\_vars works as expected
\\
\hline
\sphinxAtStartPar
{\hyperref[\detokenize{autoapi/tests/test_Workspace/index:tests.test_Workspace.test_has_dict}]{\sphinxcrossref{\sphinxcode{\sphinxupquote{test\_has\_dict}}}}}()
&
\sphinxAtStartPar
Asserts that has dict
\\
\hline
\sphinxAtStartPar
{\hyperref[\detokenize{autoapi/tests/test_Workspace/index:tests.test_Workspace.test_finest}]{\sphinxcrossref{\sphinxcode{\sphinxupquote{test\_finest}}}}}()
&
\sphinxAtStartPar
Asserts isFinest
\\
\hline
\end{longtable}\sphinxatlongtableend\end{savenotes}
\index{test\_x() (in module tests.test\_Workspace)@\spxentry{test\_x()}\spxextra{in module tests.test\_Workspace}}

\begin{fulllineitems}
\phantomsection\label{\detokenize{autoapi/tests/test_Workspace/index:tests.test_Workspace.test_x}}\pysiglinewithargsret{\sphinxcode{\sphinxupquote{tests.test\_Workspace.}}\sphinxbfcode{\sphinxupquote{test\_x}}}{}{}
\sphinxAtStartPar
Asserts that we can retrieve x and that it is non zero

\end{fulllineitems}

\index{test\_xc() (in module tests.test\_Workspace)@\spxentry{test\_xc()}\spxextra{in module tests.test\_Workspace}}

\begin{fulllineitems}
\phantomsection\label{\detokenize{autoapi/tests/test_Workspace/index:tests.test_Workspace.test_xc}}\pysiglinewithargsret{\sphinxcode{\sphinxupquote{tests.test\_Workspace.}}\sphinxbfcode{\sphinxupquote{test\_xc}}}{}{}
\sphinxAtStartPar
Asserts that we can retrieve xc and that it is non zero

\end{fulllineitems}

\index{test\_vol() (in module tests.test\_Workspace)@\spxentry{test\_vol()}\spxextra{in module tests.test\_Workspace}}

\begin{fulllineitems}
\phantomsection\label{\detokenize{autoapi/tests/test_Workspace/index:tests.test_Workspace.test_vol}}\pysiglinewithargsret{\sphinxcode{\sphinxupquote{tests.test\_Workspace.}}\sphinxbfcode{\sphinxupquote{test\_vol}}}{}{}
\sphinxAtStartPar
Asserts that we can retrieve vol and that it is non zero

\end{fulllineitems}

\index{test\_init\_vars() (in module tests.test\_Workspace)@\spxentry{test\_init\_vars()}\spxextra{in module tests.test\_Workspace}}

\begin{fulllineitems}
\phantomsection\label{\detokenize{autoapi/tests/test_Workspace/index:tests.test_Workspace.test_init_vars}}\pysiglinewithargsret{\sphinxcode{\sphinxupquote{tests.test\_Workspace.}}\sphinxbfcode{\sphinxupquote{test\_init\_vars}}}{}{}
\sphinxAtStartPar
Verify that init\_vars works as expected

\end{fulllineitems}

\index{test\_has\_dict() (in module tests.test\_Workspace)@\spxentry{test\_has\_dict()}\spxextra{in module tests.test\_Workspace}}

\begin{fulllineitems}
\phantomsection\label{\detokenize{autoapi/tests/test_Workspace/index:tests.test_Workspace.test_has_dict}}\pysiglinewithargsret{\sphinxcode{\sphinxupquote{tests.test\_Workspace.}}\sphinxbfcode{\sphinxupquote{test\_has\_dict}}}{}{}
\sphinxAtStartPar
Asserts that has dict

\end{fulllineitems}

\index{test\_finest() (in module tests.test\_Workspace)@\spxentry{test\_finest()}\spxextra{in module tests.test\_Workspace}}

\begin{fulllineitems}
\phantomsection\label{\detokenize{autoapi/tests/test_Workspace/index:tests.test_Workspace.test_finest}}\pysiglinewithargsret{\sphinxcode{\sphinxupquote{tests.test\_Workspace.}}\sphinxbfcode{\sphinxupquote{test\_finest}}}{}{}
\sphinxAtStartPar
Asserts isFinest

\end{fulllineitems}



\subsubsection{\sphinxstyleliteralintitle{\sphinxupquote{tests.test\_bcfar}}}
\label{\detokenize{autoapi/tests/test_bcfar/index:module-tests.test_bcfar}}\label{\detokenize{autoapi/tests/test_bcfar/index:tests-test-bcfar}}\label{\detokenize{autoapi/tests/test_bcfar/index::doc}}\index{module@\spxentry{module}!tests.test\_bcfar@\spxentry{tests.test\_bcfar}}\index{tests.test\_bcfar@\spxentry{tests.test\_bcfar}!module@\spxentry{module}}
\sphinxAtStartPar
import sys
sys.path.append(“../”)

\sphinxAtStartPar
import bcfar\_fort
import numpy as np
\#from eflux\_arr import eflux
from Field import Field

\sphinxAtStartPar
\# grab grid related parameter
\#G = ws.grid
nx = 4
ny = 10
il = nx+1
jl = ny+1
ie = il+1
je = jl+1
itl = 1
itu = 3
ib = il + 2
jb = jl + 2

\sphinxAtStartPar
\# flow related vars
w = Field({[}ib,jb{]},4) \# state
w.vals = np.array(w.vals + 15*np.random.standard\_normal({[}ib,jb,4{]}),order = ‘f’)
P = Field({[}ib,jb{]}) \# pressure
lv = Field({[}ib,jb{]}) \# laminar viscocity
ev = Field({[}ib,jb{]}) \# eddy viscocity

\sphinxAtStartPar
\# mesh related vars
porI = Field({[}ib,jb{]},2) \# mesh vertices
porI.vals = np.array(porI.vals + 15*np.random.standard\_normal({[}ib,jb,2{]}),order = ‘f’)
porJ = Field({[}ib,jb{]},2) \# mesh centers
porJ.vals = np.array(porJ.vals + 15*np.random.standard\_normal({[}ib,jb,2{]}),order = ‘f’)
xc = Field({[}ib,jb{]},2) \# mesh vertices
xc.vals = np.array(porI.vals + 15*np.random.standard\_normal({[}ib,jb,2{]}),order = ‘f’)
x = Field({[}ib,jb{]},2) \# mesh centers
x.vals = np.array(porJ.vals + 15*np.random.standard\_normal({[}ib,jb,2{]}),order = ‘f’)

\sphinxAtStartPar
\# solver related vars
fw = Field({[}ib,jb{]},4)
radI = Field({[}ib,jb{]},2) \# stability I
radJ = Field({[}ib,jb{]},2) \# stability J

\sphinxAtStartPar
gamma = 1.4
rm = 1.2
scal = 1.8
re = 50000
chord = 2.6
prn = 1000
prt = 10000
mode = 1
rfil = 0.8
vis0 = 0.5
rho0 = 1
p0 = 1;h0 = 1;c0 = 1;u0 = 1;v0 = 1;ca= 1;sa = 1; xm = 1; ym = 1; kvis = 1; bc = 1

\sphinxAtStartPar
print(w.vals{[}0{]}{[}0{]}{[}0{]})
print(bcfar\_fort.\_\_doc\_\_)
\# residuals returned in Field dw
bcfar\_fort.bcfar(il, jl, ie, je, itl, itu,       w.vals, P.vals, lv.vals, ev.vals,        x.vals, xc.vals,       gamma,rm,rho0,p0,h0,c0,u0,v0,ca,sa,re,prn,prt,scal,chord,xm,       ym,kvis,       bc,       mode)

\sphinxAtStartPar
print(w.vals{[}0{]}{[}0{]}{[}0{]})


\subsubsection{\sphinxstyleliteralintitle{\sphinxupquote{tests.test\_bcwall}}}
\label{\detokenize{autoapi/tests/test_bcwall/index:module-tests.test_bcwall}}\label{\detokenize{autoapi/tests/test_bcwall/index:tests-test-bcwall}}\label{\detokenize{autoapi/tests/test_bcwall/index::doc}}\index{module@\spxentry{module}!tests.test\_bcwall@\spxentry{tests.test\_bcwall}}\index{tests.test\_bcwall@\spxentry{tests.test\_bcwall}!module@\spxentry{module}}
\sphinxAtStartPar
import bcwall\_fort

\sphinxAtStartPar
print(bcwall\_fort.\_\_doc\_\_)

\sphinxAtStartPar
\# fxn sig
bcwall\_fort.bcwall(ny,ie,itl,itu,w,p,ev,x,kvis,isym,mode)


\subsubsection{\sphinxstyleliteralintitle{\sphinxupquote{tests.test\_dflux}}}
\label{\detokenize{autoapi/tests/test_dflux/index:module-tests.test_dflux}}\label{\detokenize{autoapi/tests/test_dflux/index:tests-test-dflux}}\label{\detokenize{autoapi/tests/test_dflux/index::doc}}\index{module@\spxentry{module}!tests.test\_dflux@\spxentry{tests.test\_dflux}}\index{tests.test\_dflux@\spxentry{tests.test\_dflux}!module@\spxentry{module}}
\sphinxAtStartPar
import dflux\_fort

\sphinxAtStartPar
print(dflux\_fort.\_\_doc\_\_)

\sphinxAtStartPar
\# function signature
dflux\_fort.dflux(ny,il,jl,ie,je,w,p,pori,porj,fw,radi,radj,rfil,vis2,vis4)


\subsubsection{\sphinxstyleliteralintitle{\sphinxupquote{tests.test\_dfluxc}}}
\label{\detokenize{autoapi/tests/test_dfluxc/index:module-tests.test_dfluxc}}\label{\detokenize{autoapi/tests/test_dfluxc/index:tests-test-dfluxc}}\label{\detokenize{autoapi/tests/test_dfluxc/index::doc}}\index{module@\spxentry{module}!tests.test\_dfluxc@\spxentry{tests.test\_dfluxc}}\index{tests.test\_dfluxc@\spxentry{tests.test\_dfluxc}!module@\spxentry{module}}
\sphinxAtStartPar
import dfluxc\_fort

\sphinxAtStartPar
print(dfluxc\_fort.\_\_doc\_\_)

\sphinxAtStartPar
\# here is what the call should look like
dfluxc\_fort.dfluxc(ny,il,jl,w,p,porj,fw,radi,radj,rfil,vis0)


\subsubsection{\sphinxstyleliteralintitle{\sphinxupquote{tests.test\_eflux}}}
\label{\detokenize{autoapi/tests/test_eflux/index:module-tests.test_eflux}}\label{\detokenize{autoapi/tests/test_eflux/index:tests-test-eflux}}\label{\detokenize{autoapi/tests/test_eflux/index::doc}}\index{module@\spxentry{module}!tests.test\_eflux@\spxentry{tests.test\_eflux}}\index{tests.test\_eflux@\spxentry{tests.test\_eflux}!module@\spxentry{module}}

\subsubsection{\sphinxstyleliteralintitle{\sphinxupquote{tests.test\_halo}}}
\label{\detokenize{autoapi/tests/test_halo/index:module-tests.test_halo}}\label{\detokenize{autoapi/tests/test_halo/index:tests-test-halo}}\label{\detokenize{autoapi/tests/test_halo/index::doc}}\index{module@\spxentry{module}!tests.test\_halo@\spxentry{tests.test\_halo}}\index{tests.test\_halo@\spxentry{tests.test\_halo}!module@\spxentry{module}}
\sphinxAtStartPar
import halo\_fort

\sphinxAtStartPar
print(halo\_fort.\_\_doc\_\_)

\sphinxAtStartPar
halo\_fort.halo(il,jl,ie,je,itl,itu,w,p,vol)


\subsubsection{\sphinxstyleliteralintitle{\sphinxupquote{tests.test\_nsflux}}}
\label{\detokenize{autoapi/tests/test_nsflux/index:module-tests.test_nsflux}}\label{\detokenize{autoapi/tests/test_nsflux/index:tests-test-nsflux}}\label{\detokenize{autoapi/tests/test_nsflux/index::doc}}\index{module@\spxentry{module}!tests.test\_nsflux@\spxentry{tests.test\_nsflux}}\index{tests.test\_nsflux@\spxentry{tests.test\_nsflux}!module@\spxentry{module}}
\sphinxAtStartPar
import sys
sys.path.append(“../”)

\sphinxAtStartPar
import nsflux\_fort
import numpy as np
\#from eflux\_arr import eflux
from Field import Field

\sphinxAtStartPar
\# grab grid related parameter
\#G = ws.grid
nx = 4
ny = 10
il = nx+1
jl = ny+1
ie = il+1
je = jl+1
itl = 1
itu = 3
ib = il + 2
jb = jl + 2

\sphinxAtStartPar
\# flow related vars
w = Field({[}ib,jb{]},4) \# state
w.vals = np.array(w.vals + 15*np.random.standard\_normal({[}ib,jb,4{]}),order = ‘f’)
P = Field({[}ib,jb{]}) \# pressure
lv = Field({[}ib,jb{]}) \# laminar viscocity
ev = Field({[}ib,jb{]}) \# eddy viscocity
vw = Field({[}ib,jb{]},4) \# residuals

\sphinxAtStartPar
\# mesh related vars
porI = Field({[}ib,jb{]},2) \# mesh vertices
porI.vals = np.array(porI.vals + 15*np.random.standard\_normal({[}ib,jb,2{]}),order = ‘f’)
porJ = Field({[}ib,jb{]},2) \# mesh centers
porJ.vals = np.array(porJ.vals + 15*np.random.standard\_normal({[}ib,jb,2{]}),order = ‘f’)
xc = Field({[}ib,jb{]},2) \# mesh vertices
xc.vals = np.array(porI.vals + 15*np.random.standard\_normal({[}ib,jb,2{]}),order = ‘f’)
x = Field({[}ib,jb{]},2) \# mesh centers
x.vals = np.array(porJ.vals + 15*np.random.standard\_normal({[}ib,jb,2{]}),order = ‘f’)

\sphinxAtStartPar
\# solver related vars
fw = Field({[}ib,jb{]},4)
radI = Field({[}ib,jb{]},2) \# stability I
radJ = Field({[}ib,jb{]},2) \# stability J

\sphinxAtStartPar
gamma = 1.4
rm = 1.2
scal = 1.8
re = 50000
chord = 2.6
prn = 1000
prt = 10000
mode = 1
rfil = 0.8
vis0 = 0.5
rho0 = 1
p0 = 1;h0 = 1;c0 = 1;u0 = 1;v0 = 1;ca= 1;sa = 1; xm = 1; ym = 1; kvis = 1; bc = 1

\sphinxAtStartPar
print(vw.vals{[}:{]}{[}:{]}{[}0{]})
print(nsflux\_fort.\_\_doc\_\_)
\# residuals returned in Field dw
nsflux\_fort.nsflux(il, jl, ie, je,       w.vals, P.vals, lv.vals, ev.vals,        x.vals, xc.vals,       vw.vals,
\begin{quote}

\sphinxAtStartPar
gamma,rm,scal,re,chord,prn,prt,       rfil)
\end{quote}

\sphinxAtStartPar
print(vw.vals{[}:{]}{[}:{]}{[}0{]})


\chapter{Indices and tables}
\label{\detokenize{index:indices-and-tables}}\begin{itemize}
\item {} 
\sphinxAtStartPar
\DUrole{xref,std,std-ref}{genindex}

\item {} 
\sphinxAtStartPar
\DUrole{xref,std,std-ref}{modindex}

\item {} 
\sphinxAtStartPar
\DUrole{xref,std,std-ref}{search}

\end{itemize}


\renewcommand{\indexname}{Python Module Index}
\begin{sphinxtheindex}
\let\bigletter\sphinxstyleindexlettergroup
\bigletter{a}
\item\relax\sphinxstyleindexentry{airfoil\_map}\sphinxstyleindexpageref{autoapi/airfoil_map/index:\detokenize{module-airfoil_map}}
\item\relax\sphinxstyleindexentry{AirfoilMap}\sphinxstyleindexpageref{autoapi/AirfoilMap/index:\detokenize{module-AirfoilMap}}
\indexspace
\bigletter{b}
\item\relax\sphinxstyleindexentry{BaldwinLomax}\sphinxstyleindexpageref{autoapi/BaldwinLomax/index:\detokenize{module-BaldwinLomax}}
\item\relax\sphinxstyleindexentry{bc\_metric}\sphinxstyleindexpageref{autoapi/bc_metric/index:\detokenize{module-bc_metric}}
\item\relax\sphinxstyleindexentry{bc\_transfer}\sphinxstyleindexpageref{autoapi/bc_transfer/index:\detokenize{module-bc_transfer}}
\item\relax\sphinxstyleindexentry{bcfar}\sphinxstyleindexpageref{autoapi/bcfar/index:\detokenize{module-bcfar}}
\item\relax\sphinxstyleindexentry{bcfar\_wrap}\sphinxstyleindexpageref{autoapi/bcfar_wrap/index:\detokenize{module-bcfar_wrap}}
\item\relax\sphinxstyleindexentry{bcwall}\sphinxstyleindexpageref{autoapi/bcwall/index:\detokenize{module-bcwall}}
\item\relax\sphinxstyleindexentry{bcwall\_wrap}\sphinxstyleindexpageref{autoapi/bcwall_wrap/index:\detokenize{module-bcwall_wrap}}
\item\relax\sphinxstyleindexentry{BoundaryConditioner}\sphinxstyleindexpageref{autoapi/BoundaryConditioner/index:\detokenize{module-BoundaryConditioner}}
\item\relax\sphinxstyleindexentry{BoundaryThickness}\sphinxstyleindexpageref{autoapi/BoundaryThickness/index:\detokenize{module-BoundaryThickness}}
\indexspace
\bigletter{c}
\item\relax\sphinxstyleindexentry{CellCenterWS}\sphinxstyleindexpageref{autoapi/CellCenterWS/index:\detokenize{module-CellCenterWS}}
\item\relax\sphinxstyleindexentry{Contractinator}\sphinxstyleindexpageref{autoapi/Contractinator/index:\detokenize{module-Contractinator}}
\item\relax\sphinxstyleindexentry{coord\_strch\_func}\sphinxstyleindexpageref{autoapi/coord_strch_func/index:\detokenize{module-coord_strch_func}}
\item\relax\sphinxstyleindexentry{Cycle}\sphinxstyleindexpageref{autoapi/Cycle/index:\detokenize{module-Cycle}}
\indexspace
\bigletter{d}
\item\relax\sphinxstyleindexentry{dflux}\sphinxstyleindexpageref{autoapi/dflux/index:\detokenize{module-dflux}}
\item\relax\sphinxstyleindexentry{dflux\_wrap}\sphinxstyleindexpageref{autoapi/dflux_wrap/index:\detokenize{module-dflux_wrap}}
\item\relax\sphinxstyleindexentry{dfluxc}\sphinxstyleindexpageref{autoapi/dfluxc/index:\detokenize{module-dfluxc}}
\item\relax\sphinxstyleindexentry{dfluxc\_wrap}\sphinxstyleindexpageref{autoapi/dfluxc_wrap/index:\detokenize{module-dfluxc_wrap}}
\item\relax\sphinxstyleindexentry{dims\_func}\sphinxstyleindexpageref{autoapi/dims_func/index:\detokenize{module-dims_func}}
\indexspace
\bigletter{e}
\item\relax\sphinxstyleindexentry{eflux}\sphinxstyleindexpageref{autoapi/eflux/index:\detokenize{module-eflux}}
\item\relax\sphinxstyleindexentry{eflux\_wrap}\sphinxstyleindexpageref{autoapi/eflux_wrap/index:\detokenize{module-eflux_wrap}}
\item\relax\sphinxstyleindexentry{Expandinator}\sphinxstyleindexpageref{autoapi/Expandinator/index:\detokenize{module-Expandinator}}
\indexspace
\bigletter{f}
\item\relax\sphinxstyleindexentry{Field}\sphinxstyleindexpageref{autoapi/Field/index:\detokenize{module-Field}}
\item\relax\sphinxstyleindexentry{flo103\_ConvergenceChecker}\sphinxstyleindexpageref{autoapi/flo103_ConvergenceChecker/index:\detokenize{module-flo103_ConvergenceChecker}}
\item\relax\sphinxstyleindexentry{flo103\_PostProcessor}\sphinxstyleindexpageref{autoapi/flo103_PostProcessor/index:\detokenize{module-flo103_PostProcessor}}
\indexspace
\bigletter{g}
\item\relax\sphinxstyleindexentry{geom\_func}\sphinxstyleindexpageref{autoapi/geom_func/index:\detokenize{module-geom_func}}
\item\relax\sphinxstyleindexentry{Grid}\sphinxstyleindexpageref{autoapi/Grid/index:\detokenize{module-Grid}}
\indexspace
\bigletter{h}
\item\relax\sphinxstyleindexentry{halo}\sphinxstyleindexpageref{autoapi/halo/index:\detokenize{module-halo}}
\item\relax\sphinxstyleindexentry{halo\_wrap}\sphinxstyleindexpageref{autoapi/halo_wrap/index:\detokenize{module-halo_wrap}}
\indexspace
\bigletter{i}
\item\relax\sphinxstyleindexentry{ImplicitEuler}\sphinxstyleindexpageref{autoapi/ImplicitEuler/index:\detokenize{module-ImplicitEuler}}
\item\relax\sphinxstyleindexentry{Input}\sphinxstyleindexpageref{autoapi/Input/index:\detokenize{module-Input}}
\item\relax\sphinxstyleindexentry{Integrator}\sphinxstyleindexpageref{autoapi/Integrator/index:\detokenize{module-Integrator}}
\indexspace
\bigletter{m}
\item\relax\sphinxstyleindexentry{mesh\_func}\sphinxstyleindexpageref{autoapi/mesh_func/index:\detokenize{module-mesh_func}}
\item\relax\sphinxstyleindexentry{metric\_func}\sphinxstyleindexpageref{autoapi/metric_func/index:\detokenize{module-metric_func}}
\item\relax\sphinxstyleindexentry{Model}\sphinxstyleindexpageref{autoapi/Model/index:\detokenize{module-Model}}
\item\relax\sphinxstyleindexentry{MultiGrid}\sphinxstyleindexpageref{autoapi/MultiGrid/index:\detokenize{module-MultiGrid}}
\indexspace
\bigletter{n}
\item\relax\sphinxstyleindexentry{NavierStokes}\sphinxstyleindexpageref{autoapi/NavierStokes/index:\detokenize{module-NavierStokes}}
\item\relax\sphinxstyleindexentry{NS\_Airfoil}\sphinxstyleindexpageref{autoapi/NS_Airfoil/index:\detokenize{module-NS_Airfoil}}
\item\relax\sphinxstyleindexentry{nsflux\_wrap}\sphinxstyleindexpageref{autoapi/nsflux_wrap/index:\detokenize{module-nsflux_wrap}}
\indexspace
\bigletter{p}
\item\relax\sphinxstyleindexentry{plot\_mesh\_func}\sphinxstyleindexpageref{autoapi/plot_mesh_func/index:\detokenize{module-plot_mesh_func}}
\indexspace
\bigletter{s}
\item\relax\sphinxstyleindexentry{sangho\_func}\sphinxstyleindexpageref{autoapi/sangho_func/index:\detokenize{module-sangho_func}}
\item\relax\sphinxstyleindexentry{stability}\sphinxstyleindexpageref{autoapi/stability/index:\detokenize{module-stability}}
\item\relax\sphinxstyleindexentry{stability\_fast}\sphinxstyleindexpageref{autoapi/stability_fast/index:\detokenize{module-stability_fast}}
\indexspace
\bigletter{t}
\item\relax\sphinxstyleindexentry{tests}\sphinxstyleindexpageref{autoapi/tests/index:\detokenize{module-tests}}
\item\relax\sphinxstyleindexentry{tests.test\_AirfoilMap}\sphinxstyleindexpageref{autoapi/tests/test_AirfoilMap/index:\detokenize{module-tests.test_AirfoilMap}}
\item\relax\sphinxstyleindexentry{tests.test\_bcfar}\sphinxstyleindexpageref{autoapi/tests/test_bcfar/index:\detokenize{module-tests.test_bcfar}}
\item\relax\sphinxstyleindexentry{tests.test\_bcwall}\sphinxstyleindexpageref{autoapi/tests/test_bcwall/index:\detokenize{module-tests.test_bcwall}}
\item\relax\sphinxstyleindexentry{tests.test\_Contractinator}\sphinxstyleindexpageref{autoapi/tests/test_Contractinator/index:\detokenize{module-tests.test_Contractinator}}
\item\relax\sphinxstyleindexentry{tests.test\_dflux}\sphinxstyleindexpageref{autoapi/tests/test_dflux/index:\detokenize{module-tests.test_dflux}}
\item\relax\sphinxstyleindexentry{tests.test\_dfluxc}\sphinxstyleindexpageref{autoapi/tests/test_dfluxc/index:\detokenize{module-tests.test_dfluxc}}
\item\relax\sphinxstyleindexentry{tests.test\_eflux}\sphinxstyleindexpageref{autoapi/tests/test_eflux/index:\detokenize{module-tests.test_eflux}}
\item\relax\sphinxstyleindexentry{tests.test\_Expandinator}\sphinxstyleindexpageref{autoapi/tests/test_Expandinator/index:\detokenize{module-tests.test_Expandinator}}
\item\relax\sphinxstyleindexentry{tests.test\_Field}\sphinxstyleindexpageref{autoapi/tests/test_Field/index:\detokenize{module-tests.test_Field}}
\item\relax\sphinxstyleindexentry{tests.test\_halo}\sphinxstyleindexpageref{autoapi/tests/test_halo/index:\detokenize{module-tests.test_halo}}
\item\relax\sphinxstyleindexentry{tests.test\_Input}\sphinxstyleindexpageref{autoapi/tests/test_Input/index:\detokenize{module-tests.test_Input}}
\item\relax\sphinxstyleindexentry{tests.test\_NavierStokes}\sphinxstyleindexpageref{autoapi/tests/test_NavierStokes/index:\detokenize{module-tests.test_NavierStokes}}
\item\relax\sphinxstyleindexentry{tests.test\_nsflux}\sphinxstyleindexpageref{autoapi/tests/test_nsflux/index:\detokenize{module-tests.test_nsflux}}
\item\relax\sphinxstyleindexentry{tests.test\_Workspace}\sphinxstyleindexpageref{autoapi/tests/test_Workspace/index:\detokenize{module-tests.test_Workspace}}
\indexspace
\bigletter{v}
\item\relax\sphinxstyleindexentry{Viscosity}\sphinxstyleindexpageref{autoapi/Viscosity/index:\detokenize{module-Viscosity}}
\indexspace
\bigletter{w}
\item\relax\sphinxstyleindexentry{Workspace}\sphinxstyleindexpageref{autoapi/Workspace/index:\detokenize{module-Workspace}}
\end{sphinxtheindex}

\renewcommand{\indexname}{Index}
\printindex
\end{document}